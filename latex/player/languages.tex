% Options for packages loaded elsewhere
\PassOptionsToPackage{unicode}{hyperref}
\PassOptionsToPackage{hyphens}{url}
%
\documentclass[
]{article}
\usepackage{lmodern}
\usepackage{amssymb,amsmath}
\usepackage{ifxetex,ifluatex}
\ifnum 0\ifxetex 1\fi\ifluatex 1\fi=0 % if pdftex
  \usepackage[T1]{fontenc}
  \usepackage[utf8]{inputenc}
  \usepackage{textcomp} % provide euro and other symbols
\else % if luatex or xetex
  \usepackage{unicode-math}
  \defaultfontfeatures{Scale=MatchLowercase}
  \defaultfontfeatures[\rmfamily]{Ligatures=TeX,Scale=1}
\fi
% Use upquote if available, for straight quotes in verbatim environments
\IfFileExists{upquote.sty}{\usepackage{upquote}}{}
\IfFileExists{microtype.sty}{% use microtype if available
  \usepackage[]{microtype}
  \UseMicrotypeSet[protrusion]{basicmath} % disable protrusion for tt fonts
}{}
\makeatletter
\@ifundefined{KOMAClassName}{% if non-KOMA class
  \IfFileExists{parskip.sty}{%
    \usepackage{parskip}
  }{% else
    \setlength{\parindent}{0pt}
    \setlength{\parskip}{6pt plus 2pt minus 1pt}}
}{% if KOMA class
  \KOMAoptions{parskip=half}}
\makeatother
\usepackage{xcolor}
\IfFileExists{xurl.sty}{\usepackage{xurl}}{} % add URL line breaks if available
\IfFileExists{bookmark.sty}{\usepackage{bookmark}}{\usepackage{hyperref}}
\hypersetup{
  hidelinks,
  pdfcreator={LaTeX via pandoc}}
\urlstyle{same} % disable monospaced font for URLs
\usepackage{longtable,booktabs}
% Correct order of tables after \paragraph or \subparagraph
\usepackage{etoolbox}
\makeatletter
\patchcmd\longtable{\par}{\if@noskipsec\mbox{}\fi\par}{}{}
\makeatother
% Allow footnotes in longtable head/foot
\IfFileExists{footnotehyper.sty}{\usepackage{footnotehyper}}{\usepackage{footnote}}
\makesavenoteenv{longtable}
\setlength{\emergencystretch}{3em} % prevent overfull lines
\providecommand{\tightlist}{%
  \setlength{\itemsep}{0pt}\setlength{\parskip}{0pt}}
\setcounter{secnumdepth}{-\maxdimen} % remove section numbering

\date{}

\begin{document}

\hypertarget{languages}{%
\section{Languages}\label{languages}}

Your race indicates the languages your character can speak by default,
and your background might give you access to one or more additional
languages of your choice. Note these languages on your character sheet.

Choose your languages from the Standard Languages table, or choose one
that is common in your campaign. With your GM's permission, you can
instead choose a language from the Exotic

Languages table or a secret language, such as thieves' cant or the
tongue of druids.

Some of these languages are actually families of languages with many
dialects. For example, the Primordial language includes the Auran,
Aquan, Ignan, and Terran dialects, one for each of the four elemental
planes. Creatures that speak different dialects of the same language can
communicate with one another.

\hypertarget{standard-languages}{%
\paragraph{Standard Languages}\label{standard-languages}}

\begin{longtable}[]{@{}lll@{}}
\toprule
Language & Typical Speakers & Script\tabularnewline
\midrule
\endhead
Common & Humans & Common\tabularnewline
Dwarvish & Dwarves & Dwarvish\tabularnewline
Elvish & Elves & Elvish\tabularnewline
Giant & Ogres, giants & Dwarvish\tabularnewline
Gnomish & Gnomes & Dwarvish\tabularnewline
Goblin & Goblinoids & Dwarvish\tabularnewline
Halfling & Halflings & Common\tabularnewline
Orc & Orcs & Dwarvish\tabularnewline
\bottomrule
\end{longtable}

\hypertarget{exotic-languages}{%
\paragraph{Exotic Languages}\label{exotic-languages}}

\begin{longtable}[]{@{}lll@{}}
\toprule
Language & Typical Speakers & Script\tabularnewline
\midrule
\endhead
Abyssal & Demons & Infernal\tabularnewline
Celestial & Celestials & Celestial\tabularnewline
Draconic & Dragons, dragonborn & Draconic\tabularnewline
Deep Speech & Aboleths, cloakers & ---\tabularnewline
Infernal & Devils & Infernal\tabularnewline
Primordial & Elementals & Dwarvish\tabularnewline
Sylvan & Fey creatures & Elvish\tabularnewline
Undercommon & Underworld traders & Elvish\tabularnewline
\bottomrule
\end{longtable}

\end{document}
