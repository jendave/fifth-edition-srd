% Options for packages loaded elsewhere
\PassOptionsToPackage{unicode}{hyperref}
\PassOptionsToPackage{hyphens}{url}
%
\documentclass[
]{article}
\usepackage{lmodern}
\usepackage{amssymb,amsmath}
\usepackage{ifxetex,ifluatex}
\ifnum 0\ifxetex 1\fi\ifluatex 1\fi=0 % if pdftex
  \usepackage[T1]{fontenc}
  \usepackage[utf8]{inputenc}
  \usepackage{textcomp} % provide euro and other symbols
\else % if luatex or xetex
  \usepackage{unicode-math}
  \defaultfontfeatures{Scale=MatchLowercase}
  \defaultfontfeatures[\rmfamily]{Ligatures=TeX,Scale=1}
\fi
% Use upquote if available, for straight quotes in verbatim environments
\IfFileExists{upquote.sty}{\usepackage{upquote}}{}
\IfFileExists{microtype.sty}{% use microtype if available
  \usepackage[]{microtype}
  \UseMicrotypeSet[protrusion]{basicmath} % disable protrusion for tt fonts
}{}
\makeatletter
\@ifundefined{KOMAClassName}{% if non-KOMA class
  \IfFileExists{parskip.sty}{%
    \usepackage{parskip}
  }{% else
    \setlength{\parindent}{0pt}
    \setlength{\parskip}{6pt plus 2pt minus 1pt}}
}{% if KOMA class
  \KOMAoptions{parskip=half}}
\makeatother
\usepackage{xcolor}
\IfFileExists{xurl.sty}{\usepackage{xurl}}{} % add URL line breaks if available
\IfFileExists{bookmark.sty}{\usepackage{bookmark}}{\usepackage{hyperref}}
\hypersetup{
  hidelinks,
  pdfcreator={LaTeX via pandoc}}
\urlstyle{same} % disable monospaced font for URLs
\setlength{\emergencystretch}{3em} % prevent overfull lines
\providecommand{\tightlist}{%
  \setlength{\itemsep}{0pt}\setlength{\parskip}{0pt}}
\setcounter{secnumdepth}{-\maxdimen} % remove section numbering

\date{}

\begin{document}

\hypertarget{feats}{%
\section{Feats}\label{feats}}

A feat represents a talent or an area of expertise that gives a
character special capabilities. It embodies training, experience, and
abilities beyond what a class provides.

At certain levels, your class gives you the Ability Score Improvement
feature. Using the optional feats rule, you can forgo taking that
feature to take a feat of your choice instead. You can take each feat
only once, unless the feat's description says otherwise.

You must meet any prerequisite specified in a feat to take that feat. If
you ever lose a feat's prerequisite, you can't use that feat until you
regain the prerequisite. For example, the Grappler feat requires you to
have a Strength of 13 or higher. If your Strength is reduced below 13
somehow---perhaps by a withering curse---you can't benefit from the
Grappler feat until your Strength is restored.

\#\#Grappler

\textbf{Prerequisite:} Strength 13 or higher

You've developed the skills necessary to hold your own in close-quarters
grappling. You gain the following benefits:

\begin{itemize}
\item
  You have advantage on attack rolls against a creature you are
  grappling.
\item
  You can use your action to try to pin a creature grappled by you. To
  do so, make another grapple check. If you succeed, you and the
  creature are both restrained until the grapple ends.
\end{itemize}

\end{document}
