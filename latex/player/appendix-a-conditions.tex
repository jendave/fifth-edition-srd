% Options for packages loaded elsewhere
\PassOptionsToPackage{unicode}{hyperref}
\PassOptionsToPackage{hyphens}{url}
%
\documentclass[
]{article}
\usepackage{lmodern}
\usepackage{amssymb,amsmath}
\usepackage{ifxetex,ifluatex}
\ifnum 0\ifxetex 1\fi\ifluatex 1\fi=0 % if pdftex
  \usepackage[T1]{fontenc}
  \usepackage[utf8]{inputenc}
  \usepackage{textcomp} % provide euro and other symbols
\else % if luatex or xetex
  \usepackage{unicode-math}
  \defaultfontfeatures{Scale=MatchLowercase}
  \defaultfontfeatures[\rmfamily]{Ligatures=TeX,Scale=1}
\fi
% Use upquote if available, for straight quotes in verbatim environments
\IfFileExists{upquote.sty}{\usepackage{upquote}}{}
\IfFileExists{microtype.sty}{% use microtype if available
  \usepackage[]{microtype}
  \UseMicrotypeSet[protrusion]{basicmath} % disable protrusion for tt fonts
}{}
\makeatletter
\@ifundefined{KOMAClassName}{% if non-KOMA class
  \IfFileExists{parskip.sty}{%
    \usepackage{parskip}
  }{% else
    \setlength{\parindent}{0pt}
    \setlength{\parskip}{6pt plus 2pt minus 1pt}}
}{% if KOMA class
  \KOMAoptions{parskip=half}}
\makeatother
\usepackage{xcolor}
\IfFileExists{xurl.sty}{\usepackage{xurl}}{} % add URL line breaks if available
\IfFileExists{bookmark.sty}{\usepackage{bookmark}}{\usepackage{hyperref}}
\hypersetup{
  hidelinks,
  pdfcreator={LaTeX via pandoc}}
\urlstyle{same} % disable monospaced font for URLs
\usepackage{longtable,booktabs}
% Correct order of tables after \paragraph or \subparagraph
\usepackage{etoolbox}
\makeatletter
\patchcmd\longtable{\par}{\if@noskipsec\mbox{}\fi\par}{}{}
\makeatother
% Allow footnotes in longtable head/foot
\IfFileExists{footnotehyper.sty}{\usepackage{footnotehyper}}{\usepackage{footnote}}
\makesavenoteenv{longtable}
\setlength{\emergencystretch}{3em} % prevent overfull lines
\providecommand{\tightlist}{%
  \setlength{\itemsep}{0pt}\setlength{\parskip}{0pt}}
\setcounter{secnumdepth}{-\maxdimen} % remove section numbering

\date{}

\begin{document}

\hypertarget{appendix-ph-a-conditions}{%
\section{Appendix PH-A: Conditions}\label{appendix-ph-a-conditions}}

Conditions alter a creature's capabilities in a variety of ways and can
arise as a result of a spell, a class feature, a monster's attack, or
other effect. Most conditions, such as blinded, are impairments, but a
few, such as invisible, can be advantageous.

A condition lasts either until it is countered (the prone condition is
countered by standing up, for example) or for a duration specified by
the effect that imposed the condition.

If multiple effects impose the same condition on a creature, each
instance of the condition has its own duration, but the condition's
effects don't get worse. A creature either has a condition or doesn't.

The following definitions specify what happens to a creature while it is
subjected to a condition.

\hypertarget{blinded}{%
\paragraph{Blinded}\label{blinded}}

\begin{itemize}
\tightlist
\item
  A blinded creature can't see and automatically fails any ability check
  that requires sight.
\item
  Attack rolls against the creature have advantage, and the creature's
  attack rolls have disadvantage.
\end{itemize}

\hypertarget{charmed}{%
\paragraph{Charmed}\label{charmed}}

\begin{itemize}
\tightlist
\item
  A charmed creature can't attack the charmer or target the charmer with
  harmful abilities or magical effects.
\item
  The charmer has advantage on any ability check to interact socially
  with the creature.
\end{itemize}

\hypertarget{deafened}{%
\paragraph{Deafened}\label{deafened}}

\begin{itemize}
\tightlist
\item
  A deafened creature can't hear and automatically fails any ability
  check that requires hearing.
\end{itemize}

\begin{quote}
\mbox{}%
\hypertarget{exhaustion}{%
\paragraph{Exhaustion}\label{exhaustion}}

Some special abilities and environmental hazards, such as starvation and
the longbterm effects of freezing or scorching temperatures, can lead to
a special condition called exhaustion. Exhaustion is measured in six
levels. An effect can give a creature one or more levels of exhaustion,
as specified in the effect's description.

\begin{longtable}[]{@{}ll@{}}
\toprule
Level & Effect\tabularnewline
\midrule
\endhead
1 & Disadvantage on ability checks\tabularnewline
2 & Speed halved\tabularnewline
3 & Disadvantage on attack rolls and saving throws\tabularnewline
4 & Hit point maximum halved\tabularnewline
5 & Speed reduced to 0\tabularnewline
6 & Death\tabularnewline
\bottomrule
\end{longtable}

If an already exhausted creature suffers another effect that causes
exhaustion, its current level of exhaustion increases by the amount
specified in the effect's description.

A creature suffers the effect of its current level of exhaustion as well
as all lower levels. For example, a creature suffering level 2
exhaustion has its speed halved and has disadvantage on ability checks.

An effect that removes exhaustion reduces its level as specified in the
effect's description, with all exhaustion effects ending if a creature's
exhaustion level is reduced below 1.

Finishing a long rest reduces a creature's exhaustion level by 1,
provided that the creature has also ingested some food and drink.
\end{quote}

\hypertarget{frightened}{%
\paragraph{Frightened}\label{frightened}}

\begin{itemize}
\tightlist
\item
  A frightened creature has disadvantage on ability checks and attack
  rolls while the source of its fear is within line of sight.
\item
  The creature can't willingly move closer to the source of its fear.
\end{itemize}

\hypertarget{grappled}{%
\paragraph{Grappled}\label{grappled}}

\begin{itemize}
\tightlist
\item
  A grappled creature's speed becomes 0, and it can't benefit from any
  bonus to its speed.
\item
  The condition ends if the grappler is incapacitated (see the
  condition).
\item
  The condition also ends if an effect removes the grappled creature
  from the reach of the grappler or grappling effect, such as when a
  creature is hurled away by the \emph{thunderwave} spell.
\end{itemize}

\hypertarget{incapacitated}{%
\paragraph{Incapacitated}\label{incapacitated}}

\begin{itemize}
\tightlist
\item
  An incapacitated creature can't take actions or reactions.
\end{itemize}

\hypertarget{invisible}{%
\paragraph{Invisible}\label{invisible}}

\begin{itemize}
\item
  An invisible creature is impossible to see without the aid of magic or
  a special sense. For the purpose of hiding, the creature is heavily
  obscured. The creature's location can be detected by any noise it
  makes or any tracks it leaves.
\item
  Attack rolls against the creature have disadvantage, and the
  creature's attack rolls have advantage.
\end{itemize}

\hypertarget{paralyzed}{%
\paragraph{Paralyzed}\label{paralyzed}}

\begin{itemize}
\tightlist
\item
  A paralyzed creature is incapacitated (see the condition) and can't
  move or speak.
\item
  The creature automatically fails Strength and Dexterity saving throws.
\item
  Attack rolls against the creature have advantage.
\item
  Any attack that hits the creature is a critical hit if the attacker is
  within 5 feet of the creature.
\end{itemize}

\hypertarget{petrified}{%
\paragraph{Petrified}\label{petrified}}

\begin{itemize}
\tightlist
\item
  A petrified creature is transformed, along with any nonmagical object
  it is wearing or carrying, into a solid inanimate substance (usually
  stone). Its weight increases by a factor of ten, and it ceases aging.
\item
  The creature is incapacitated (see the condition), can't move or
  speak, and is unaware of its surroundings.
\item
  Attack rolls against the creature have advantage.
\item
  The creature automatically fails Strength and Dexterity saving throws.
\item
  The creature has resistance to all damage.
\item
  The creature is immune to poison and disease, although a poison or
  disease already in its system is suspended, not neutralized.
\end{itemize}

\hypertarget{poisoned}{%
\paragraph{Poisoned}\label{poisoned}}

\begin{itemize}
\tightlist
\item
  A poisoned creature has disadvantage on attack rolls and ability
  checks.
\end{itemize}

\hypertarget{prone}{%
\paragraph{Prone}\label{prone}}

\begin{itemize}
\tightlist
\item
  A prone creature's only movement option is to crawl, unless it stands
  up and thereby ends the condition.
\item
  The creature has disadvantage on attack rolls.
\item
  An attack roll against the creature has advantage if the attacker is
  within 5 feet of the creature. Otherwise, the attack roll has
  disadvantage.
\end{itemize}

\hypertarget{restrained}{%
\paragraph{Restrained}\label{restrained}}

\begin{itemize}
\tightlist
\item
  A restrained creature's speed becomes 0, and it can't benefit from any
  bonus to its speed.
\item
  Attack rolls against the creature have advantage, and the creature's
  attack rolls have disadvantage.
\item
  The creature has disadvantage on Dexterity saving throws.
\end{itemize}

\hypertarget{stunned}{%
\paragraph{Stunned}\label{stunned}}

\begin{itemize}
\tightlist
\item
  A stunned creature is incapacitated (see the condition), can't move,
  and can speak only falteringly.
\item
  The creature automatically fails Strength and Dexterity saving throws.
\item
  Attack rolls against the creature have advantage.
\end{itemize}

\hypertarget{unconscious}{%
\paragraph{Unconscious}\label{unconscious}}

\begin{itemize}
\tightlist
\item
  An unconscious creature is incapacitated (see the condition), can't
  move or speak, and is unaware of its surroundings.
\item
  The creature drops whatever it's holding and falls prone.
\item
  The creature automatically fails Strength and Dexterity saving throws.
\item
  Attack rolls against the creature have advantage.
\item
  Any attack that hits the creature is a critical hit if the attacker is
  within 5 feet of the creature.
\end{itemize}

\end{document}
