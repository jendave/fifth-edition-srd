% Options for packages loaded elsewhere
\PassOptionsToPackage{unicode}{hyperref}
\PassOptionsToPackage{hyphens}{url}
%
\documentclass[
]{article}
\usepackage{lmodern}
\usepackage{amssymb,amsmath}
\usepackage{ifxetex,ifluatex}
\ifnum 0\ifxetex 1\fi\ifluatex 1\fi=0 % if pdftex
  \usepackage[T1]{fontenc}
  \usepackage[utf8]{inputenc}
  \usepackage{textcomp} % provide euro and other symbols
\else % if luatex or xetex
  \usepackage{unicode-math}
  \defaultfontfeatures{Scale=MatchLowercase}
  \defaultfontfeatures[\rmfamily]{Ligatures=TeX,Scale=1}
\fi
% Use upquote if available, for straight quotes in verbatim environments
\IfFileExists{upquote.sty}{\usepackage{upquote}}{}
\IfFileExists{microtype.sty}{% use microtype if available
  \usepackage[]{microtype}
  \UseMicrotypeSet[protrusion]{basicmath} % disable protrusion for tt fonts
}{}
\makeatletter
\@ifundefined{KOMAClassName}{% if non-KOMA class
  \IfFileExists{parskip.sty}{%
    \usepackage{parskip}
  }{% else
    \setlength{\parindent}{0pt}
    \setlength{\parskip}{6pt plus 2pt minus 1pt}}
}{% if KOMA class
  \KOMAoptions{parskip=half}}
\makeatother
\usepackage{xcolor}
\IfFileExists{xurl.sty}{\usepackage{xurl}}{} % add URL line breaks if available
\IfFileExists{bookmark.sty}{\usepackage{bookmark}}{\usepackage{hyperref}}
\hypersetup{
  hidelinks,
  pdfcreator={LaTeX via pandoc}}
\urlstyle{same} % disable monospaced font for URLs
\setlength{\emergencystretch}{3em} % prevent overfull lines
\providecommand{\tightlist}{%
  \setlength{\itemsep}{0pt}\setlength{\parskip}{0pt}}
\setcounter{secnumdepth}{-\maxdimen} % remove section numbering

\date{}

\begin{document}

\hypertarget{backgrounds}{%
\section{Backgrounds}\label{backgrounds}}

Every story has a beginning. Your character's background reveals where
you came from, how you became an adventurer, and your place in the
world. Your fighter might have been a courageous knight or a grizzled
soldier. Your wizard could have been a sage or an artisan. Your rogue
might have gotten by as a guild thief or commanded audiences as a
jester.

Choosing a background provides you with important story cues about your
character's identity. The most important question to ask about your
background is \emph{what changed}? Why did you stop doing whatever your
background describes and start adventuring? Where did you get the money
to purchase your starting gear, or, if you come from a wealthy
background, why don't you have \emph{more} money? How did you learn the
skills of your class?

What sets you apart from ordinary people who share your background?

The sample background presented here provides both concrete benefits
(features, proficiencies, and languages) and roleplaying suggestions.

\hypertarget{proficiencies}{%
\subsubsection{Proficiencies}\label{proficiencies}}

Each background gives a character proficiency in two skills (described
in ``Using Ability Scores'').

In addition, most backgrounds give a character proficiency with one or
more tools (detailed in ``Equipment'').

If a character would gain the same proficiency from two different
sources, he or she can choose a different proficiency of the same kind
(skill or tool) instead.

\hypertarget{languages}{%
\subsubsection{Languages}\label{languages}}

Some backgrounds also allow characters to learn additional languages
beyond those given by race. See ``Languages.''

\hypertarget{equipment}{%
\subsubsection{Equipment}\label{equipment}}

Each background provides a package of starting equipment. If you use the
optional rule to spend coin on gear, you do not receive the starting
equipment from your background.

\hypertarget{suggested-characteristics}{%
\subsubsection{Suggested
Characteristics}\label{suggested-characteristics}}

A background contains suggested personal characteristics based on your
background. You can pick characteristics, roll dice to determine them
randomly, or use the suggestions as inspiration for characteristics of
your own creation.

\hypertarget{customizing-a-background}{%
\subsubsection{Customizing a
Background}\label{customizing-a-background}}

You might want to tweak some of the features of a background so it
better fits your character or the campaign setting. To customize a
background, you can replace one feature with any other one, choose any
two skills, and choose a total of two tool proficiencies or languages
from the sample backgrounds. You can either use the equipment package
from your background or spend coin on gear as described in the equipment
section. (If you spend coin, you can't also take the equipment package
suggested for your class.) Finally, choose two personality traits, one
ideal, one bond, and one flaw. If you can't find a feature that matches
your desired background, work with your GM to create one.

\hypertarget{acolyte}{%
\subsection{Acolyte}\label{acolyte}}

You have spent your life in the service of a temple to a specific god or
pantheon of gods. You act as an intermediary between the realm of the
holy and the mortal world, performing sacred rites and offering
sacrifices in order to conduct worshipers into the presence of the
divine. You are not necessarily a cleric---performing sacred rites is
not the same thing as channeling divine power.

Not for resale. Permission granted to print or photocopy this document
for personal use only. System Reference Document 5.0 60

Choose a god, a pantheon of gods, or some other quasi-divine being from
among those listed in "Fantasy-Historical Pantheons" or those specified
by your GM, and work with your GM to detail the nature of your religious
service. Were you a lesser functionary in a temple, raised from
childhood to assist the priests in the sacred rites? Or were you a high
priest who suddenly experienced a call to serve your god in a different
way? Perhaps you were the leader of a small cult outside of any
established temple structure, or even an occult group that served a
fiendish master that you now deny.

\textbf{Skill Proficiencies:} Insight, Religion

\textbf{Languages:} Two of your choice

\textbf{Equipment:} A holy symbol (a gift to you when you entered the
priesthood), a prayer book or prayer wheel, 5 sticks of incense,
vestments, a set of common clothes, and a pouch containing 15 gp

\hypertarget{feature-shelter-of-the-faithful}{%
\subsubsection{Feature: Shelter of the
Faithful}\label{feature-shelter-of-the-faithful}}

As an acolyte, you command the respect of those who share your faith,
and you can perform the religious ceremonies of your deity. You and your
adventuring companions can expect to receive free healing and care at a
temple, shrine, or other established presence of your faith, though you
must provide any material components needed for spells. Those who share
your religion will support you (but only you) at a modest lifestyle.

You might also have ties to a specific temple dedicated to your chosen
deity or pantheon, and you have a residence there. This could be the
temple where you used to serve, if you remain on good terms with it, or
a temple where you have found a new home. While near your temple, you
can call upon the priests for assistance, provided the assistance you
ask for is not hazardous and you remain in good standing with your
temple.

\hypertarget{suggested-characteristics-1}{%
\subsubsection{Suggested
Characteristics}\label{suggested-characteristics-1}}

Acolytes are shaped by their experience in temples or other religious
communities. Their study of the history and tenets of their faith and
their relationships to temples, shrines, or hierarchies affect their
mannerisms and ideals. Their flaws might be some hidden hypocrisy or
heretical idea, or an ideal or bond taken to an extreme.

\hypertarget{d8-personality-trait}{%
\paragraph{d8 Personality Trait}\label{d8-personality-trait}}

\begin{enumerate}
\def\labelenumi{\arabic{enumi}.}
\tightlist
\item
  I idolize a particular hero of my faith, and constantly refer to that
  person's deeds and example.
\item
  I can find common ground between the fiercest enemies, empathizing
  with them and always working toward peace.
\item
  I see omens in every event and action. The gods try to speak to us, we
  just need to listen
\item
  Nothing can shake my optimistic attitude.
\item
  I quote (or misquote) sacred texts and proverbs in almost every
  situation.
\item
  I am tolerant (or intolerant) of other faiths and respect (or condemn)
  the worship of other gods.
\item
  I've enjoyed fine food, drink, and high society among my temple's
  elite. Rough living grates on me.
\item
  I've spent so long in the temple that I have little practical
  experience dealing with people in the outside world.
\end{enumerate}

\hypertarget{d6-ideal}{%
\paragraph{d6 Ideal}\label{d6-ideal}}

\begin{enumerate}
\def\labelenumi{\arabic{enumi}.}
\tightlist
\item
  Tradition. The ancient traditions of worship and sacrifice must be
  preserved and upheld. (Lawful)
\item
  Charity. I always try to help those in need, no matter what the
  personal cost. (Good)
\item
  Change. We must help bring about the changes the gods are constantly
  working in the world. (Chaotic)
\item
  Power. I hope to one day rise to the top of my faith's religious
  hierarchy. (Lawful)
\item
  Faith. I trust that my deity will guide my actions. I have faith that
  if I work hard, things will go well. (Lawful)
\item
  Aspiration. I seek to prove myself worthy of my god's favor by
  matching my actions against his or her teachings. (Any)
\end{enumerate}

\hypertarget{d6-bond}{%
\paragraph{d6 Bond}\label{d6-bond}}

\begin{enumerate}
\def\labelenumi{\arabic{enumi}.}
\tightlist
\item
  I would die to recover an ancient relic of my faith that was lost long
  ago.
\item
  I will someday get revenge on the corrupt temple hierarchy who branded
  me a heretic.
\item
  I owe my life to the priest who took me in when my parents died.
\item
  Everything I do is for the common people.
\item
  I will do anything to protect the temple where I served.
\item
  I seek to preserve a sacred text that my enemies consider heretical
  and seek to destroy.
\end{enumerate}

\hypertarget{d6-flaw}{%
\paragraph{d6 Flaw}\label{d6-flaw}}

\begin{enumerate}
\def\labelenumi{\arabic{enumi}.}
\tightlist
\item
  I judge others harshly, and myself even more severely.
\item
  I put too much trust in those who wield power within my temple's
  hierarchy.
\item
  My piety sometimes leads me to blindly trust those that profess faith
  in my god.
\item
  I am inflexible in my thinking.
\item
  I am suspicious of strangers and expect the worst of them.
\item
  Once I pick a goal, I become obsessed with it to the detriment of
  everything else in my life.
\end{enumerate}

\end{document}
