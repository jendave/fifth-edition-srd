% Options for packages loaded elsewhere
\PassOptionsToPackage{unicode}{hyperref}
\PassOptionsToPackage{hyphens}{url}
%
\documentclass[
]{article}
\usepackage{lmodern}
\usepackage{amssymb,amsmath}
\usepackage{ifxetex,ifluatex}
\ifnum 0\ifxetex 1\fi\ifluatex 1\fi=0 % if pdftex
  \usepackage[T1]{fontenc}
  \usepackage[utf8]{inputenc}
  \usepackage{textcomp} % provide euro and other symbols
\else % if luatex or xetex
  \usepackage{unicode-math}
  \defaultfontfeatures{Scale=MatchLowercase}
  \defaultfontfeatures[\rmfamily]{Ligatures=TeX,Scale=1}
\fi
% Use upquote if available, for straight quotes in verbatim environments
\IfFileExists{upquote.sty}{\usepackage{upquote}}{}
\IfFileExists{microtype.sty}{% use microtype if available
  \usepackage[]{microtype}
  \UseMicrotypeSet[protrusion]{basicmath} % disable protrusion for tt fonts
}{}
\makeatletter
\@ifundefined{KOMAClassName}{% if non-KOMA class
  \IfFileExists{parskip.sty}{%
    \usepackage{parskip}
  }{% else
    \setlength{\parindent}{0pt}
    \setlength{\parskip}{6pt plus 2pt minus 1pt}}
}{% if KOMA class
  \KOMAoptions{parskip=half}}
\makeatother
\usepackage{xcolor}
\IfFileExists{xurl.sty}{\usepackage{xurl}}{} % add URL line breaks if available
\IfFileExists{bookmark.sty}{\usepackage{bookmark}}{\usepackage{hyperref}}
\hypersetup{
  hidelinks,
  pdfcreator={LaTeX via pandoc}}
\urlstyle{same} % disable monospaced font for URLs
\usepackage{longtable,booktabs}
% Correct order of tables after \paragraph or \subparagraph
\usepackage{etoolbox}
\makeatletter
\patchcmd\longtable{\par}{\if@noskipsec\mbox{}\fi\par}{}{}
\makeatother
% Allow footnotes in longtable head/foot
\IfFileExists{footnotehyper.sty}{\usepackage{footnotehyper}}{\usepackage{footnote}}
\makesavenoteenv{longtable}
\setlength{\emergencystretch}{3em} % prevent overfull lines
\providecommand{\tightlist}{%
  \setlength{\itemsep}{0pt}\setlength{\parskip}{0pt}}
\setcounter{secnumdepth}{-\maxdimen} % remove section numbering

\date{}

\begin{document}

\hypertarget{races}{%
\section{Races}\label{races}}

\hypertarget{racial-traits}{%
\subsubsection{Racial Traits}\label{racial-traits}}

The description of each race includes racial traits that are common to
members of that race. The following entries appear among the traits of
most races.

\hypertarget{ability-score-increase}{%
\paragraph{Ability Score Increase}\label{ability-score-increase}}

Every race increases one or more of a character's ability scores.

\hypertarget{age}{%
\paragraph{Age}\label{age}}

The age entry notes the age when a member of the race is considered an
adult, as well as the race's expected lifespan. This information can
help you decide how old your character is at the start of the game. You
can choose any age for your character, which could provide an
explanation for some of your ability scores. For example, if you play a
young or very old character, your age could explain a particularly low
Strength or Constitution score, while advanced age could account for a
high Intelligence or Wisdom.

\hypertarget{alignment}{%
\paragraph{Alignment}\label{alignment}}

Most races have tendencies toward certain alignments, described in this
entry. These are not binding for player characters, but considering why
your dwarf is chaotic, for example, in defiance of lawful dwarf society
can help you better define your character.

\hypertarget{size}{%
\paragraph{Size}\label{size}}

Characters of most races are Medium, a size category including creatures
that are roughly 4 to 8 feet tall. Members of a few races are Small
(between 2 and 4 feet tall), which means that certain rules of the game
affect them differently. The most important of these rules is that Small
characters have trouble wielding heavy weapons, as explained in
``Equipment.''

\hypertarget{speed}{%
\paragraph{Speed}\label{speed}}

Your speed determines how far you can move when traveling (
``Adventuring'') and fighting (``Combat'').

\hypertarget{languages}{%
\paragraph{Languages}\label{languages}}

By virtue of your race, your character can speak, read, and write
certain languages.

\hypertarget{subraces}{%
\paragraph{Subraces}\label{subraces}}

Some races have subraces. Members of a subrace have the traits of the
parent race in addition to the traits specified for their subrace.
Relationships among subraces vary significantly from race to race and
world to world.

\hypertarget{dwarf}{%
\subsection{Dwarf}\label{dwarf}}

\hypertarget{dwarf-traits}{%
\subsubsection{Dwarf Traits}\label{dwarf-traits}}

Your dwarf character has an assortment of inborn abilities, part and
parcel of dwarven nature.

\textbf{Ability Score Increase.} Your Constitution score increases by 2.

\textbf{Age.} Dwarves mature at the same rate as humans, but they're
considered young until they reach the age of 50. On average, they live
about 350 years.

\textbf{Alignment.} Most dwarves are lawful, believing firmly in the
benefits of a well-ordered society. They tend toward good as well, with
a strong sense of fair play and a belief that everyone deserves to share
in the benefits of a just order.

\textbf{Size.} Dwarves stand between 4 and 5 feet tall and average about
150 pounds. Your size is Medium.

\textbf{Speed.} Your base walking speed is 25 feet. Your speed is not
reduced by wearing heavy armor.

\textbf{Darkvision.} Accustomed to life underground, you have superior
vision in dark and dim conditions. You can see in dim light within 60
feet of you as if it were bright light, and in darkness as if it were
dim light. You can't discern color in darkness, only shades of gray.

\textbf{Dwarven Resilience.} You have advantage on saving throws against
poison, and you have resistance against poison damage.

\textbf{Dwarven Combat Training.} You have proficiency with the
battleaxe, handaxe, light hammer, and warhammer.

\textbf{Tool Proficiency.} You gain proficiency with the artisan's tools
of your choice: smith's tools, brewer's supplies, or mason's tools.

\textbf{Stonecunning.} Whenever you make an Intelligence (History) check
related to the origin of stonework, you are considered proficient in the
History skill and add double your proficiency bonus to the check,
instead of your normal proficiency bonus.

\textbf{Languages.} You can speak, read, and write Common and Dwarvish.
Dwarvish is full of hard consonants and guttural sounds, and those
characteristics spill over into whatever other language a dwarf might
speak.

\hypertarget{hill-dwarf}{%
\paragraph{Hill Dwarf}\label{hill-dwarf}}

As a hill dwarf, you have keen senses, deep intuition, and remarkable
resilience.

\textbf{Ability Score Increase.} Your Wisdom score increases by 1.

\textbf{Dwarven Toughness.} Your hit point maximum increases by 1, and
it increases by 1 every time you gain a level.

\hypertarget{elf}{%
\subsection{Elf}\label{elf}}

\hypertarget{elf-traits}{%
\subsubsection{Elf Traits}\label{elf-traits}}

Your elf character has a variety of natural abilities, the result of
thousands of years of elven refinement.

\textbf{Ability Score Increase.} Your Dexterity score increases by 2.

\textbf{Age.} Although elves reach physical maturity at about the same
age as humans, the elven understanding of adulthood goes beyond physical
growth to encompass worldly experience. An elf typically claims
adulthood and an adult name around the age of 100 and can live to be 750
years old.

\textbf{Alignment.} Elves love freedom, variety, and self- expression,
so they lean strongly toward the gentler aspects of chaos. They value
and protect others' freedom as well as their own, and they are more
often good than not.

\textbf{Size.} Elves range from under 5 to over 6 feet tall and have
slender builds. Your size is Medium.

\textbf{Speed.} Your base walking speed is 30 feet.

\textbf{Darkvision.} Accustomed to twilit forests and the night sky, you
have superior vision in dark and dim conditions. You can see in dim
light within 60 feet of you as if it were bright light, and in darkness
as if it were dim light. You can't discern color in darkness, only
shades of gray.

\textbf{Keen Senses.} You have proficiency in the Perception skill.

\textbf{Fey Ancestry.} You have advantage on saving throws against being
charmed, and magic can't put you to sleep.

\textbf{Trance.} Elves don't need to sleep. Instead, they meditate
deeply, remaining semiconscious, for 4 hours a day. (The Common word for
such meditation is ``trance.'') While meditating, you can dream after a
fashion; such dreams are actually mental exercises that have become
reflexive through years of practice.

After resting in this way, you gain the same benefit that a human does
from 8 hours of sleep.

\textbf{Languages.} You can speak, read, and write Common and Elvish.
Elvish is fluid, with subtle intonations and intricate grammar. Elven
literature is rich and varied, and their songs and poems are famous
among other races. Many bards learn their language so they can add
Elvish ballads to their repertoires.

\hypertarget{high-elf}{%
\paragraph{High Elf}\label{high-elf}}

As a high elf, you have a keen mind and a mastery of at least the basics
of magic. In many fantasy gaming worlds, there are two kinds of high
elves. One type is haughty and reclusive, believing themselves to be
superior to non-elves and even other elves. The other type is more
common and more friendly, and often encountered among humans and other
races.

\textbf{Ability Score Increase.} Your Intelligence score increases by 1.

\textbf{Elf Weapon Training.} You have proficiency with the longsword,
shortsword, shortbow, and longbow.

\textbf{Cantrip.} You know one cantrip of your choice from the wizard
spell list. Intelligence is your spellcasting ability for it.

\textbf{Extra Language.} You can speak, read, and write one extra
language of your choice.

\hypertarget{halfling}{%
\subsection{Halfling}\label{halfling}}

\hypertarget{halfling-traits}{%
\subsubsection{Halfling Traits}\label{halfling-traits}}

Your halfling character has a number of traits in common with all other
halflings.

\textbf{Ability Score Increase.} Your Dexterity score increases by 2.

\textbf{Age.} A halfling reaches adulthood at the age of 20 and
generally lives into the middle of his or her second century.

\textbf{Alignment.} Most halflings are lawful good. As a rule, they are
good-hearted and kind, hate to see others in pain, and have no tolerance
for oppression. They are also very orderly and traditional, leaning
heavily on the support of their community and the comfort of their old
ways.

\textbf{Size.} Halflings average about 3 feet tall and weigh about 40
pounds. Your size is Small.

\textbf{Speed.} Your base walking speed is 25 feet.

\textbf{Lucky.} When you roll a 1 on the d20 for an attack roll, ability
check, or saving throw, you can reroll the die and must use the new
roll.

\textbf{Brave.} You have advantage on saving throws against being
frightened.

\textbf{Halfling Nimbleness.} You can move through the space of any
creature that is of a size larger than yours.

\textbf{Languages.} You can speak, read, and write Common and Halfling.
The Halfling language isn't secret, but halflings are loath to share it
with others. They write very little, so they don't have a rich body of
literature. Their oral tradition, however, is very strong. Almost all
halflings speak Common to converse with the people in whose lands they
dwell or through which they are traveling.

\hypertarget{lightfoot}{%
\paragraph{Lightfoot}\label{lightfoot}}

As a lightfoot halfling, you can easily hide from notice, even using
other people as cover. You're inclined to be affable and get along well
with others.

Lightfoots are more prone to wanderlust than other halflings, and often
dwell alongside other races or take up a nomadic life.

\textbf{Ability Score Increase.} Your Charisma score increases by 1.

\textbf{Naturally Stealthy.} You can attempt to hide even when you are
obscured only by a creature that is at least one size larger than you.

\hypertarget{human}{%
\subsection{Human}\label{human}}

\hypertarget{human-traits}{%
\subsubsection{Human Traits}\label{human-traits}}

It's hard to make generalizations about humans, but your human character
has these traits.

\textbf{Ability Score Increase.} Your ability scores each increase by 1.

\textbf{Age.} Humans reach adulthood in their late teens and live less
than a century.

\textbf{Alignment.} Humans tend toward no particular Alignment. The best
and the worst are found among them.

\textbf{Size.} Humans vary widely in height and build, from barely 5
feet to well over 6 feet tall. Regardless of your position in that
range, your size is Medium.

\textbf{Speed.} Your base walking speed is 30 feet.

\textbf{Languages.} You can speak, read, and write Common and one extra
language of your choice.

Humans typically learn the languages of other peoples they deal with,
including obscure dialects. They are fond of sprinkling their speech
with words borrowed from other tongues: Orc curses, Elvish musical
expressions, Dwarvish military phrases, and so on.

\hypertarget{dragonborn}{%
\subsection{Dragonborn}\label{dragonborn}}

\hypertarget{dragonborn-traits}{%
\subsubsection{Dragonborn Traits}\label{dragonborn-traits}}

Your draconic heritage manifests in a variety of traits you share with
other dragonborn.

\textbf{Ability Score Increase.} Your Strength score increases by 2, and
your Charisma score increases by 1.

\textbf{Age.} Young dragonborn grow quickly. They walk hours after
hatching, attain the size and development of a 10-year-old human child
by the age of 3, and reach adulthood by 15. They live to be around 80.

\textbf{Alignment.} Dragonborn tend to extremes, making a conscious
choice for one side or the other in the cosmic war between good and
evil. Most dragonborn are good, but those who side with evil can be
terrible villains.

\textbf{Size.} Dragonborn are taller and heavier than humans, standing
well over 6 feet tall and averaging almost 250 pounds. Your size is
Medium.

\textbf{Speed.} Your base walking speed is 30 feet.

\hypertarget{draconic-ancestry}{%
\paragraph{Draconic Ancestry}\label{draconic-ancestry}}

\begin{longtable}[]{@{}lll@{}}
\toprule
Dragon & Damage Type & Breath Weapon\tabularnewline
\midrule
\endhead
Black & Acid & 5 by 30 ft. line (Dex. save)\tabularnewline
Blue & Lightning & 5 by 30 ft. line (Dex. save)\tabularnewline
Brass & Fire & 5 by 30 ft. line (Dex. save)\tabularnewline
Bronze & Lightning & 5 by 30 ft. line (Dex. save)\tabularnewline
Copper & Acid & 5 by 30 ft. line (Dex. save)\tabularnewline
Gold & Fire & 15 ft. cone (Dex. save)\tabularnewline
Green & Poison & 15 ft. cone (Con. save)\tabularnewline
Red & Fire & 15 ft. cone (Dex. save)\tabularnewline
Silver & Cold & 15 ft. cone (Con. save)\tabularnewline
White & Cold & 15 ft. cone (Con. save)\tabularnewline
\bottomrule
\end{longtable}

\textbf{Draconic Ancestry.} You have draconic ancestry. Choose one type
of dragon from the Draconic Ancestry table. Your breath weapon and
damage resistance are determined by the dragon type, as shown in the
table.

\textbf{Breath Weapon.} You can use your action to exhale destructive
energy. Your draconic ancestry determines the size, shape, and damage
type of the exhalation.

When you use your breath weapon, each creature in the area of the
exhalation must make a saving throw, the type of which is determined by
your draconic ancestry. The DC for this saving throw equals 8 + your
Constitution modifier + your proficiency bonus. A creature takes 2d6
damage on a failed save, and half as much damage on a successful one.
The damage increases to 3d6 at 6th level, 4d6 at 11th level, and 5d6 at
16th level.

After you use your breath weapon, you can't use it again until you
complete a short or long rest.

\textbf{Damage Resistance.} You have resistance to the damage type
associated with your draconic ancestry.

\textbf{Languages.} You can speak, read, and write Common and Draconic.
Draconic is thought to be one of the oldest languages and is often used
in the study of magic. The language sounds harsh to most other creatures
and includes numerous hard consonants and sibilants.

\hypertarget{gnome}{%
\subsection{Gnome}\label{gnome}}

\hypertarget{gnome-traits}{%
\subsubsection{Gnome Traits}\label{gnome-traits}}

Your gnome character has certain characteristics in common with all
other gnomes.

\textbf{Ability Score Increase.} Your Intelligence score increases by 2.

\textbf{Age.} Gnomes mature at the same rate humans do, and most are
expected to settle down into an adult life by around age 40. They can
live 350 to almost 500 years.

\textbf{Alignment.} Gnomes are most often good. Those who tend toward
law are sages, engineers, researchers, scholars, investigators, or
inventors. Those who tend toward chaos are minstrels, tricksters,
wanderers, or fanciful jewelers. Gnomes are good-hearted, and even the
tricksters among them are more playful than vicious.

\textbf{Size.} Gnomes are between 3 and 4 feet tall and average about 40
pounds. Your size is Small.

\textbf{Speed.} Your base walking speed is 25 feet.

\textbf{Darkvision.} Accustomed to life underground, you have superior
vision in dark and dim conditions. You can see in dim light within 60
feet of you as if it were bright light, and in darkness as if it were
dim light.

You can't discern color in darkness, only shades of gray.

\textbf{Gnome Cunning.} You have advantage on all Intelligence, Wisdom,
and Charisma saving throws against magic.

\textbf{Languages.} You can speak, read, and write Common and Gnomish.
The Gnomish language, which uses the Dwarvish script, is renowned for
its technical treatises and its catalogs of knowledge about the natural
world.

\hypertarget{rock-gnome}{%
\paragraph{Rock Gnome}\label{rock-gnome}}

As a rock gnome, you have a natural inventiveness and hardiness beyond
that of other gnomes.

\textbf{Ability Score Increase.} Your Constitution score increases by 1.

\textbf{Artificer's Lore.} Whenever you make an Intelligence (History)
check related to magic items, alchemical objects, or technological
devices, you can add twice your proficiency bonus, instead of any
proficiency bonus you normally apply.

\textbf{Tinker.} You have proficiency with artisan's tools (tinker's
tools). Using those tools, you can spend 1 hour and 10 gp worth of
materials to construct a Tiny clockwork device (AC 5, 1 hp). The device
ceases to function after 24 hours (unless you spend 1 hour repairing it
to keep the device functioning), or when you use your action to
dismantle it; at that time, you can reclaim the materials used to create
it. You can have up to three such devices active at a time.

When you create a device, choose one of the following options:

\textbf{Clockwork Toy.} This toy is a clockwork animal, monster, or
person, such as a frog, mouse, bird, dragon, or soldier. When placed on
the ground, the toy moves 5 feet across the ground on each of your turns
in a random direction. It makes noises as appropriate to the creature it
represents.

\textbf{Fire Starter.} The device produces a miniature flame, which you
can use to light a candle, torch, or campfire. Using the device requires
your action.

\textbf{Music Box.} When opened, this music box plays a single song at a
moderate volume. The box stops playing when it reaches the song's end or
when it is closed.

\hypertarget{half-elf}{%
\subsection{Half-Elf}\label{half-elf}}

\hypertarget{half-elf-traits}{%
\subsubsection{Half-Elf Traits}\label{half-elf-traits}}

Your half-elf character has some qualities in common with elves and some
that are unique to half-elves.

\textbf{Ability Score Increase.} Your Charisma score increases by 2, and
two other ability scores of your choice increase by 1.

\textbf{Age.} Half-elves mature at the same rate humans do and reach
adulthood around the age of 20. They live much longer than humans,
however, often exceeding 180 years.

\textbf{Alignment.} Half-elves share the chaotic bent of their elven
heritage They value both personal freedom and creative expression,
demonstrating neither love of leaders nor desire for followers. They
chafe at rules, resent others' demands, and sometimes prove unreliable,
or at least unpredictable.

\textbf{Size.} Half-elves are about the same size as humans, ranging
from 5 to 6 feet tall. Your size is Medium.

\textbf{Speed.} Your base walking speed is 30 feet.

\textbf{Darkvision.} Thanks to your elf blood, you have superior vision
in dark and dim conditions. You can see in dim light within 60 feet of
you as if it were bright light, and in darkness as if it were dim light.
You can't discern color in darkness, only shades of gray.

\textbf{Fey Ancestry.} You have advantage on saving throws against being
charmed, and magic can't put you to sleep.

\textbf{Skill Versatility.} You gain proficiency in two skills of your
choice.

\textbf{Languages.} You can speak, read, and write Common, Elvish, and
one extra language of your choice.

\hypertarget{half-orc}{%
\subsection{Half-Orc}\label{half-orc}}

\hypertarget{half-orc-traits}{%
\subsubsection{Half-Orc Traits}\label{half-orc-traits}}

Your half-orc character has certain traits deriving from your orc
ancestry.

\textbf{Ability Score Increase.} Your Strength score increases by 2, and
your Constitution score increases by 1.

\textbf{Age.} Half-orcs mature a little faster than humans, reaching
adulthood around age 14. They age noticeably faster and rarely live
longer than 75 years.

\textbf{Alignment.} Half-orcs inherit a tendency toward chaos from their
orc parents and are not strongly inclined toward good. Half-orcs raised
among orcs and willing to live out their lives among them are usually
evil.

\textbf{Size.} Half-orcs are somewhat larger and bulkier than humans,
and they range from 5 to well over 6 feet tall. Your size is Medium.

\textbf{Speed.} Your base walking speed is 30 feet.

\textbf{Darkvision.} Thanks to your orc blood, you have superior vision
in dark and dim conditions. You can see in dim light within 60 feet of
you as if it were bright light, and in darkness as if it were dim light.
You can't discern color in darkness, only shades of gray.

\textbf{Menacing.} You gain proficiency in the Intimidation skill.

\textbf{Relentless Endurance.} When you are reduced to 0 hit points but
not killed outright, you can drop to 1 hit point instead. You can't use
this feature again until you finish a long rest.

\textbf{Savage Attacks.} When you score a critical hit with a melee
weapon attack, you can roll one of the weapon's damage dice one
additional time and add it to the extra damage of the critical hit.

\textbf{Languages.} You can speak, read, and write Common and Orc. Orc
is a harsh, grating language with hard consonants. It has no script of
its own but is written in the Dwarvish script.

\hypertarget{tiefling}{%
\subsection{Tiefling}\label{tiefling}}

\hypertarget{tiefling-traits}{%
\subsubsection{Tiefling Traits}\label{tiefling-traits}}

Tieflings share certain racial traits as a result of their infernal
descent.

\textbf{Ability Score Increase.} Your Intelligence score increases by 1,
and your Charisma score increases by 2.

\textbf{Age.} Tieflings mature at the same rate as humans but live a few
years longer.

\textbf{Alignment.} Tieflings might not have an innate tendency toward
evil, but many of them end up there. Evil or not, an independent nature
inclines many tieflings toward a chaotic alignment.

\textbf{Size.} Tieflings are about the same size and build as humans.
Your size is Medium.

\textbf{Speed.} Your base walking speed is 30 feet.

\textbf{Darkvision.} Thanks to your infernal heritage, you have superior
vision in dark and dim conditions. You can see in dim light within 60
feet of you as if it were bright light, and in darkness as if it were
dim light.

You can't discern color in darkness, only shades of gray.

\textbf{Hellish Resistance.} You have resistance to fire damage.

\textbf{Infernal Legacy.} You know the \emph{thaumaturgy} cantrip. When
you reach 3rd level, you can cast the \emph{hellish rebuke} spell as a
2nd-level spell once with this trait and regain the ability to do so
when you finish a long rest. When you reach 5th level, you can cast the
\emph{darkness} spell once with this trait and regain the ability to do
so when you finish a long rest. Charisma is your spellcasting ability
for these spells.

\textbf{Languages.} You can speak, read, and write Common and Infernal.

\end{document}
