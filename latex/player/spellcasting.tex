% Options for packages loaded elsewhere
\PassOptionsToPackage{unicode}{hyperref}
\PassOptionsToPackage{hyphens}{url}
%
\documentclass[
]{article}
\usepackage{lmodern}
\usepackage{amssymb,amsmath}
\usepackage{ifxetex,ifluatex}
\ifnum 0\ifxetex 1\fi\ifluatex 1\fi=0 % if pdftex
  \usepackage[T1]{fontenc}
  \usepackage[utf8]{inputenc}
  \usepackage{textcomp} % provide euro and other symbols
\else % if luatex or xetex
  \usepackage{unicode-math}
  \defaultfontfeatures{Scale=MatchLowercase}
  \defaultfontfeatures[\rmfamily]{Ligatures=TeX,Scale=1}
\fi
% Use upquote if available, for straight quotes in verbatim environments
\IfFileExists{upquote.sty}{\usepackage{upquote}}{}
\IfFileExists{microtype.sty}{% use microtype if available
  \usepackage[]{microtype}
  \UseMicrotypeSet[protrusion]{basicmath} % disable protrusion for tt fonts
}{}
\makeatletter
\@ifundefined{KOMAClassName}{% if non-KOMA class
  \IfFileExists{parskip.sty}{%
    \usepackage{parskip}
  }{% else
    \setlength{\parindent}{0pt}
    \setlength{\parskip}{6pt plus 2pt minus 1pt}}
}{% if KOMA class
  \KOMAoptions{parskip=half}}
\makeatother
\usepackage{xcolor}
\IfFileExists{xurl.sty}{\usepackage{xurl}}{} % add URL line breaks if available
\IfFileExists{bookmark.sty}{\usepackage{bookmark}}{\usepackage{hyperref}}
\hypersetup{
  hidelinks,
  pdfcreator={LaTeX via pandoc}}
\urlstyle{same} % disable monospaced font for URLs
\setlength{\emergencystretch}{3em} % prevent overfull lines
\providecommand{\tightlist}{%
  \setlength{\itemsep}{0pt}\setlength{\parskip}{0pt}}
\setcounter{secnumdepth}{-\maxdimen} % remove section numbering

\date{}

\begin{document}

\hypertarget{spellcasting}{%
\section{Spellcasting}\label{spellcasting}}

Magic permeates fantasy gaming worlds and often appears in the form of a
spell. This chapter provides the rules for casting spells. Different
character classes have distinctive ways of learning and preparing their
spells, and monsters use spells in unique ways. Regardless of its
source, a spell follows the rules here.

\#\#What Is a Spell? A spell is a discrete magical effect, a single
shaping of the magical energies that suffuse the multiverse into a
specific, limited expression. In casting a spell, a character carefully
plucks at the invisible strands of raw magic suffusing the world, pins
them in place in a particular pattern, sets them vibrating in a specific
way, and then releases them to unleash the desired effect---in most
cases, all in the span of seconds.

Spells can be versatile tools, weapons, or protective wards. They can
deal damage or undo it, impose or remove conditions (see appendix A),
drain life energy away, and restore life to the dead.

Uncounted thousands of spells have been created over the course of the
multiverse's history, and many of them are long forgotten. Some might
yet lie recorded in crumbling spellbooks hidden in ancient ruins or
trapped in the minds of dead gods. Or they might someday be reinvented
by a character who has amassed enough power and wisdom to do so.

\hypertarget{spell-level}{%
\subsubsection{Spell Level}\label{spell-level}}

Every spell has a level from 0 to 9. A spell's level is a general
indicator of how powerful it is, with the lowly (but still impressive)
magic missile at 1st level and the earth shaking wish at 9th.
Cantrips---simple but powerful spells that characters can cast almost by
rote---are level 0. The higher a spell's level, the higher level a
spellcaster must be to use that spell.

Spell level and character level don't correspond directly. Typically, a
character has to be at least 17th level, not 9th level, to cast a 9th
level spell.

\hypertarget{known-and-prepared-spells}{%
\subsubsection{Known and Prepared
Spells}\label{known-and-prepared-spells}}

Before a spellcaster can use a spell, he or she must have the spell
firmly fixed in mind, or must have access to the spell in a magic item.
Members of a few classes, including bards and sorcerers, have a limited
list of spells they know that are always fixed in mind. The same thing
is true of many magic using monsters. Other spellcasters, such as
clerics and wizards, undergo a process of preparing spells. This process
varies for different classes, as detailed in their descriptions.

In every case, the number of spells a caster can have fixed in mind at
any given time depends on the character's level.

\hypertarget{spell-slots}{%
\subsubsection{Spell Slots}\label{spell-slots}}

Regardless of how many spells a caster knows or prepares, he or she can
cast only a limited number of spells before resting. Manipulating the
fabric of magic and channeling its energy into even a simple spell is
physically and mentally taxing, and higher level spells are even more
so. Thus, each spellcasting class's description (except that of the
warlock) includes a table showing how many spell slots of each spell
level a character can use at each character level. For example, the 3rd
level wizard Umara has four 1st level spell slots and two 2nd level
slots.

When a character casts a spell, he or she expends a slot of that spell's
level or higher, effectively ``filling'' a slot with the spell. You can
think of a spell slot as a groove of a certain size---small for a 1st
level slot, larger for a spell of higher level. A 1st level spell fits
into a slot of any size, but a 9th level spell fits only in a 9th level
slot. So when Umara casts magic missile, a 1st level spell, she spends
one of her four 1st level slots and has three remaining.

Finishing a long rest restores any expended spell slots.

Some characters and monsters have special abilities that let them cast
spells without using spell slots. For example, a monk who follows the
Way of the Four Elements, a warlock who chooses certain eldritch
invocations, and a pit fiend from the Nine Hells can all cast spells in
such a way.

\hypertarget{casting-a-spell-at-a-higher-level}{%
\paragraph{Casting a Spell at a Higher
Level}\label{casting-a-spell-at-a-higher-level}}

When a spellcaster casts a spell using a slot that is of a higher level
than the spell, the spell assumes the higher level for that casting. For
instance, if Umara casts magic missile using one of her 2nd level slots,
that magic missile is 2nd level. Effectively, the spell expands to fill
the slot it is put into.

Some spells, such as magic missile and cure wounds, have more powerful
effects when cast at a higher level, as detailed in a spell's
description.

\begin{quote}
\mbox{}%
\hypertarget{casting-in-armor}{%
\paragraph{Casting in Armor}\label{casting-in-armor}}

Because of the mental focus and precise gestures required for
spellcasting, you must be proficient with the armor you are wearing to
cast a spell. You are otherwise too distracted and physically hampered
by your armor for spellcasting.
\end{quote}

\hypertarget{cantrips}{%
\subsubsection{Cantrips}\label{cantrips}}

A cantrip is a spell that can be cast at will, without using a spell
slot and without being prepared in advance. Repeated practice has fixed
the spell in the caster's mind and infused the caster with the magic
needed to produce the effect over and over. A cantrip's spell level is
0.

\hypertarget{rituals}{%
\subsubsection{Rituals}\label{rituals}}

Certain spells have a special tag: ritual. Such a spell can be cast
following the normal rules for spellcasting, or the spell can be cast as
a ritual. The ritual version of a spell takes 10 minutes longer to cast
than normal. It also doesn't expend a spell slot, which means the ritual
version of a spell can't be cast at a higher level.

To cast a spell as a ritual, a spellcaster must have a feature that
grants the ability to do so. The cleric and the druid, for example, have
such a feature. The caster must also have the spell prepared or on his
or her list of spells known, unless the character's ritual feature
specifies otherwise, as the wizard's does.

\#\#Casting a Spell When a character casts any spell, the same basic
rules are followed, regardless of the character's class or the spell's
effects.

Each spell description begins with a block of information, including the
spell's name, level, school of magic, casting time, range, components,
and duration. The rest of a spell entry describes the spell's effect.

\hypertarget{casting-time}{%
\subsubsection{Casting Time}\label{casting-time}}

Most spells require a single action to cast, but some spells require a
bonus action, a reaction, or much more time to cast.

\hypertarget{bonus-action}{%
\paragraph{Bonus Action}\label{bonus-action}}

A spell cast with a bonus action is especially swift. You must use a
bonus action on your turn to cast the spell, provided that you haven't
already taken a bonus action this turn. You can't cast another spell
during the same turn, except for a cantrip with a casting time of 1
action.

\hypertarget{reactions}{%
\paragraph{Reactions}\label{reactions}}

Some spells can be cast as reactions. These spells take a fraction of a
second to bring about and are cast in response to some event. If a spell
can be cast as a reaction, the spell description tells you exactly when
you can do so.

\hypertarget{longer-casting-times}{%
\paragraph{Longer Casting Times}\label{longer-casting-times}}

Certain spells (including spells cast as rituals) require more time to
cast: minutes or even hours. When you cast a spell with a casting time
longer than a single action or reaction, you must spend your action each
turn casting the spell, and you must maintain your concentration while
you do so (see ``Concentration'' below). If your concentration is
broken, the spell fails, but you don't expend a spell slot. If you want
to try casting the spell again, you must start over.

\#\#\#Range The target of a spell must be within the spell's range. For
a spell like \emph{magic missile}, the target is a creature. For a spell
like \emph{fireball}, the target is the point in space where the ball of
fire erupts.

Most spells have ranges expressed in feet. Some spells can target only a
creature (including you) that you touch. Other spells, such as the
\emph{shield} spell, affect only you. These spells have a range of self.

Spells that create cones or lines of effect that originate from you also
have a range of self, indicating that the origin point of the spell's
effect must be you (see ``Areas of Effect'').

Once a spell is cast, its effects aren't limited by its range, unless
the spell's description says otherwise.

\hypertarget{components}{%
\subsubsection{Components}\label{components}}

A spell's components are the physical requirements you must meet in
order to cast it. Each spell's description indicates whether it requires
verbal (V), somatic (S), or material (M) components. If you can't
provide one or more of a spell's components, you are unable to cast the
spell.

\hypertarget{verbal-v}{%
\paragraph{Verbal (V)}\label{verbal-v}}

Most spells require the chanting of mystic words. The words themselves
aren't the source of the spell's power; rather, the particular
combination of sounds, with specific pitch and resonance, sets the
threads of magic in motion. Thus, a character who is gagged or in an
area of silence, such as one created by the silence spell, can't cast a
spell with a verbal component.

\hypertarget{somatic-s}{%
\paragraph{Somatic (S)}\label{somatic-s}}

Spellcasting gestures might include a forceful gesticulation or an
intricate set of gestures. If a spell requires a somatic component, the
caster must have free use of at least one hand to perform these
gestures.

\hypertarget{material-m}{%
\paragraph{Material (M)}\label{material-m}}

Casting some spells requires particular objects, specified in
parentheses in the component entry. A character can use a
\textbf{component pouch} or a \textbf{spellcasting focus} (found in
``Equipment'') in place of the components specified for a spell. But if
a cost is indicated for a component, a character must have that specific
component before he or she can cast the spell.

If a spell states that a material component is consumed by the spell,
the caster must provide this component for each casting of the spell.

A spellcaster must have a hand free to access a spell's material
components---or to hold a spellcasting focus---but it can be the same
hand that he or she uses to perform somatic components.

\hypertarget{duration}{%
\subsubsection{Duration}\label{duration}}

A spell's duration is the length of time the spell persists. A duration
can be expressed in rounds, minutes, hours, or even years. Some spells
specify that their effects last until the spells are dispelled or
destroyed.

\hypertarget{instantaneous}{%
\paragraph{Instantaneous}\label{instantaneous}}

Many spells are instantaneous. The spell harms, heals, creates, or
alters a creature or an object in a way that can't be dispelled, because
its magic exists only for an instant.

\hypertarget{concentration}{%
\paragraph{Concentration}\label{concentration}}

Some spells require you to maintain concentration in order to keep their
magic active. If you lose concentration, such a spell ends.

If a spell must be maintained with concentration, that fact appears in
its Duration entry, and the spell specifies how long you can concentrate
on it. You can end concentration at any time (no action required).

Normal activity, such as moving and attacking, doesn't interfere with
concentration. The following factors can break concentration:

\begin{itemize}
\tightlist
\item
  \textbf{Casting another spell that requires concentration.} You lose
  concentration on a spell if you cast another spell that requires
  concentration. You can't concentrate on two spells at once.
\item
  \textbf{Taking damage.} Whenever you take damage while you are
  concentrating on a spell, you must make a Constitution saving throw to
  maintain your concentration. The DC equals 10 or half the damage you
  take, whichever number is higher. If you take damage from multiple
  sources, such as an arrow and a dragon's breath, you make a separate
  saving throw for each source of damage.
\item
  \textbf{Being incapacitated or killed.} You lose concentration on a
  spell if you are incapacitated or if you die.
\end{itemize}

The GM might also decide that certain environmental phenomena, such as a
wave crashing over you while you're on a storm tossed ship, require you
to succeed on a DC 10 Constitution saving throw to maintain
concentration on a spell.

\hypertarget{targets}{%
\subsubsection{Targets}\label{targets}}

A typical spell requires you to pick one or more targets to be affected
by the spell's magic. A spell's description tells you whether the spell
targets creatures, objects, or a point of origin for an area of effect
(described below).

Unless a spell has a perceptible effect, a creature might not know it
was targeted by a spell at all. An effect like crackling lightning is
obvious, but a more subtle effect, such as an attempt to read a
creature's thoughts, typically goes unnoticed, unless a spell says
otherwise.

\hypertarget{a-clear-path-to-the-target}{%
\paragraph{A Clear Path to the
Target}\label{a-clear-path-to-the-target}}

To target something, you must have a clear path to it, so it can't be
behind total cover.

If you place an area of effect at a point that you can't see and an
obstruction, such as a wall, is between you and that point, the point of
origin comes into being on the near side of that obstruction.

\hypertarget{targeting-yourself}{%
\paragraph{Targeting Yourself}\label{targeting-yourself}}

If a spell targets a creature of your choice, you can choose yourself,
unless the creature must be hostile or specifically a creature other
than you. If you are in the area of effect of a spell you cast, you can
target yourself.

\hypertarget{areas-of-effect}{%
\subsubsection{Areas of Effect}\label{areas-of-effect}}

Spells such as burning hands and cone of cold cover an area, allowing
them to affect multiple creatures at once.

A spell's description specifies its area of effect, which typically has
one of five different shapes: cone, cube, cylinder, line, or sphere.
Every area of effect has a \textbf{point of origin}, a location from
which the spell's energy erupts. The rules for each shape specify how
you position its point of origin. Typically, a point of origin is a
point in space, but some spells have an area whose origin is a creature
or an object.

A spell's effect expands in straight lines from the point of origin. If
no unblocked straight line extends from the point of origin to a
location within the area of effect, that location isn't included in the
spell's area. To block one of these imaginary lines, an obstruction must
provide total cover.

\hypertarget{cone}{%
\paragraph{Cone}\label{cone}}

A cone extends in a direction you choose from its point of origin. A
cone's width at a given point along its length is equal to that point's
distance from the point of origin. A cone's area of effect specifies its
maximum length.

A cone's point of origin is not included in the cone's area of effect,
unless you decide otherwise.

\hypertarget{cube}{%
\paragraph{Cube}\label{cube}}

You select a cube's point of origin, which lies anywhere on a face of
the cubic effect. The cube's size is expressed as the length of each
side.

A cube's point of origin is not included in the cube's area of effect,
unless you decide otherwise.

\hypertarget{cylinder}{%
\paragraph{Cylinder}\label{cylinder}}

A cylinder's point of origin is the center of a circle of a particular
radius, as given in the spell description. The circle must either be on
the ground or at the height of the spell effect. The energy in a
cylinder expands in straight lines from the point of origin to the
perimeter of the circle, forming the base of the cylinder. The spell's
effect then shoots up from the base or down from the top, to a distance
equal to the height of the cylinder.

A cylinder's point of origin is included in the cylinder's area of
effect.

\hypertarget{line}{%
\paragraph{Line}\label{line}}

A line extends from its point of origin in a straight path up to its
length and covers an area defined by its width.

A line's point of origin is not included in the line's area of effect,
unless you decide otherwise.

\hypertarget{sphere}{%
\paragraph{Sphere}\label{sphere}}

You select a sphere's point of origin, and the sphere extends outward
from that point. The sphere's size is expressed as a radius in feet that
extends from the point.

A sphere's point of origin is included in the sphere's area of effect.

\hypertarget{saving-throws}{%
\subsubsection{Saving Throws}\label{saving-throws}}

Many spells specify that a target can make a saving throw to avoid some
or all of a spell's effects. The spell specifies the ability that the
target uses for the save and what happens on a success or failure.

The DC to resist one of your spells equals 8 + your spellcasting ability
modifier + your proficiency bonus + any special modifiers.

\hypertarget{attack-rolls}{%
\subsubsection{Attack Rolls}\label{attack-rolls}}

Some spells require the caster to make an attack roll to determine
whether the spell effect hits the intended target. Your attack bonus
with a spell attack equals your spellcasting ability modifier + your
proficiency bonus.

Most spells that require attack rolls involve ranged attacks. Remember
that you have disadvantage on a ranged attack roll if you are within 5
feet of a hostile creature that can see you and that isn't
incapacitated.

\begin{quote}
\mbox{}%
\hypertarget{the-schools-of-magic}{%
\paragraph{The Schools of Magic}\label{the-schools-of-magic}}
\end{quote}

Academies of magic group spells into eight categories called schools of
magic. Scholars, particularly wizards, apply these categories to all
spells, believing that all magic functions in essentially the same way,
whether it derives from rigorous study or is bestowed by a deity.

\begin{quote}
The schools of magic help describe spells; they have no rules of their
own, although some rules refer to the schools.

\textbf{Abjuration} spells are protective in nature, though some of them
have aggressive uses. They create magical barriers, negate harmful
effects, harm trespassers, or banish creatures to other planes of
existence.

\textbf{Conjuration} spells involve the transportation of objects and
creatures from one location to another. Some spells summon creatures or
objects to the caster's side, whereas others allow the caster to
teleport to another location. Some conjurations create objects or
effects out of nothing.

\textbf{Divination} spells reveal information, whether in the form of
secrets long forgotten, glimpses of the future, the locations of hidden
things, the truth behind illusions, or visions of distant people or
places.

\textbf{Enchantment} spells affect the minds of others, influencing or
controlling their behavior. Such spells can make enemies see the caster
as a friend, force creatures to take a course of action, or even control
another creature like a puppet.

\textbf{Evocation} spells manipulate magical energy to produce a desired
effect. Some call up blasts of fire or lightning. Others channel
positive energy to heal wounds.

\textbf{Illusion} spells deceive the senses or minds of others. They
cause people to see things that are not there, to miss things that are
there, to hear phantom noises, or to remember things that never
happened. Some illusions create phantom images that any creature can
see, but the most insidious illusions plant an image directly in the
mind of a creature.

\textbf{Necromancy} spells manipulate the energies of life and death.
Such spells can grant an extra reserve of life force, drain the life
energy from another creature, create the undead, or even bring the dead
back to life.

Creating the undead through the use of necromancy spells such as animate
dead is not a good act, and only evil casters use such spells
frequently.

\textbf{Transmutation} spells change the properties of a creature,
object, or environment. They might turn an enemy into a harmless
creature, bolster the strength of an ally, make an object move at the
caster's command, or enhance a creature's innate healing abilities to
rapidly recover from injury.
\end{quote}

\hypertarget{combining-magical-effects}{%
\subsubsection{Combining Magical
Effects}\label{combining-magical-effects}}

The effects of different spells add together while the durations of
those spells overlap. The effects of the same spell cast multiple times
don't combine, however. Instead, the most potent effect---such as the
highest bonus---from those castings applies while their durations
overlap.

For example, if two clerics cast bless on the same target, that
character gains the spell's benefit only once; he or she doesn't get to
roll two bonus dice.

\end{document}
