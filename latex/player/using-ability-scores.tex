% Options for packages loaded elsewhere
\PassOptionsToPackage{unicode}{hyperref}
\PassOptionsToPackage{hyphens}{url}
%
\documentclass[
]{article}
\usepackage{lmodern}
\usepackage{amssymb,amsmath}
\usepackage{ifxetex,ifluatex}
\ifnum 0\ifxetex 1\fi\ifluatex 1\fi=0 % if pdftex
  \usepackage[T1]{fontenc}
  \usepackage[utf8]{inputenc}
  \usepackage{textcomp} % provide euro and other symbols
\else % if luatex or xetex
  \usepackage{unicode-math}
  \defaultfontfeatures{Scale=MatchLowercase}
  \defaultfontfeatures[\rmfamily]{Ligatures=TeX,Scale=1}
\fi
% Use upquote if available, for straight quotes in verbatim environments
\IfFileExists{upquote.sty}{\usepackage{upquote}}{}
\IfFileExists{microtype.sty}{% use microtype if available
  \usepackage[]{microtype}
  \UseMicrotypeSet[protrusion]{basicmath} % disable protrusion for tt fonts
}{}
\makeatletter
\@ifundefined{KOMAClassName}{% if non-KOMA class
  \IfFileExists{parskip.sty}{%
    \usepackage{parskip}
  }{% else
    \setlength{\parindent}{0pt}
    \setlength{\parskip}{6pt plus 2pt minus 1pt}}
}{% if KOMA class
  \KOMAoptions{parskip=half}}
\makeatother
\usepackage{xcolor}
\IfFileExists{xurl.sty}{\usepackage{xurl}}{} % add URL line breaks if available
\IfFileExists{bookmark.sty}{\usepackage{bookmark}}{\usepackage{hyperref}}
\hypersetup{
  hidelinks,
  pdfcreator={LaTeX via pandoc}}
\urlstyle{same} % disable monospaced font for URLs
\usepackage{longtable,booktabs}
% Correct order of tables after \paragraph or \subparagraph
\usepackage{etoolbox}
\makeatletter
\patchcmd\longtable{\par}{\if@noskipsec\mbox{}\fi\par}{}{}
\makeatother
% Allow footnotes in longtable head/foot
\IfFileExists{footnotehyper.sty}{\usepackage{footnotehyper}}{\usepackage{footnote}}
\makesavenoteenv{longtable}
\setlength{\emergencystretch}{3em} % prevent overfull lines
\providecommand{\tightlist}{%
  \setlength{\itemsep}{0pt}\setlength{\parskip}{0pt}}
\setcounter{secnumdepth}{-\maxdimen} % remove section numbering

\date{}

\begin{document}

\hypertarget{using-ability-scores}{%
\section{Using Ability Scores}\label{using-ability-scores}}

Six abilities provide a quick description of every creature's physical
and mental characteristics:

\begin{itemize}
\item
  \textbf{Strength}, measuring physical power
\item
  \textbf{Dexterity}, measuring agility
\item
  \textbf{Constitution}, measuring endurance
\item
  \textbf{Intelligence}, measuring reasoning and memory
\item
  \textbf{Wisdom}, measuring perception and insight
\item
  \textbf{Charisma}, measuring force of personality
\end{itemize}

Is a character muscle-bound and insightful? Brilliant and charming?
Nimble and hardy? Ability scores define these qualities---a creature's
assets as well as weaknesses.

The three main rolls of the game---the ability check, the saving throw,
and the attack roll---rely on the six ability scores. The book's
introduction describes the basic rule behind these rolls: roll a d20,
add an ability modifier derived from one of the six ability scores, and
compare the total to a target number.

\hypertarget{ability-scores-and-modifiers}{%
\subsection{Ability Scores and
Modifiers}\label{ability-scores-and-modifiers}}

Each of a creature's abilities has a score, a number that defines the
magnitude of that ability. An ability score is not just a measure of
innate capabilities, but also encompasses a creature's training and
competence in activities related to that ability.

A score of 10 or 11 is the normal human average, but adventurers and
many monsters are a cut above average in most abilities. A score of 18
is the highest that a person usually reaches. Adventurers can have
scores as high as 20, and monsters and divine beings can have scores as
high as 30.

Each ability also has a modifier, derived from the score and ranging
from −5 (for an ability score of 1) to +10 (for a score of 30). The
Ability Scores and Modifiers table notes the ability modifiers for the
range of possible ability scores, from 1 to 30.

\hypertarget{ability-scores-and-modifiers-1}{%
\paragraph{Ability Scores and
Modifiers}\label{ability-scores-and-modifiers-1}}

\begin{longtable}[]{@{}ll@{}}
\toprule
Score & Modifier\tabularnewline
\midrule
\endhead
1 & −5\tabularnewline
2--3 & −4\tabularnewline
4--5 & −3\tabularnewline
6--7 & −2\tabularnewline
8--9 & −1\tabularnewline
10--11 & +0\tabularnewline
12--13 & +1\tabularnewline
14--15 & +2\tabularnewline
16--17 & +3\tabularnewline
18--19 & +4\tabularnewline
20--21 & +5\tabularnewline
22--23 & +6\tabularnewline
24--25 & +7\tabularnewline
26--27 & +8\tabularnewline
28--29 & +9\tabularnewline
30 & +10\tabularnewline
\bottomrule
\end{longtable}

To determine an ability modifier without consulting the table, subtract
10 from the ability score and then divide the total by 2 (round down).

Because ability modifiers affect almost every attack roll, ability
check, and saving throw, ability modifiers come up in play more often
than their associated scores.

\hypertarget{advantage-and-disadvantage}{%
\subsection{Advantage and
Disadvantage}\label{advantage-and-disadvantage}}

Sometimes a special ability or spell tells you that you have advantage
or disadvantage on an ability check, a saving throw, or an attack roll.
When that happens, you roll a second d20 when you make the roll. Use the
higher of the two rolls if you have advantage, and use the lower roll if
you have disadvantage. For example, if you have disadvantage and roll a
17 and a 5, you use the 5. If you instead have advantage and roll those
numbers, you use the 17.

If multiple situations affect a roll and each one grants advantage or
imposes disadvantage on it, you don't roll more than one additional d20.
If two favorable situations grant advantage, for example, you still roll
only one additional d20.

If circumstances cause a roll to have both advantage and disadvantage,
you are considered to have neither of them, and you roll one d20. This
is true even if multiple circumstances impose disadvantage and only one
grants advantage or vice versa. In such a situation, you have neither
advantage nor disadvantage.

When you have advantage or disadvantage and something in the game, such
as the halfling's Lucky trait, lets you reroll the d20, you can reroll
only one of the dice. You choose which one. For example, if a halfling
has advantage or disadvantage on an ability check and rolls a 1 and a
13, the halfling could use the Lucky trait to reroll the 1.

You usually gain advantage or disadvantage through the use of special
abilities, actions, or spells. Inspiration can also give a character
advantage. The GM can also decide that circumstances influence a roll in
one direction or the other and grant advantage or impose disadvantage as
a result.

\hypertarget{proficiency-bonus}{%
\subsection{Proficiency Bonus}\label{proficiency-bonus}}

Characters have a proficiency bonus determined by level. Monsters also
have this bonus, which is incorporated in their stat blocks. The bonus
is used in the rules on ability checks, saving throws, and attack rolls.

Your proficiency bonus can't be added to a single die roll or other
number more than once. For example, if two different rules say you can
add your proficiency bonus to a Wisdom saving throw, you nevertheless
add the bonus only once when you make the save.

Occasionally, your proficiency bonus might be multiplied or divided
(doubled or halved, for example) before you apply it. For example, the
rogue's Expertise feature doubles the proficiency bonus for certain
ability checks. If a circumstance suggests that your proficiency bonus
applies more than once to the same roll, you still add it only once and
multiply or divide it only once.

By the same token, if a feature or effect allows you to multiply your
proficiency bonus when making an ability check that wouldn't normally
benefit from your proficiency bonus, you still don't add the bonus to
the check. For that check your proficiency bonus is 0, given the fact
that multiplying 0 by any number is still 0. For instance, if you lack
proficiency in the History skill, you gain no benefit from a feature
that lets you double your proficiency bonus when you make Intelligence
(History) checks.

In general, you don't multiply your proficiency bonus for attack rolls
or saving throws. If a feature or effect allows you to do so, these same
rules apply.

\hypertarget{ability-checks}{%
\subsection{Ability Checks}\label{ability-checks}}

An ability check tests a character's or monster's innate talent and
training in an effort to overcome a challenge. The GM calls for an
ability check when a character or monster attempts an action (other than
an attack) that has a chance of failure. When the outcome is uncertain,
the dice determine the results.

For every ability check, the GM decides which of the six abilities is
relevant to the task at hand and the difficulty of the task, represented
by a Difficulty Class. The more difficult a task, the higher its DC. The
Typical Difficulty Classes table shows the most common DCs.

\hypertarget{typical-difficulty-classes}{%
\paragraph{Typical Difficulty
Classes}\label{typical-difficulty-classes}}

\begin{longtable}[]{@{}ll@{}}
\toprule
Task Difficulty & DC\tabularnewline
\midrule
\endhead
Very easy & 5\tabularnewline
Easy & 10\tabularnewline
Medium & 15\tabularnewline
Hard & 20\tabularnewline
Very hard & 25\tabularnewline
Nearly impossible & 30\tabularnewline
\bottomrule
\end{longtable}

To make an ability check, roll a d20 and add the relevant ability
modifier. As with other d20 rolls, apply bonuses and penalties, and
compare the total to the DC. If the total equals or exceeds the DC, the
ability check is a success---the creature overcomes the challenge at
hand. Otherwise, it's a failure, which means the character or monster
makes no progress toward the objective or makes progress combined with a
setback determined by the GM.

\hypertarget{contests}{%
\subsubsection{Contests}\label{contests}}

Sometimes one character's or monster's efforts are directly opposed to
another's. This can occur when both of them are trying to do the same
thing and only one can succeed, such as attempting to snatch up a magic
ring that has fallen on the floor. This situation also applies when one
of them is trying to prevent the other one from accomplishing a
goal---for example, when a monster tries to force open a door that an
adventurer is holding closed. In situations like these, the outcome is
determined by a special form of ability check, called a contest.

Both participants in a contest make ability checks appropriate to their
efforts. They apply all appropriate bonuses and penalties, but instead
of comparing the total to a DC, they compare the totals of their two
checks. The participant with the higher check total wins the contest.
That character or monster either succeeds at the action or prevents the
other one from succeeding.

If the contest results in a tie, the situation remains the same as it
was before the contest. Thus, one contestant might win the contest by
default. If two characters tie in a contest to snatch a ring off the
floor, neither character grabs it. In a contest between a monster trying
to open a door and an adventurer trying to keep the door closed, a tie
means that the door remains shut.

\hypertarget{skills}{%
\subsubsection{Skills}\label{skills}}

Each ability covers a broad range of capabilities, including skills that
a character or a monster can be proficient in. A skill represents a
specific aspect of an ability score, and an individual's proficiency in
a skill demonstrates a focus on that aspect. (A character's starting
skill proficiencies are determined at character creation, and a
monster's skill proficiencies appear in the monster's stat block.)

For example, a Dexterity check might reflect a character's attempt to
pull off an acrobatic stunt, to palm an object, or to stay hidden. Each
of these aspects of Dexterity has an associated skill: Acrobatics,
Sleight of Hand, and Stealth, respectively. So a character who has
proficiency in the Stealth skill is particularly good at Dexterity
checks related to sneaking and hiding.

The skills related to each ability score are shown in the following
list. (No skills are related to Constitution.) See an ability's
description in the later sections of this section for examples of how to
use a skill associated with an ability.

\hypertarget{strength}{%
\paragraph{Strength}\label{strength}}

\begin{itemize}
\tightlist
\item
  Athletics
\end{itemize}

\hypertarget{dexterity}{%
\paragraph{Dexterity}\label{dexterity}}

\begin{itemize}
\tightlist
\item
  Acrobatics
\item
  Sleight of Hand
\item
  Stealth
\end{itemize}

\hypertarget{intelligence}{%
\paragraph{Intelligence}\label{intelligence}}

\begin{itemize}
\tightlist
\item
  Arcana
\item
  History
\item
  Investigation
\item
  Nature
\item
  Religion
\end{itemize}

\hypertarget{wisdom}{%
\paragraph{Wisdom}\label{wisdom}}

\begin{itemize}
\tightlist
\item
  Animal Handling
\item
  Insight
\item
  Medicine
\item
  Perception
\item
  Survival
\end{itemize}

\hypertarget{charisma}{%
\paragraph{Charisma}\label{charisma}}

\begin{itemize}
\tightlist
\item
  Deception
\item
  Intimidation
\item
  Performance
\item
  Persuasion
\end{itemize}

Sometimes, the GM might ask for an ability check using a specific
skill---for example, ``Make a Wisdom (Perception) check.'' At other
times, a player might ask the GM if proficiency in a particular skill
applies to a check. In either case, proficiency in a skill means an
individual can add his or her proficiency bonus to ability checks that
involve that skill. Without proficiency in the skill, the individual
makes a normal ability check.

For example, if a character attempts to climb up a dangerous cliff, the
GM might ask for a Strength (Athletics) check. If the character is
proficient in Athletics, the character's proficiency bonus is added to
the Strength check. If the character lacks that proficiency, he or she
just makes a Strength check.

\hypertarget{variant-skills-with-different-abilities}{%
\paragraph{Variant: Skills with Different
Abilities}\label{variant-skills-with-different-abilities}}

Normally, your proficiency in a skill applies only to a specific kind of
ability check. Proficiency in Athletics, for example, usually applies to
Strength checks. In some situations, though, your proficiency might
reasonably apply to a different kind of check. In such cases, the GM
might ask for a check using an unusual combination of ability and skill,
or you might ask your GM if you can apply a proficiency to a different
check. For example, if you have to swim from an offshore island to the
mainland, your GM might call for a Constitution check to see if you have
the stamina to make it that far. In this case, your GM might allow you
to apply your proficiency in Athletics and ask for a Constitution
(Athletics) check. So if you're proficient in Athletics, you apply your
proficiency bonus to the Constitution check just as you would normally
do for a Strength (Athletics) check. Similarly, when your half-orc
barbarian uses a display of raw strength to intimidate an enemy, your GM
might ask for a Strength (Intimidation) check, even though Intimidation
is normally associated with Charisma.

\hypertarget{passive-checks}{%
\subsubsection{Passive Checks}\label{passive-checks}}

A passive check is a special kind of ability check that doesn't involve
any die rolls. Such a check can represent the average result for a task
done repeatedly, such as searching for secret doors over and over again,
or can be used when the GM wants to secretly determine whether the
characters succeed at something without rolling dice, such as noticing a
hidden monster.

Here's how to determine a character's total for a passive check:

\begin{quote}
10 + all modifiers that normally apply to the check
\end{quote}

If the character has advantage on the check, add 5. For disadvantage,
subtract 5. The game refers to a passive check total as a
\textbf{score}.

For example, if a 1st-level character has a Wisdom of 15 and proficiency
in Perception, he or she has a passive Wisdom (Perception) score of 14.

The rules on hiding in the ``Dexterity'' section below rely on passive
checks, as do the exploration rules.

\hypertarget{working-together}{%
\subsubsection{Working Together}\label{working-together}}

Sometimes two or more characters team up to attempt a task. The
character who's leading the effort---or the one with the highest ability
modifier---can make an ability check with advantage, reflecting the help
provided by the other characters. In combat, this requires the Help
action.

A character can only provide help if the task is one that he or she
could attempt alone. For example, trying to open a lock requires
proficiency with thieves' tools, so a character who lacks that
proficiency can't help another character in that task. Moreover, a
character can help only when two or more individuals working together
would actually be productive. Some tasks, such as threading a needle,
are no easier with help.

\hypertarget{group-checks}{%
\paragraph{Group Checks}\label{group-checks}}

When a number of individuals are trying to accomplish something as a
group, the GM might ask for a group ability check. In such a situation,
the characters who are skilled at a particular task help cover those who
aren't.

To make a group ability check, everyone in the group makes the ability
check. If at least half the group succeeds, the whole group succeeds.
Otherwise, the group fails.

Group checks don't come up very often, and they're most useful when all
the characters succeed or fail as a group. For example, when adventurers
are navigating a swamp, the GM might call for a group Wisdom (Survival)
check to see if the characters can avoid the quicksand, sinkholes, and
other natural hazards of the environment. If at least half the group
succeeds, the successful characters are able to guide their companions
out of danger. Otherwise, the group stumbles into one of these hazards.

\hypertarget{using-each-ability}{%
\subsection{Using Each Ability}\label{using-each-ability}}

Every task that a character or monster might attempt in the game is
covered by one of the six abilities. This section explains in more
detail what those abilities mean and the ways they are used in the game.

\hypertarget{strength-1}{%
\subsubsection{Strength}\label{strength-1}}

Strength measures bodily power, athletic training, and the extent to
which you can exert raw physical force.

\hypertarget{strength-checks}{%
\paragraph{Strength Checks}\label{strength-checks}}

A Strength check can model any attempt to lift, push, pull, or break
something, to force your body through a space, or to otherwise apply
brute force to a situation. The Athletics skill reflects aptitude in
certain kinds of Strength checks.

\textbf{Athletics.} Your Strength (Athletics) check covers difficult
situations you encounter while climbing, jumping, or swimming. Examples
include the following activities:

\begin{itemize}
\tightlist
\item
  You attempt to climb a sheer or slippery cliff, avoid hazards while
  scaling a wall, or cling to a surface while something is trying to
  knock you off.
\item
  You try to jump an unusually long distance or pull off a stunt
  midjump.
\item
  You struggle to swim or stay afloat in treacherous currents,
  storm-tossed waves, or areas of thick seaweed. Or another creature
  tries to push or pull you underwater or otherwise interfere with your
  swimming.
\end{itemize}

\textbf{Other Strength Checks.} The GM might also call for a Strength
check when you try to accomplish tasks like the following:

\begin{itemize}
\tightlist
\item
  Force open a stuck, locked, or barred door
\item
  Break free of bonds
\item
  Push through a tunnel that is too small
\item
  Hang on to a wagon while being dragged behind it
\item
  Tip over a statue
\item
  Keep a boulder from rolling
\end{itemize}

\hypertarget{attack-rolls-and-damage}{%
\paragraph{Attack Rolls and Damage}\label{attack-rolls-and-damage}}

You add your Strength modifier to your attack roll and your damage roll
when attacking with a melee weapon such as a mace, a battleaxe, or a
javelin. You use melee weapons to make melee attacks in hand-to-hand
combat, and some of them can be thrown to make a ranged attack.

\hypertarget{lifting-and-carrying}{%
\paragraph{Lifting and Carrying}\label{lifting-and-carrying}}

Your Strength score determines the amount of weight you can bear. The
following terms define what you can lift or carry.

\textbf{Carrying Capacity.} Your carrying capacity is your Strength
score multiplied by 15. This is the weight (in pounds) that you can
carry, which is high enough that most characters don't usually have to
worry about it.

\textbf{Push, Drag, or Lift.} You can push, drag, or lift a weight in
pounds up to twice your carrying capacity (or 30 times your Strength
score). While pushing or dragging weight in excess of your carrying
capacity, your speed drops to 5 feet.

\textbf{Size and Strength.} Larger creatures can bear more weight,
whereas Tiny creatures can carry less. For each size category above
Medium, double the creature's carrying capacity and the amount it can
push, drag, or lift. For a Tiny creature, halve these weights.

\hypertarget{variant-encumbrance}{%
\paragraph{Variant: Encumbrance}\label{variant-encumbrance}}

The rules for lifting and carrying are intentionally simple. Here is a
variant if you are looking for more detailed rules for determining how a
character is hindered by the weight of equipment. When you use this
variant, ignore the Strength column of the Armor table.

If you carry weight in excess of 5 times your Strength score, you are
\textbf{encumbered}, which means your speed drops by 10 feet.

If you carry weight in excess of 10 times your Strength score, up to
your maximum carrying capacity, you are instead \textbf{heavily
encumbered}, which means your speed drops by 20 feet and you have
disadvantage on ability checks, attack rolls, and saving throws that use
Strength, Dexterity, or Constitution.

\hypertarget{dexterity-1}{%
\subsubsection{Dexterity}\label{dexterity-1}}

Dexterity measures agility, reflexes, and balance.

\hypertarget{dexterity-checks}{%
\paragraph{Dexterity Checks}\label{dexterity-checks}}

A Dexterity check can model any attempt to move nimbly, quickly, or
quietly, or to keep from falling on tricky footing. The Acrobatics,
Sleight of Hand, and Stealth skills reflect aptitude in certain kinds of
Dexterity checks.

\textbf{Acrobatics.} Your Dexterity (Acrobatics) check covers your
attempt to stay on your feet in a tricky situation, such as when you're
trying to run across a sheet of ice, balance on a tightrope, or stay
upright on a rocking ship's deck. The GM might also call for a Dexterity
(Acrobatics) check to see if you can perform acrobatic stunts, including
dives, rolls, somersaults, and flips.

\textbf{Sleight of Hand.} Whenever you attempt an act of legerdemain or
manual trickery, such as planting something on someone else or
concealing an object on your person, make a Dexterity (Sleight of Hand)
check. The GM might also call for a Dexterity (Sleight of Hand) check to
determine whether you can lift a coin purse off another person or slip
something out of another person's pocket.

\textbf{Stealth.} Make a Dexterity (Stealth) check when you attempt to
conceal yourself from enemies, slink past guards, slip away without
being noticed, or sneak up on someone without being seen or heard.

\textbf{Other Dexterity Checks.} The GM might call for a Dexterity check
when you try to accomplish tasks like the following:

\begin{itemize}
\tightlist
\item
  Control a heavily laden cart on a steep descent
\item
  Steer a chariot around a tight turn
\item
  Pick a lock
\item
  Disable a trap
\item
  Securely tie up a prisoner
\item
  Wriggle free of bonds
\item
  Play a stringed instrument
\item
  Craft a small or detailed object
\end{itemize}

\hypertarget{attack-rolls-and-damage-1}{%
\paragraph{Attack Rolls and Damage}\label{attack-rolls-and-damage-1}}

You add your Dexterity modifier to your attack roll and your damage roll
when attacking with a ranged weapon, such as a sling or a longbow. You
can also add your Dexterity modifier to your attack roll and your damage
roll when attacking with a melee weapon that has the finesse property,
such as a dagger or a rapier.

\hypertarget{armor-class}{%
\paragraph{Armor Class}\label{armor-class}}

Depending on the armor you wear, you might add some or all of your
Dexterity modifier to your Armor Class.

\hypertarget{initiative}{%
\paragraph{Initiative}\label{initiative}}

At the beginning of every combat, you roll initiative by making a
Dexterity check. Initiative determines the order of creatures' turns in
combat.

\begin{quote}
\mbox{}%
\hypertarget{hiding}{%
\paragraph{Hiding}\label{hiding}}

The GM decides when circumstances are appropriate for hiding. When you
try to hide, make a Dexterity (Stealth) check. Until you are discovered
or you stop hiding, that check's total is contested by the Wisdom
(Perception) check of any creature that actively searches for signs of
your presence.

You can't hide from a creature that can see you clearly, and you give
away your position if you make noise, such as shouting a warning or
knocking over a vase.

An invisible creature can always try to hide. Signs of its passage might
still be noticed, and it does have to stay quiet.

In combat, most creatures stay alert for signs of danger all around, so
if you come out of hiding and approach a creature, it usually sees you.
However, under certain circumstances, the GM might allow you to stay
hidden as you approach a creature that is distracted, allowing you to
gain advantage on an attack roll before you are seen.

\textbf{Passive Perception.} When you hide, there's a chance someone
will notice you even if they aren't searching. To determine whether such
a creature notices you, the GM compares your Dexterity (Stealth) check
with that creature's passive Wisdom (Perception) score, which equals 10
+ the creature's Wisdom modifier, as well as any other bonuses or
penalties. If the creature has advantage, add 5. For disadvantage,
subtract 5. For example, if a 1st-level character (with a proficiency
bonus of +2) has a Wisdom of 15 (a +2 modifier) and proficiency in
Perception, he or she has a passive Wisdom (Perception) of 14.

\textbf{What Can You See?} One of the main factors in determining
whether you can find a hidden creature or object is how well you can see
in an area, which might be \textbf{lightly} or \textbf{heavily
obscured}, as explained in "The Environment."
\end{quote}

\hypertarget{constitution}{%
\subsubsection{Constitution}\label{constitution}}

Constitution measures health, stamina, and vital force.

\hypertarget{constitution-checks}{%
\paragraph{Constitution Checks}\label{constitution-checks}}

Constitution checks are uncommon, and no skills apply to Constitution
checks, because the endurance this ability represents is largely passive
rather than involving a specific effort on the part of a character or
monster. A Constitution check can model your attempt to push beyond
normal limits, however.

The GM might call for a Constitution check when you try to accomplish
tasks like the following:

\begin{itemize}
\tightlist
\item
  Hold your breath
\item
  March or labor for hours without rest
\item
  Go without sleep
\item
  Survive without food or water
\item
  Quaff an entire stein of ale in one go
\end{itemize}

\hypertarget{hit-points}{%
\paragraph{Hit Points}\label{hit-points}}

Your Constitution modifier contributes to your hit points. Typically,
you add your Constitution modifier to each Hit Die you roll for your hit
points.

If your Constitution modifier changes, your hit point maximum changes as
well, as though you had the new modifier from 1st level. For example, if
you raise your Constitution score when you reach 4th level and your
Constitution modifier increases from +1 to +2, you adjust your hit point
maximum as though the modifier had always been +2. So you add 3 hit
points for your first three levels, and then roll your hit points for
4th level using your new modifier. Or if you're 7th level and some
effect lowers your Constitution score so as to reduce your Constitution
modifier by 1, your hit point maximum is reduced by 7.

\hypertarget{intelligence-1}{%
\subsubsection{Intelligence}\label{intelligence-1}}

Intelligence measures mental acuity, accuracy of recall, and the ability
to reason.

\hypertarget{intelligence-checks}{%
\paragraph{Intelligence Checks}\label{intelligence-checks}}

An Intelligence check comes into play when you need to draw on logic,
education, memory, or deductive reasoning. The Arcana, History,
Investigation, Nature, and Religion skills reflect aptitude in certain
kinds of Intelligence checks.

\textbf{Arcana.} Your Intelligence (Arcana) check measures your ability
to recall lore about spells, magic items, eldritch symbols, magical
traditions, the planes of existence, and the inhabitants of those
planes.

\textbf{History.} Your Intelligence (History) check measures your
ability to recall lore about historical events, legendary people,
ancient kingdoms, past disputes, recent wars, and lost civilizations.

\textbf{Investigation.} When you look around for clues and make
deductions based on those clues, you make an Intelligence
(Investigation) check. You might deduce the location of a hidden object,
discern from the appearance of a wound what kind of weapon dealt it, or
determine the weakest point in a tunnel that could cause it to collapse.
Poring through ancient scrolls in search of a hidden fragment of
knowledge might also call for an Intelligence (Investigation) check.

\textbf{Nature.} Your Intelligence (Nature) check measures your ability
to recall lore about terrain, plants and animals, the weather, and
natural cycles.

\textbf{Religion.} Your Intelligence (Religion) check measures your
ability to recall lore about deities, rites and prayers, religious
hierarchies, holy symbols, and the practices of secret cults.

\textbf{Other Intelligence Checks.} The GM might call for an
Intelligence check when you try to accomplish tasks like the following:

\begin{itemize}
\tightlist
\item
  Communicate with a creature without using words
\item
  Estimate the value of a precious item
\item
  Pull together a disguise to pass as a city guard
\item
  Forge a document
\item
  Recall lore about a craft or trade
\item
  Win a game of skill
\end{itemize}

\hypertarget{spellcasting-ability}{%
\paragraph{Spellcasting Ability}\label{spellcasting-ability}}

Wizards use Intelligence as their spellcasting ability, which helps
determine the saving throw DCs of spells they cast.

\hypertarget{wisdom-1}{%
\subsubsection{Wisdom}\label{wisdom-1}}

Wisdom reflects how attuned you are to the world around you and
represents perceptiveness and intuition.

\hypertarget{wisdom-checks}{%
\paragraph{Wisdom Checks}\label{wisdom-checks}}

A Wisdom check might reflect an effort to read body language, understand
someone's feelings, notice things about the environment, or care for an
injured person. The Animal Handling, Insight, Medicine, Perception, and
Survival skills reflect aptitude in certain kinds of Wisdom checks.

\textbf{Animal Handling.} When there is any question whether you can
calm down a domesticated animal, keep a mount from getting spooked, or
intuit an animal's intentions, the GM might call for a Wisdom (Animal
Handling) check. You also make a Wisdom (Animal Handling) check to
control your mount when you attempt a risky maneuver.

\textbf{Insight.} Your Wisdom (Insight) check decides whether you can
determine the true intentions of a creature, such as when searching out
a lie or predicting someone's next move. Doing so involves gleaning
clues from body language, speech habits, and changes in mannerisms.

\textbf{Medicine.} A Wisdom (Medicine) check lets you try to stabilize a
dying companion or diagnose an illness.

\textbf{Perception.} Your Wisdom (Perception) check lets you spot, hear,
or otherwise detect the presence of something. It measures your general
awareness of your surroundings and the keenness of your senses. For
example, you might try to hear a conversation through a closed door,
eavesdrop under an open window, or hear monsters moving stealthily in
the forest. Or you might try to spot things that are obscured or easy to
miss, whether they are orcs lying in ambush on a road, thugs hiding in
the shadows of an alley, or candlelight under a closed secret door.

\textbf{Survival.} The GM might ask you to make a Wisdom (Survival)
check to follow tracks, hunt wild game, guide your group through frozen
wastelands, identify signs that owlbears live nearby, predict the
weather, or avoid quicksand and other natural hazards.

\textbf{Other Wisdom Checks.} The GM might call for a Wisdom check when
you try to accomplish tasks like the following:

\begin{itemize}
\tightlist
\item
  Get a gut feeling about what course of action to follow
\item
  Discern whether a seemingly dead or living creature is undead
\end{itemize}

\hypertarget{spellcasting-ability-1}{%
\paragraph{Spellcasting Ability}\label{spellcasting-ability-1}}

Clerics, druids, and rangers use Wisdom as their spellcasting ability,
which helps determine the saving throw DCs of spells they cast.

\hypertarget{charisma-1}{%
\subsubsection{Charisma}\label{charisma-1}}

Charisma measures your ability to interact effectively with others. It
includes such factors as confidence and eloquence, and it can represent
a charming or commanding personality.

\hypertarget{charisma-checks}{%
\paragraph{Charisma Checks}\label{charisma-checks}}

A Charisma check might arise when you try to influence or entertain
others, when you try to make an impression or tell a convincing lie, or
when you are navigating a tricky social situation. The Deception,
Intimidation, Performance, and Persuasion skills reflect aptitude in
certain kinds of Charisma checks.

\textbf{Deception.} Your Charisma (Deception) check determines whether
you can convincingly hide the truth, either verbally or through your
actions. This deception can encompass everything from misleading others
through ambiguity to telling outright lies. Typical situations include
trying to fast-talk a guard, con a merchant, earn money through
gambling, pass yourself off in a disguise, dull someone's suspicions
with false assurances, or maintain a straight face while telling a
blatant lie.

\textbf{Intimidation.} When you attempt to influence someone through
overt threats, hostile actions, and physical violence, the GM might ask
you to make a Charisma (Intimidation) check. Examples include trying to
pry information out of a prisoner, convincing street thugs to back down
from a confrontation, or using the edge of a broken bottle to convince a
sneering vizier to reconsider a decision.

\textbf{Performance.} Your Charisma (Performance) check determines how
well you can delight an audience with music, dance, acting,
storytelling, or some other form of entertainment.

\textbf{Persuasion.} When you attempt to influence someone or a group of
people with tact, social graces, or good nature, the GM might ask you to
make a Charisma (Persuasion) check. Typically, you use persuasion when
acting in good faith, to foster friendships, make cordial requests, or
exhibit proper etiquette. Examples of persuading others include
convincing a chamberlain to let your party see the king, negotiating
peace between warring tribes, or inspiring a crowd of townsfolk.

\textbf{Other Charisma Checks.} The GM might call for a Charisma check
when you try to accomplish tasks like the following:

\begin{itemize}
\tightlist
\item
  Find the best person to talk to for news, rumors, and gossip
\item
  Blend into a crowd to get the sense of key topics of conversation
\end{itemize}

\hypertarget{spellcasting-ability-2}{%
\paragraph{Spellcasting Ability}\label{spellcasting-ability-2}}

Bards, paladins, sorcerers, and warlocks use Charisma as their
spellcasting ability, which helps determine the saving throw DCs of
spells they cast.

\hypertarget{saving-throws}{%
\subsection{Saving Throws}\label{saving-throws}}

A saving throw---also called a save---represents an attempt to resist a
spell, a trap, a poison, a disease, or a similar threat. You don't
normally decide to make a saving throw; you are forced to make one
because your character or monster is at risk of harm.

To make a saving throw, roll a d20 and add the appropriate ability
modifier. For example, you use your Dexterity modifier for a Dexterity
saving throw.

A saving throw can be modified by a situational bonus or penalty and can
be affected by advantage and disadvantage, as determined by the GM.

Each class gives proficiency in at least two saving throws. The wizard,
for example, is proficient in Intelligence saves. As with skill
proficiencies, proficiency in a saving throw lets a character add his or
her proficiency bonus to saving throws made using a particular ability
score. Some monsters have saving throw proficiencies as well.

The Difficulty Class for a saving throw is determined by the effect that
causes it. For example, the DC for a saving throw allowed by a spell is
determined by the caster's spellcasting ability and proficiency bonus.

The result of a successful or failed saving throw is also detailed in
the effect that allows the save. Usually, a successful save means that a
creature suffers no harm, or reduced harm, from an effect.

\end{document}
