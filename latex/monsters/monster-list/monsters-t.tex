% Options for packages loaded elsewhere
\PassOptionsToPackage{unicode}{hyperref}
\PassOptionsToPackage{hyphens}{url}
%
\documentclass[
]{article}
\usepackage{lmodern}
\usepackage{amssymb,amsmath}
\usepackage{ifxetex,ifluatex}
\ifnum 0\ifxetex 1\fi\ifluatex 1\fi=0 % if pdftex
  \usepackage[T1]{fontenc}
  \usepackage[utf8]{inputenc}
  \usepackage{textcomp} % provide euro and other symbols
\else % if luatex or xetex
  \usepackage{unicode-math}
  \defaultfontfeatures{Scale=MatchLowercase}
  \defaultfontfeatures[\rmfamily]{Ligatures=TeX,Scale=1}
\fi
% Use upquote if available, for straight quotes in verbatim environments
\IfFileExists{upquote.sty}{\usepackage{upquote}}{}
\IfFileExists{microtype.sty}{% use microtype if available
  \usepackage[]{microtype}
  \UseMicrotypeSet[protrusion]{basicmath} % disable protrusion for tt fonts
}{}
\makeatletter
\@ifundefined{KOMAClassName}{% if non-KOMA class
  \IfFileExists{parskip.sty}{%
    \usepackage{parskip}
  }{% else
    \setlength{\parindent}{0pt}
    \setlength{\parskip}{6pt plus 2pt minus 1pt}}
}{% if KOMA class
  \KOMAoptions{parskip=half}}
\makeatother
\usepackage{xcolor}
\IfFileExists{xurl.sty}{\usepackage{xurl}}{} % add URL line breaks if available
\IfFileExists{bookmark.sty}{\usepackage{bookmark}}{\usepackage{hyperref}}
\hypersetup{
  hidelinks,
  pdfcreator={LaTeX via pandoc}}
\urlstyle{same} % disable monospaced font for URLs
\usepackage{longtable,booktabs}
% Correct order of tables after \paragraph or \subparagraph
\usepackage{etoolbox}
\makeatletter
\patchcmd\longtable{\par}{\if@noskipsec\mbox{}\fi\par}{}{}
\makeatother
% Allow footnotes in longtable head/foot
\IfFileExists{footnotehyper.sty}{\usepackage{footnotehyper}}{\usepackage{footnote}}
\makesavenoteenv{longtable}
\setlength{\emergencystretch}{3em} % prevent overfull lines
\providecommand{\tightlist}{%
  \setlength{\itemsep}{0pt}\setlength{\parskip}{0pt}}
\setcounter{secnumdepth}{-\maxdimen} % remove section numbering

\date{}

\begin{document}

\hypertarget{monsters-t}{%
\section{Monsters (T)}\label{monsters-t}}

\hypertarget{tarrasque}{%
\subsubsection{Tarrasque}\label{tarrasque}}

\emph{Gargantuan monstrosity (titan), unaligned}

\textbf{Armor Class} 25 (natural armor)

\textbf{Hit Points} 676 (33d20 + 330)

\textbf{Speed} 40 ft.

\begin{longtable}[]{@{}llllll@{}}
\toprule
STR & DEX & CON & INT & WIS & CHA\tabularnewline
\midrule
\endhead
30 (+10) & 11 (+0) & 30 (+10) & 3 (−4) & 11 (+0) & 11
(+0)\tabularnewline
\bottomrule
\end{longtable}

\textbf{Saving Throws} Int +5, Wis +9, Cha +9

\textbf{Damage Immunities} fire, poison; bludgeoning, piercing, and
slashing from nonmagical attacks

\textbf{Condition Immunities} charmed, frightened, paralyzed, poisoned

\textbf{Senses} blindsight 120 ft., passive Perception 10

\textbf{Languages} ---

\textbf{Challenge} 30 (155,000 XP)

\textbf{Legendary Resistance (3/Day).} If the tarrasque fails a saving
throw, it can choose to succeed instead.

\textbf{Magic Resistance.} The tarrasque has advantage on saving throws
against spells and other magical effects.

\textbf{Reflective Carapace.} Any time the tarrasque is targeted by a
\emph{magic missile} spell, a line spell, or a spell that requires a
ranged attack roll, roll a d6. On a 1 to 5, the tarrasque is unaffected.
On a 6, the tarrasque is unaffected, and the effect is reflected back at
the caster as though it originated from the tarrasque, turning the
caster into the target.

\textbf{Siege Monster.} The tarrasque deals double damage to objects and
structures.

\hypertarget{actions}{%
\paragraph{Actions}\label{actions}}

\textbf{Multiattack.} The tarrasque can use its Frightful Presence. It
then makes five attacks: one with its bite, two with its claws, one with
its horns, and one with its tail. It can use its Swallow instead of its
\textbf{Bite.}

\textbf{Bite.} \emph{Melee Weapon Attack:} +19 to hit, reach 10 ft., one
target. \emph{Hit:} 36 (4d12 + 10) piercing damage. If the target is a
creature, it is grappled (escape DC 20). Until this grapple ends, the
target is restrained, and the tarrasque can't bite another target.

\textbf{Claw.} \emph{Melee Weapon Attack:} +19 to hit, reach 15 ft., one
target. \emph{Hit:} 28 (4d8 + 10) slashing damage.

\textbf{Horns.} \emph{Melee Weapon Attack:} +19 to hit, reach 10 ft.,
one target. \emph{Hit:} 32 (4d10 + 10) piercing damage.

\textbf{Tail.} \emph{Melee Weapon Attack:} +19 to hit, reach 20 ft., one
target. \emph{Hit:} 24 (4d6 + 10) bludgeoning damage. If the target is a
creature, it must succeed on a DC 20 Strength saving throw or be knocked
prone.

\textbf{Frightful Presence.} Each creature of the tarrasque's choice
within 120 feet of it and aware of it must succeed on a DC 17 Wisdom
saving throw or become frightened for 1 minute. A creature can repeat
the saving throw at the end of each of its turns, with disadvantage if
the tarrasque is within line of sight, ending the effect on itself on a
success. If a creature's saving throw is successful or the effect ends
for it, the creature is immune to the tarrasque's Frightful Presence for
the next 24 hours.

\textbf{Swallow.} The tarrasque makes one bite attack against a Large or
smaller creature it is grappling. If the attack hits, the target takes
the bite's damage, the target is swallowed, and the grapple ends. While
swallowed, the creature is blinded and restrained, it has total cover
against attacks and other effects outside the tarrasque, and it takes 56
(16d6) acid damage at the start of each of the tarrasque's turns.

If the tarrasque takes 60 damage or more on a single turn from a
creature inside it, the tarrasque must succeed on a DC 20 Constitution
saving throw at the end of that turn or regurgitate all swallowed
creatures, which fall prone in a space within 10 feet of the tarrasque.
If the tarrasque dies, a swallowed creature is no longer restrained by
it and can escape from the corpse by using 30 feet of movement, exiting
prone.

\hypertarget{legendary-actions}{%
\paragraph{Legendary Actions}\label{legendary-actions}}

The tarrasque can take 3 legendary actions, choosing from the options
below. Only one legendary action option can be used at a time and only
at the end of another creature's turn. The tarrasque regains spent
legendary actions at the start of its turn.

\textbf{Attack.} The tarrasque makes one claw attack or tail attack

\textbf{Move.} The tarrasque moves up to half its speed.

\textbf{Chomp (Costs 2 Actions).} The tarrasque makes one bite attack or
uses its Swallow.

\hypertarget{treant}{%
\subsubsection{Treant}\label{treant}}

\emph{Huge plant, chaotic good}

\textbf{Armor Class} 16 (natural armor)

\textbf{Hit Points} 138 (12d12 + 60)

\textbf{Speed} 30 ft.

\begin{longtable}[]{@{}llllll@{}}
\toprule
STR & DEX & CON & INT & WIS & CHA\tabularnewline
\midrule
\endhead
23 (+6) & 8 (−1) & 21 (+5) & 12 (+1) & 16 (+3) & 12 (+1)\tabularnewline
\bottomrule
\end{longtable}

\textbf{Damage Resistances} bludgeoning, piercing

\textbf{Damage Vulnerabilities} fire

\textbf{Senses} passive Perception 13

\textbf{Languages} Common, Druidic, Elvish, Sylvan

\textbf{Challenge} 9 (5,000 XP)

\textbf{False Appearance.} While the treant remains motionless, it is
indistinguishable from a normal tree.

\textbf{Siege Monster.} The treant deals double damage to objects and
structures.

\hypertarget{actions-1}{%
\paragraph{Actions}\label{actions-1}}

\textbf{Multiattack.} The treant makes two slam attacks.

\textbf{Slam.} \emph{Melee Weapon Attack:} +10 to hit, reach 5 ft., one
target. \emph{Hit:} 16 (3d6 + 6) bludgeoning damage.

\textbf{Rock.} \emph{Ranged Weapon Attack:} +10 to hit, range 60/180
ft., one target. \emph{Hit:} 28 (4d10 + 6) bludgeoning damage.

\textbf{Animate Trees (1/Day).} The treant magically animates one or two
trees it can see within 60 feet of it. These trees have the same
statistics as a treant, except they have Intelligence and Charisma
scores of 1, they can't speak, and they have only the Slam action
option. An animated tree acts as an ally of the treant. The tree remains
animate for 1 day or until it dies; until the treant dies or is more
than 120 feet from the tree; or until the treant takes a bonus action to
turn it back into an inanimate tree. The tree then takes root if
possible.

\hypertarget{troll}{%
\subsubsection{Troll}\label{troll}}

\emph{Large giant, chaotic evil}

\textbf{Armor Class} 15 (natural armor)

\textbf{Hit Points} 84 (8d10 + 40)

\textbf{Speed} 30 ft.

\begin{longtable}[]{@{}llllll@{}}
\toprule
STR & DEX & CON & INT & WIS & CHA\tabularnewline
\midrule
\endhead
18 (+4) & 13 (+1) & 20 (+5) & 7 (−2) & 9 (-1) & 7 (-2)\tabularnewline
\bottomrule
\end{longtable}

\textbf{Skills} Perception +2

\textbf{Senses} darkvision 60 ft., passive Perception 12

\textbf{Languages} Giant

\textbf{Challenge} 5 (1,800 XP)

\textbf{Keen Smell.} The troll has advantage on Wisdom (Perception)
checks that rely on smell.

\textbf{Regeneration.} The troll regains 10 hit points at the start of
its turn. If the troll takes acid or fire damage, this trait doesn't
function at the start of the troll's next turn. The troll dies only if
it starts its turn with 0 hit points and doesn't regenerate.

\hypertarget{actions-2}{%
\paragraph{Actions}\label{actions-2}}

\textbf{Multiattack.} The troll makes three attacks: one with its bite
and two with its claws.

\textbf{Bite.} \emph{Melee Weapon Attack:} +7 to hit, reach 5 ft., one
target. \emph{Hit:} 7 (1d6 + 4) piercing damage.

\textbf{Claw.} \emph{Melee Weapon Attack:} +7 to hit, reach 5 ft., one
target. \emph{Hit:} 11 (2d6 + 4) slashing damage.

\end{document}
