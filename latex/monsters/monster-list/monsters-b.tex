% Options for packages loaded elsewhere
\PassOptionsToPackage{unicode}{hyperref}
\PassOptionsToPackage{hyphens}{url}
%
\documentclass[
]{article}
\usepackage{lmodern}
\usepackage{amssymb,amsmath}
\usepackage{ifxetex,ifluatex}
\ifnum 0\ifxetex 1\fi\ifluatex 1\fi=0 % if pdftex
  \usepackage[T1]{fontenc}
  \usepackage[utf8]{inputenc}
  \usepackage{textcomp} % provide euro and other symbols
\else % if luatex or xetex
  \usepackage{unicode-math}
  \defaultfontfeatures{Scale=MatchLowercase}
  \defaultfontfeatures[\rmfamily]{Ligatures=TeX,Scale=1}
\fi
% Use upquote if available, for straight quotes in verbatim environments
\IfFileExists{upquote.sty}{\usepackage{upquote}}{}
\IfFileExists{microtype.sty}{% use microtype if available
  \usepackage[]{microtype}
  \UseMicrotypeSet[protrusion]{basicmath} % disable protrusion for tt fonts
}{}
\makeatletter
\@ifundefined{KOMAClassName}{% if non-KOMA class
  \IfFileExists{parskip.sty}{%
    \usepackage{parskip}
  }{% else
    \setlength{\parindent}{0pt}
    \setlength{\parskip}{6pt plus 2pt minus 1pt}}
}{% if KOMA class
  \KOMAoptions{parskip=half}}
\makeatother
\usepackage{xcolor}
\IfFileExists{xurl.sty}{\usepackage{xurl}}{} % add URL line breaks if available
\IfFileExists{bookmark.sty}{\usepackage{bookmark}}{\usepackage{hyperref}}
\hypersetup{
  hidelinks,
  pdfcreator={LaTeX via pandoc}}
\urlstyle{same} % disable monospaced font for URLs
\usepackage{longtable,booktabs}
% Correct order of tables after \paragraph or \subparagraph
\usepackage{etoolbox}
\makeatletter
\patchcmd\longtable{\par}{\if@noskipsec\mbox{}\fi\par}{}{}
\makeatother
% Allow footnotes in longtable head/foot
\IfFileExists{footnotehyper.sty}{\usepackage{footnotehyper}}{\usepackage{footnote}}
\makesavenoteenv{longtable}
\setlength{\emergencystretch}{3em} % prevent overfull lines
\providecommand{\tightlist}{%
  \setlength{\itemsep}{0pt}\setlength{\parskip}{0pt}}
\setcounter{secnumdepth}{-\maxdimen} % remove section numbering

\date{}

\begin{document}

\hypertarget{monsters-b}{%
\section{Monsters (B)}\label{monsters-b}}

\hypertarget{basilisk}{%
\subsubsection{Basilisk}\label{basilisk}}

\emph{Medium monstrosity, unaligned}

\textbf{Armor Class} 15 (natural armor)

\textbf{Hit Points} 52 (8d8 + 16)

\textbf{Speed} 20 ft.

\begin{longtable}[]{@{}llllll@{}}
\toprule
STR & DEX & CON & INT & WIS & CHA\tabularnewline
\midrule
\endhead
16 (+3) & 8 (−1) & 15 (+2) & 2 (−4) & 8 (−1) & 7 (−2)\tabularnewline
\bottomrule
\end{longtable}

\textbf{Senses} darkvision 60 ft., passive Perception 9

\textbf{Languages} ---

\textbf{Challenge} 3 (700 XP)

\textbf{Petrifying Gaze.} If a creature starts its turn within 30 feet
of the basilisk and the two of them can see each other, the basilisk can
force the creature to make a DC 12 Constitution saving throw if the
basilisk isn't incapacitated. On a failed save, the creature magically
begins to turn to stone and is restrained. It must repeat the saving
throw at the end of its next turn. On a success, the effect ends. On a
failure, the creature is petrified until freed by the \emph{greater
restoration} spell or other magic.

A creature that isn't surprised can avert its eyes to avoid the saving
throw at the start of its turn. If it does so, it can't see the basilisk
until the start of its next turn, when it can avert its eyes again. If
it looks at the basilisk in the meantime, it must immediately make the
save.

If the basilisk sees its reflection within 30 feet of it in bright
light, it mistakes itself for a rival and targets itself with its gaze.

\hypertarget{actions}{%
\paragraph{Actions}\label{actions}}

\textbf{Bite.} \emph{Melee Weapon Attack:} +5 to hit, reach 5 ft., one
target. \emph{Hit:} 10 (2d6 + 3) piercing damage plus 7 (2d6) poison
damage.

\hypertarget{behir}{%
\subsubsection{Behir}\label{behir}}

\emph{Huge monstrosity, neutral evil}

\textbf{Armor Class} 17 (natural armor)

\textbf{Hit Points} 168 (16d12 + 64)

\textbf{Speed} 50 ft., climb 40 ft.

\begin{longtable}[]{@{}llllll@{}}
\toprule
STR & DEX & CON & INT & WIS & CHA\tabularnewline
\midrule
\endhead
23 (+6) & 16 (+3) & 18 (+4) & 7 (−2) & 14 (+2) & 12 (+1)\tabularnewline
\bottomrule
\end{longtable}

\textbf{Skills} Perception +6, Stealth +7

\textbf{Damage Immunities} lightning

\textbf{Senses} darkvision 90 ft., passive Perception 16

\textbf{Languages} Draconic

\textbf{Challenge} 11 (7,200 XP)

\hypertarget{actions-1}{%
\paragraph{Actions}\label{actions-1}}

\textbf{Multiattack.} The behir makes two attacks: one with its bite and
one to constrict.

\textbf{Bite.} \emph{Melee Weapon Attack:} +10 to hit, reach 10 ft., one
target. \emph{Hit:} 22 (3d10 + 6) piercing damage.

\textbf{Constrict.} \emph{Melee Weapon Attack:} +10 to hit, reach 5 ft.,
one Large or smaller creature. \emph{Hit:} 17 (2d10 + 6) bludgeoning
damage plus 17 (2d10 + 6) slashing damage. The target is grappled
(escape DC 16) if the behir isn't already constricting a creature, and
the target is restrained until this grapple ends.

\textbf{Lightning Breath (Recharge 5--6).} The behir exhales a line of
lightning that is 20 feet long and 5 feet wide. Each creature in that
line must make a DC 16 Dexterity saving throw, taking 66 (12d10)
lightning damage on a failed save, or half as much damage on a
successful one.

\textbf{Swallow.} The behir makes one bite attack against a Medium or
smaller target it is grappling. If the attack hits, the target is also
swallowed, and the grapple ends. While swallowed, the target is blinded
and restrained, it has total cover against attacks and other effects
outside the behir, and it takes 21 (6d6) acid damage at the start of
each of the behir's turns. A behir can have only one creature swallowed
at a time.

If the behir takes 30 damage or more on a single turn from the swallowed
creature, the behir must succeed on a DC 14 Constitution saving throw at
the end of that turn or regurgitate the creature, which falls prone in a
space within 10 feet of the behir. If the behir dies, a swallowed
creature is no longer restrained by it and can escape from the corpse by
using 15 feet of movement, exiting prone.

\hypertarget{bugbear}{%
\subsubsection{Bugbear}\label{bugbear}}

\emph{Medium humanoid (goblinoid), chaotic evil}

\textbf{Armor Class} 16 (hide armor, shield)

\textbf{Hit Points} 27 (5d8 + 5)

\textbf{Speed} 30 ft.

\begin{longtable}[]{@{}llllll@{}}
\toprule
STR & DEX & CON & INT & WIS & CHA\tabularnewline
\midrule
\endhead
15 (+2) & 14 (+2) & 13 (+1) & 8 (−1) & 11 (+0) & 9 (−1)\tabularnewline
\bottomrule
\end{longtable}

\textbf{Skills} Stealth +6, Survival +2

\textbf{Senses} darkvision 60 ft., passive Perception 10

\textbf{Languages} Common, Goblin

\textbf{Challenge} 1 (200 XP)

\textbf{Brute} A melee weapon deals one extra die of its damage when the
bugbear hits with it (included in the attack).

\textbf{Surprise Attack.} If the bugbear surprises a creature and hits
it with an attack during the first round of combat, the target takes an
extra 7 (2d6) damage from the attack.

\hypertarget{actions-2}{%
\paragraph{Actions}\label{actions-2}}

\textbf{Morningstar.} \emph{Melee Weapon Attack:} +4 to hit, reach 5
ft., one target. \emph{Hit:} 11 (2d8 + 2) piercing damage.

\textbf{Javelin} \emph{Melee or Ranged Weapon Attack:} +4 to hit, reach
5 ft. or range 30/120 ft., one target. \emph{Hit:} 9 (2d6 + 2) piercing
damage in melee or 5 (1d6 + 2) piercing damage at range.

\hypertarget{bulette}{%
\subsubsection{Bulette}\label{bulette}}

\emph{Large monstrosity, unaligned}

\textbf{Armor Class} 17 (natural armor)

\textbf{Hit Points} 94 (9d10 + 45)

\textbf{Speed} 40 ft., burrow 40 ft.

\begin{longtable}[]{@{}llllll@{}}
\toprule
STR & DEX & CON & INT & WIS & CHA\tabularnewline
\midrule
\endhead
19 (+4) & 11 (+0) & 21 (+5) & 2 (−4) & 10 (+0) & 5 (−3)\tabularnewline
\bottomrule
\end{longtable}

\textbf{Skills} Perception +6

\textbf{Senses} darkvision 60 ft., tremorsense 60 ft., passive
Perception 16

\textbf{Languages} ---

\textbf{Challenge} 5 (1,800 XP)

\textbf{Standing Leap.} The bulette's long jump is up to 30 feet and its
high jump is up to 15 feet, with or without a running start.

\hypertarget{actions-3}{%
\paragraph{Actions}\label{actions-3}}

\textbf{Bite.} \emph{Melee Weapon Attack:} +7 to hit, reach 5 ft., one
target. \emph{Hit:} 30 (4d12 + 4) piercing damage.

\textbf{Standing Leap.} If the bulette jumps at least 15 feet as part of
its movement, it can then use this action to land on its feet in a space
that contains one or more other creatures. Each of those creatures must
succeed on a DC 16 Strength or Dexterity saving throw (target's choice)
or be knocked prone and take 14 (3d6 + 4) bludgeoning damage plus 14
(3d6 + 4) slashing damage. On a successful save, the creature takes only
half the damage, isn't knocked prone, and is pushed 5 feet out of the
bulette's space into an unoccupied space of the creature's choice. If no
unoccupied space is within range, the creature instead falls prone in
the bulette's space.

\end{document}
