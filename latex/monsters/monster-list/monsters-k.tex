% Options for packages loaded elsewhere
\PassOptionsToPackage{unicode}{hyperref}
\PassOptionsToPackage{hyphens}{url}
%
\documentclass[
]{article}
\usepackage{lmodern}
\usepackage{amssymb,amsmath}
\usepackage{ifxetex,ifluatex}
\ifnum 0\ifxetex 1\fi\ifluatex 1\fi=0 % if pdftex
  \usepackage[T1]{fontenc}
  \usepackage[utf8]{inputenc}
  \usepackage{textcomp} % provide euro and other symbols
\else % if luatex or xetex
  \usepackage{unicode-math}
  \defaultfontfeatures{Scale=MatchLowercase}
  \defaultfontfeatures[\rmfamily]{Ligatures=TeX,Scale=1}
\fi
% Use upquote if available, for straight quotes in verbatim environments
\IfFileExists{upquote.sty}{\usepackage{upquote}}{}
\IfFileExists{microtype.sty}{% use microtype if available
  \usepackage[]{microtype}
  \UseMicrotypeSet[protrusion]{basicmath} % disable protrusion for tt fonts
}{}
\makeatletter
\@ifundefined{KOMAClassName}{% if non-KOMA class
  \IfFileExists{parskip.sty}{%
    \usepackage{parskip}
  }{% else
    \setlength{\parindent}{0pt}
    \setlength{\parskip}{6pt plus 2pt minus 1pt}}
}{% if KOMA class
  \KOMAoptions{parskip=half}}
\makeatother
\usepackage{xcolor}
\IfFileExists{xurl.sty}{\usepackage{xurl}}{} % add URL line breaks if available
\IfFileExists{bookmark.sty}{\usepackage{bookmark}}{\usepackage{hyperref}}
\hypersetup{
  hidelinks,
  pdfcreator={LaTeX via pandoc}}
\urlstyle{same} % disable monospaced font for URLs
\usepackage{longtable,booktabs}
% Correct order of tables after \paragraph or \subparagraph
\usepackage{etoolbox}
\makeatletter
\patchcmd\longtable{\par}{\if@noskipsec\mbox{}\fi\par}{}{}
\makeatother
% Allow footnotes in longtable head/foot
\IfFileExists{footnotehyper.sty}{\usepackage{footnotehyper}}{\usepackage{footnote}}
\makesavenoteenv{longtable}
\setlength{\emergencystretch}{3em} % prevent overfull lines
\providecommand{\tightlist}{%
  \setlength{\itemsep}{0pt}\setlength{\parskip}{0pt}}
\setcounter{secnumdepth}{-\maxdimen} % remove section numbering

\date{}

\begin{document}

\hypertarget{monsters-k}{%
\section{Monsters (K)}\label{monsters-k}}

\hypertarget{kobold}{%
\subsubsection{Kobold}\label{kobold}}

\emph{Small humanoid (kobold), lawful evil}

\textbf{Armor Class} 12

\textbf{Hit Points} 5 (2d6 − 2)

\textbf{Speed} 30 ft.

\begin{longtable}[]{@{}llllll@{}}
\toprule
STR & DEX & CON & INT & WIS & CHA\tabularnewline
\midrule
\endhead
7 (−2) & 15 (+2) & 9 (−1) & 8 (-1) & 7 (-2) & 8 (−1)\tabularnewline
\bottomrule
\end{longtable}

\textbf{Senses} darkvision 60 ft., passive Perception 8

\textbf{Languages} Common, Draconic

\textbf{Challenge} ⅛ (25 XP)

\textbf{Sunlight Sensitivity.} While in sunlight, the kobold has
disadvantage on attack rolls, as well as on Wisdom (Perception) checks
that rely on sight.

\textbf{Pack Tactics.} The kobold has advantage on an attack roll
against a creature if at least one of the kobold's allies is within 5
feet of the creature and the ally isn't incapacitated.

\hypertarget{actions}{%
\paragraph{Actions}\label{actions}}

\textbf{Dagger.} \emph{Melee Weapon Attack:} +4 to hit, reach 5 ft., one
target. \emph{Hit:} 4 (1d4 + 2) piercing damage.

\textbf{Sling.} \emph{Ranged Weapon Attack:} +4 to hit, range 30/120
ft., one target. \emph{Hit:} 4 (1d4 + 2) bludgeoning damage.

\hypertarget{kraken}{%
\subsubsection{Kraken}\label{kraken}}

\emph{Gargantuan monstrosity (titan), chaotic evil}

\textbf{Armor Class} 18 (natural armor)

\textbf{Hit Points} 472 (27d20 + 189)

\textbf{Speed} 20 ft., swim 60 ft.

\begin{longtable}[]{@{}llllll@{}}
\toprule
STR & DEX & CON & INT & WIS & CHA\tabularnewline
\midrule
\endhead
30 (+10) & 11 (+0) & 25 (+7) & 22 (+6) & 18 (+4) & 20
(+5)\tabularnewline
\bottomrule
\end{longtable}

\textbf{Saving Throws} Str +17, Dex +7, Con +14, Int +13, Wis +11

\textbf{Damage Immunities} lightning; bludgeoning, piercing, and
slashing from nonmagical attacks

\textbf{Condition Immunities} frightened, paralyzed

\textbf{Senses} truesight 120 ft., passive Perception 14

\textbf{Languages} understands Abyssal, Celestial, Infernal, and
Primordial but can't speak, telepathy 120 ft.

\textbf{Challenge} 23 (50,000 XP)

\textbf{Amphibious.} The kraken can breathe air and water.

\textbf{Freedom of Movement.} The kraken ignores difficult terrain, and
magical effects can't reduce its speed or cause it to be restrained. It
can spend 5 feet of movement to escape from nonmagical restraints or
being grappled.

\textbf{Siege Monster.} The kraken deals double damage to objects and
structures.

\hypertarget{actions-1}{%
\paragraph{Actions}\label{actions-1}}

\textbf{Multiattack.} The kraken makes three tentacle attacks, each of
which it can replace with one use of \textbf{Fling.}

\textbf{Bite.} \emph{Melee Weapon Attack:} +17 to hit, reach 5 ft., one
target. \emph{Hit:} 23 (3d8 + 10) piercing damage. If the target is a
Large or smaller creature grappled by the kraken, that creature is
swallowed, and the grapple ends.

While swallowed, the creature is blinded and restrained, it has total
cover against attacks and other effects outside the kraken, and it takes
42 (12d6) acid damage at the start of each of the kraken's turns.

If the kraken takes 50 damage or more on a single turn from a creature
inside it, the kraken must succeed on a DC 25 Constitution saving throw
at the end of that turn or regurgitate all swallowed creatures, which
fall prone in a space within 10 feet of the kraken. If the kraken dies,
a swallowed creature is no longer restrained by it and can escape from
the corpse using 15 feet of movement, exiting prone.

\textbf{Tentacle.} \emph{Melee Weapon Attack:} +17 to hit, reach 30 ft.,
one target. \emph{Hit:} 20 (3d6 + 10) bludgeoning damage, and the target
is grappled (escape DC 18). Until this grapple ends, the target is
restrained. The kraken has ten tentacles, each of which can grapple one
target.

\textbf{Fling.} One Large or smaller object held or creature grappled by
the kraken is thrown up to 60 feet in a random direction and knocked
prone. If a thrown target strikes a solid surface, the target takes 3
(1d6) bludgeoning damage for every 10 feet it was thrown. If the target
is thrown at another creature, that creature must succeed on a DC 18
Dexterity saving throw or take the same damage and be knocked prone.

\textbf{Lightning Storm.} The kraken magically creates three bolts of
lightning, each of which can strike a target the kraken can see within
120 feet of it. A target must make a DC 23 Dexterity saving throw,
taking 22 (4d10) lightning damage on a failed save, or half as much
damage on a successful one.

\hypertarget{legendary-actions}{%
\paragraph{Legendary Actions}\label{legendary-actions}}

The kraken can take 3 legendary actions, choosing from the options
below. Only one legendary action option can be used at a time and only
at the end of another creature's turn. The kraken regains spent
legendary actions at the start of its turn.

\textbf{Tentacle Attack or Fling.} The kraken makes one tentacle attack
or uses its Fling.

\textbf{Lightning Storm (Costs 2 Actions).} The kraken uses Lightning
Storm.

\textbf{Ink Cloud (Costs 3 Actions).} While underwater, the kraken
expels an ink cloud in a 60-­‐foot radius. The cloud spreads around
corners, and that area is heavily obscured to creatures other than the
kraken. Each creature other than the kraken that ends its turn there
must succeed on a DC 23 Constitution saving throw, taking 16 (3d10)
poison damage on a failed save, or half as much damage on a successful
one. A strong current disperses the cloud, which otherwise disappears at
the end of the kraken's next turn.

\end{document}
