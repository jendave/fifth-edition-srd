% Options for packages loaded elsewhere
\PassOptionsToPackage{unicode}{hyperref}
\PassOptionsToPackage{hyphens}{url}
%
\documentclass[
]{article}
\usepackage{lmodern}
\usepackage{amssymb,amsmath}
\usepackage{ifxetex,ifluatex}
\ifnum 0\ifxetex 1\fi\ifluatex 1\fi=0 % if pdftex
  \usepackage[T1]{fontenc}
  \usepackage[utf8]{inputenc}
  \usepackage{textcomp} % provide euro and other symbols
\else % if luatex or xetex
  \usepackage{unicode-math}
  \defaultfontfeatures{Scale=MatchLowercase}
  \defaultfontfeatures[\rmfamily]{Ligatures=TeX,Scale=1}
\fi
% Use upquote if available, for straight quotes in verbatim environments
\IfFileExists{upquote.sty}{\usepackage{upquote}}{}
\IfFileExists{microtype.sty}{% use microtype if available
  \usepackage[]{microtype}
  \UseMicrotypeSet[protrusion]{basicmath} % disable protrusion for tt fonts
}{}
\makeatletter
\@ifundefined{KOMAClassName}{% if non-KOMA class
  \IfFileExists{parskip.sty}{%
    \usepackage{parskip}
  }{% else
    \setlength{\parindent}{0pt}
    \setlength{\parskip}{6pt plus 2pt minus 1pt}}
}{% if KOMA class
  \KOMAoptions{parskip=half}}
\makeatother
\usepackage{xcolor}
\IfFileExists{xurl.sty}{\usepackage{xurl}}{} % add URL line breaks if available
\IfFileExists{bookmark.sty}{\usepackage{bookmark}}{\usepackage{hyperref}}
\hypersetup{
  hidelinks,
  pdfcreator={LaTeX via pandoc}}
\urlstyle{same} % disable monospaced font for URLs
\usepackage{longtable,booktabs}
% Correct order of tables after \paragraph or \subparagraph
\usepackage{etoolbox}
\makeatletter
\patchcmd\longtable{\par}{\if@noskipsec\mbox{}\fi\par}{}{}
\makeatother
% Allow footnotes in longtable head/foot
\IfFileExists{footnotehyper.sty}{\usepackage{footnotehyper}}{\usepackage{footnote}}
\makesavenoteenv{longtable}
\setlength{\emergencystretch}{3em} % prevent overfull lines
\providecommand{\tightlist}{%
  \setlength{\itemsep}{0pt}\setlength{\parskip}{0pt}}
\setcounter{secnumdepth}{-\maxdimen} % remove section numbering

\date{}

\begin{document}

\hypertarget{monsters-c}{%
\section{Monsters (C)}\label{monsters-c}}

\hypertarget{centaur}{%
\subsubsection{Centaur}\label{centaur}}

\emph{Large monstrosity, neutral good}

\textbf{Armor Class} 12

\textbf{Hit Points} 45 (6d10 + 12)

\textbf{Speed} 50 ft.

\begin{longtable}[]{@{}llllll@{}}
\toprule
STR & DEX & CON & INT & WIS & CHA\tabularnewline
\midrule
\endhead
18 (+4) & 14 (+2) & 14 (+2) & 9 (−1) & 13 (+1) & 11 (+0)\tabularnewline
\bottomrule
\end{longtable}

\textbf{Skills} Athletics +6, Perception +3, Survival +3

\textbf{Senses} passive Perception 13

\textbf{Languages} Elvish, Sylvan

\textbf{Challenge} 2 (450 XP)

\textbf{Charge} If the centaur moves at least 30 feet straight toward a
target and then hits it with a pike attack on the same turn, the target
takes an extra 10 (3d6) piercing damage.

\hypertarget{actions}{%
\paragraph{Actions}\label{actions}}

\textbf{Multiattack.} The centaur makes two attacks: one with its pike
and one with its hooves or two with its longbow.

\textbf{Pike.} \emph{Melee Weapon Attack:} +6 to hit, reach 10 ft., one
target. \emph{Hit:} 9 (1d10 + 4) piercing damage.

\textbf{Hooves.} \emph{Melee Weapon Attack:} +6 to hit, reach 5 ft., one
target. \emph{Hit:} 11 (2d6 + 4) bludgeoning damage.

\textbf{Longbow.} \emph{Ranged Weapon Attack:} +4 to hit, range 150/600
ft., one target. \emph{Hit:} 6 (1d8 + 2) piercing damage.

\hypertarget{chimera}{%
\subsubsection{Chimera}\label{chimera}}

\emph{Large monstrosity, chaotic evil}

\textbf{Armor Class} 14 (natural armor)

\textbf{Hit Points} 114 (12d10 + 48)

\textbf{Speed} 30 ft., fly 60 ft.

\begin{longtable}[]{@{}llllll@{}}
\toprule
STR & DEX & CON & INT & WIS & CHA\tabularnewline
\midrule
\endhead
19 (+4) & 11 (+0) & 19 (+4) & 3 (−4) & 14 (+2) & 10 (+0)\tabularnewline
\bottomrule
\end{longtable}

\textbf{Skills} Perception +8

\textbf{Senses} darkvision 60 ft., passive Perception 18

\textbf{Languages} understands Draconic but can't speak

\textbf{Challenge} 6 (2,300 XP)

\hypertarget{actions-1}{%
\paragraph{Actions}\label{actions-1}}

\textbf{Multiattack.} The chimera makes three attacks: one with its
bite, one with its horns, and one with its claws. When its fire breath
is available, it can use the breath in place of its bite or horns.

\textbf{Bite.} \emph{Melee Weapon Attack:} +7 to hit, reach 5 ft., one
target. \emph{Hit:} 11 (2d6 + 4) piercing damage.

\textbf{Horns.} \emph{Melee Weapon Attack:} +7 to hit, reach 5 ft., one
target. \emph{Hit:} 10 (1d12 + 4) bludgeoning damage.

\textbf{Claws.} \emph{Melee Weapon Attack:} +7 to hit, reach 5 ft., one
target. \emph{Hit:} 11 (2d6 + 4) slashing damage.

\textbf{Fire Breath (Recharge 5--6).} The dragon head exhales fire in a
15-­‐foot cone. Each creature in that area must make a DC 15 Dexterity
saving throw, taking 31 (7d8) fire damage on a failed save, or half as
much damage on a successful one.

\hypertarget{chuul}{%
\subsubsection{Chuul}\label{chuul}}

\emph{Large aberration, chaotic evil}

\textbf{Armor Class} 16 (natural armor)

\textbf{Hit Points} 93 (11d10 + 33)

\textbf{Speed} 30 ft., swim 30 ft.

\begin{longtable}[]{@{}llllll@{}}
\toprule
STR & DEX & CON & INT & WIS & CHA\tabularnewline
\midrule
\endhead
19 (+4) & 10 (+0) & 16 (+3) & 5 (−3) & 11 (+0) & 5 (−3)\tabularnewline
\bottomrule
\end{longtable}

\textbf{Skills} Perception +4

\textbf{Damage Immunities} poison

\textbf{Condition Immunities} poisoned

\textbf{Senses} darkvision 60 ft., passive Perception 14

\textbf{Languages} understands Deep Speech but can't speak

\textbf{Challenge} 4 (1,100 XP)

\textbf{Amphibious.} The chuul can breathe air and water.

\textbf{Sense Magic.} The chuul sense magic within 120 feet of it at
will. This trait otherwise works like the \emph{detect magic} spell but
isn't itself magical.

\hypertarget{actions-2}{%
\paragraph{Actions}\label{actions-2}}

\textbf{Multiattack.} The chuul makes two pincer attacks. If the chuul
is grappling a creature, the chuul can also use its tentacles once.

\textbf{Pincer.} \emph{Melee Weapon Attack:} +6 to hit, reach 10 ft.,
one target. \emph{Hit:} 11 (2d6 + 4) bludgeoning damage. The target is
grappled (escape DC 14) if it is a Large or smaller creature and the
chuul doesn't have two other creatures grappled.

\textbf{Tentacles.} One creature grappled by the chuul must succeed on a
DC 13 Constitution saving throw or be poisoned for 1 minute. Until this
poison ends, the target is paralyzed. The target can repeat the saving
throw at the end of each of its turns, ending the effect on itself on a
success.

\hypertarget{cloaker}{%
\subsubsection{Cloaker}\label{cloaker}}

\emph{Large aberration, chaotic neutral}

\textbf{Armor Class} 14 (natural armor)

\textbf{Hit Points} 78 (12d10 + 12)

\textbf{Speed} 10 ft., fly 40 ft.

\begin{longtable}[]{@{}llllll@{}}
\toprule
STR & DEX & CON & INT & WIS & CHA\tabularnewline
\midrule
\endhead
17 (+3) & 15 (+2) & 12 (+1) & 13 (+1) & 12 (+1) & 14 (+2)\tabularnewline
\bottomrule
\end{longtable}

\textbf{Skills} Stealth +5

\textbf{Senses} darkvision 60 ft., passive Perception 11

\textbf{Languages} Deep Speech, Undercommon

\textbf{Challenge} 8 (3,900 XP)

\textbf{Damage Transfer.} While attached to a creature, the cloaker
takes only half the damage dealt to it (rounded down), and that creature
takes the other half.

\textbf{False Appearance.} While the cloaker remains motionless without
its underside exposed, it is indistinguishable from a dark leather
cloak.

\textbf{Light Sensitivity.} While in bright light, the cloaker has
disadvantage on attack rolls and Wisdom (Perception) checks that rely on
sight.

\hypertarget{actions-3}{%
\paragraph{Actions}\label{actions-3}}

\textbf{Multiattack.} The cloaker makes two attacks: one with its bite
and one with its tail.

\textbf{Bite.} \emph{Melee Weapon Attack:} +6 to hit, reach 5 ft., one
creature. \emph{Hit:} 10 (2d6 + 3) piercing damage, and if the target is
Large or smaller, the cloaker attaches to it. If the cloaker has
advantage against the target, the cloaker attaches to the target's head,
and the target is blinded and unable to breathe while the cloaker is
attached. While attached, the cloaker can make this attack only against
the target and has advantage on the attack roll. The cloaker can detach
itself by spending 5 feet of its movement. A creature, including the
target, can take its action to detach the cloaker by succeeding on a DC
16 Strength check.

\textbf{Tail.} \emph{Melee Weapon Attack:} +6 to hit, reach 10 ft., one
creature. \emph{Hit:} 7 (1d8 + 3) slashing damage.

\textbf{Moan.} Each creature within 60 feet of the cloaker that can hear
its moan and that isn't an aberration must succeed on a DC 13 Wisdom
saving throw or become frightened until the end of the cloaker's next
turn. If a creature's saving throw is successful, the creature is immune
to the cloaker's moan for the next 24 hours

\textbf{Phantasms (Recharges after a Short or Long Rest).} The cloaker
magically creates three illusory duplicates of itself if it isn't in
bright light. The duplicates move with it and mimic its actions,
shifting position so as to make it impossible to track which cloaker is
the real one. If the cloaker is ever in an area of bright light, the
duplicates disappear.

Whenever any creature targets the cloaker with an attack or a harmful
spell while a duplicate remains, that creature rolls randomly to
determine whether it targets the cloaker or one of the duplicates. A
creature is unaffected by this magical effect if it can't see or if it
relies on senses other than sight.

A duplicate has the cloaker's AC and uses its saving throws. If an
attack hits a duplicate, or if a duplicate fails a saving throw against
an effect that deals damage, the duplicate disappears.

\hypertarget{cockatrice}{%
\subsubsection{Cockatrice}\label{cockatrice}}

\emph{Small monstrosity, unaligned}

\textbf{Armor Class} 11

\textbf{Hit Points} 27 (6d6 + 6)

\textbf{Speed} 20 ft., fly 40 ft.

\begin{longtable}[]{@{}llllll@{}}
\toprule
STR & DEX & CON & INT & WIS & CHA\tabularnewline
\midrule
\endhead
6 (−2) & 12 (+1) & 12 (+1) & 2 (−4) & 13 (+1) & 5 (−3)\tabularnewline
\bottomrule
\end{longtable}

\textbf{Senses} darkvision 60 ft., passive Perception 11

\textbf{Languages} ---

\textbf{Challenge} ½ (100 XP)

\hypertarget{actions-4}{%
\paragraph{Actions}\label{actions-4}}

\textbf{Bite.} \emph{Melee Weapon Attack:} +3 to hit, reach 5 ft., one
creature. \emph{Hit:} 3 (1d4 + 1) piercing damage, and the target must
succeed on a DC 11 Constitution saving throw against being magically
petrified. On a failed save, the creature begins to turn to stone and is
restrained. It must repeat the saving throw at the end of its next turn.
On a success, the effect ends. On a failure, the creature is petrified
for 24 hours.

\hypertarget{couatl}{%
\subsubsection{Couatl}\label{couatl}}

\emph{Medium celestial, lawful good}

\textbf{Armor Class} 19 (natural armor)

\textbf{Hit Points} 97 (13d8 + 39)

\textbf{Speed} 30 ft., fly 90 ft.

\begin{longtable}[]{@{}llllll@{}}
\toprule
STR & DEX & CON & INT & WIS & CHA\tabularnewline
\midrule
\endhead
16 (+3) & 20 (+5) & 17 (+3) & 18 (+4) & 20 (+5) & 18 (+4)\tabularnewline
\bottomrule
\end{longtable}

\textbf{Saving Throws} Con +5, Wis +7, Cha +6

\textbf{Damage Resistances} radiant

\textbf{Damage Immunities} psychic; bludgeoning, piercing, and slashing
from nonmagical attacks

\textbf{Senses} truesight 120 ft., passive Perception 15

\textbf{Languages} all, telepathy 120 ft.

\textbf{Challenge} 4 (1,100 XP)

\textbf{Innate Spellcasting} The couatl's spellcasting ability is
Charisma (spell save DC 14). It can innately cast the following spells,
requiring only verbal components:

At will: \emph{detect evil and good}, \emph{detect magic}, \emph{detect
thoughts}

3/day each: \emph{bless}, \emph{create food and water}, \emph{cure
wounds}, \emph{lesser restoration}, \emph{protection from posion},
\emph{sanctuary}, \emph{shield}

1/day each: \emph{dream}, \emph{greater restoration}, \emph{scrying}

\textbf{Magic Weapons.} The couatl's weapon attacks are magical.

\textbf{Shielded Mind.} The couatl is immune to scrying and to any
effect that would sense its emotions, read its thoughts, or detect its
location.

\hypertarget{actions-5}{%
\paragraph{Actions}\label{actions-5}}

\textbf{Bite.} \emph{Melee Weapon Attack:} +8 to hit, reach 5 ft., one
creature. \emph{Hit:} 8 (1d6 + 5) piercing damage, and the target must
succeed on a DC 13 Constitution saving throw or be poisoned for 24
hours. Until this poison ends, the target is unconscious. Another
creature can use an action to shake the target awake.

\textbf{Constrict.} \emph{Melee Weapon Attack:} +6 to hit, reach 10 ft.,
one Medium or smaller creature. \emph{Hit:} 10 (2d6 + 3) bludgeoning
damage, and the target is grappled (escape DC 15). Until this grapple
ends, the target is restrained, and the couatl can't constrict another
target.

\textbf{Change Shape.} The couatl magically polymorphs into a humanoid
or beast that has a challenge rating equal to or less than its own, or
back into its true form. It reverts to its true form if it dies. Any
equipment it is wearing or carrying is absorbed or borne by the new form
(the couatl's choice).

In a new form, the couatl retains its game statistics and ability to
speak, but its AC, movement modes, Strength, Dexterity, and other
\#\#\#\# Actions are replaced by those of the new form, and it gains any
statistics and capabilities (except class features, legendary actions,
and lair actions) that the new form has but that it lacks. If the new
form has a bite attack, the couatl can use its bite in that form.

\end{document}
