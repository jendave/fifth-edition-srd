% Options for packages loaded elsewhere
\PassOptionsToPackage{unicode}{hyperref}
\PassOptionsToPackage{hyphens}{url}
%
\documentclass[
]{article}
\usepackage{lmodern}
\usepackage{amssymb,amsmath}
\usepackage{ifxetex,ifluatex}
\ifnum 0\ifxetex 1\fi\ifluatex 1\fi=0 % if pdftex
  \usepackage[T1]{fontenc}
  \usepackage[utf8]{inputenc}
  \usepackage{textcomp} % provide euro and other symbols
\else % if luatex or xetex
  \usepackage{unicode-math}
  \defaultfontfeatures{Scale=MatchLowercase}
  \defaultfontfeatures[\rmfamily]{Ligatures=TeX,Scale=1}
\fi
% Use upquote if available, for straight quotes in verbatim environments
\IfFileExists{upquote.sty}{\usepackage{upquote}}{}
\IfFileExists{microtype.sty}{% use microtype if available
  \usepackage[]{microtype}
  \UseMicrotypeSet[protrusion]{basicmath} % disable protrusion for tt fonts
}{}
\makeatletter
\@ifundefined{KOMAClassName}{% if non-KOMA class
  \IfFileExists{parskip.sty}{%
    \usepackage{parskip}
  }{% else
    \setlength{\parindent}{0pt}
    \setlength{\parskip}{6pt plus 2pt minus 1pt}}
}{% if KOMA class
  \KOMAoptions{parskip=half}}
\makeatother
\usepackage{xcolor}
\IfFileExists{xurl.sty}{\usepackage{xurl}}{} % add URL line breaks if available
\IfFileExists{bookmark.sty}{\usepackage{bookmark}}{\usepackage{hyperref}}
\hypersetup{
  hidelinks,
  pdfcreator={LaTeX via pandoc}}
\urlstyle{same} % disable monospaced font for URLs
\setlength{\emergencystretch}{3em} % prevent overfull lines
\providecommand{\tightlist}{%
  \setlength{\itemsep}{0pt}\setlength{\parskip}{0pt}}
\setcounter{secnumdepth}{-\maxdimen} % remove section numbering

\date{}

\begin{document}

\hypertarget{legendary-creatures}{%
\section{Legendary Creatures}\label{legendary-creatures}}

A legendary creature can do things that ordinary creatures can't. It can
take special actions outside its turn, and it might exert magical
influence for miles around.

If a creature assumes the form of a legendary creature, such as through
a spell, it doesn't gain that form's legendary actions, lair actions, or
regional effects.

\hypertarget{legendary-actions}{%
\subsection{Legendary Actions}\label{legendary-actions}}

A legendary creature can take a certain number of special
actions---called legendary actions---outside its turn. Only one
legendary action option can be used at a time and only at the end of
another creature's turn. A creature regains its spent legendary actions
at the start of its turn. It can forgo using them, and it can't use them
while incapacitated or otherwise unable to take actions. If surprised,
it can't use them until after its first turn in the combat.

\hypertarget{a-legendary-creatures-lair}{%
\subsection{A Legendary Creature's
Lair}\label{a-legendary-creatures-lair}}

A legendary creature might have a section describing its lair and the
special effects it can create while there, either by act of will or
simply by being present. Such a section applies only to a legendary
creature that spends a great deal of time in its lair.

\hypertarget{lair-actions}{%
\subsubsection{Lair Actions}\label{lair-actions}}

If a legendary creature has lair actions, it can use them to harness the
ambient magic in its lair. On initiative count 20 (losing all initiative
ties), it can use one of its lair action options. It can't do so while
incapacitated or otherwise unable to take actions. If surprised, it
can't use one until after its first turn in the combat.

\hypertarget{regional-effects}{%
\subsubsection{Regional Effects}\label{regional-effects}}

The mere presence of a legendary creature can have strange and wondrous
effects on its environment, as noted in this section. Regional effects
end abruptly or dissipate over time when the legendary creature dies.

\end{document}
