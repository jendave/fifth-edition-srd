% Options for packages loaded elsewhere
\PassOptionsToPackage{unicode}{hyperref}
\PassOptionsToPackage{hyphens}{url}
%
\documentclass[
]{article}
\usepackage{lmodern}
\usepackage{amssymb,amsmath}
\usepackage{ifxetex,ifluatex}
\ifnum 0\ifxetex 1\fi\ifluatex 1\fi=0 % if pdftex
  \usepackage[T1]{fontenc}
  \usepackage[utf8]{inputenc}
  \usepackage{textcomp} % provide euro and other symbols
\else % if luatex or xetex
  \usepackage{unicode-math}
  \defaultfontfeatures{Scale=MatchLowercase}
  \defaultfontfeatures[\rmfamily]{Ligatures=TeX,Scale=1}
\fi
% Use upquote if available, for straight quotes in verbatim environments
\IfFileExists{upquote.sty}{\usepackage{upquote}}{}
\IfFileExists{microtype.sty}{% use microtype if available
  \usepackage[]{microtype}
  \UseMicrotypeSet[protrusion]{basicmath} % disable protrusion for tt fonts
}{}
\makeatletter
\@ifundefined{KOMAClassName}{% if non-KOMA class
  \IfFileExists{parskip.sty}{%
    \usepackage{parskip}
  }{% else
    \setlength{\parindent}{0pt}
    \setlength{\parskip}{6pt plus 2pt minus 1pt}}
}{% if KOMA class
  \KOMAoptions{parskip=half}}
\makeatother
\usepackage{xcolor}
\IfFileExists{xurl.sty}{\usepackage{xurl}}{} % add URL line breaks if available
\IfFileExists{bookmark.sty}{\usepackage{bookmark}}{\usepackage{hyperref}}
\hypersetup{
  hidelinks,
  pdfcreator={LaTeX via pandoc}}
\urlstyle{same} % disable monospaced font for URLs
\usepackage{longtable,booktabs}
% Correct order of tables after \paragraph or \subparagraph
\usepackage{etoolbox}
\makeatletter
\patchcmd\longtable{\par}{\if@noskipsec\mbox{}\fi\par}{}{}
\makeatother
% Allow footnotes in longtable head/foot
\IfFileExists{footnotehyper.sty}{\usepackage{footnotehyper}}{\usepackage{footnote}}
\makesavenoteenv{longtable}
\setlength{\emergencystretch}{3em} % prevent overfull lines
\providecommand{\tightlist}{%
  \setlength{\itemsep}{0pt}\setlength{\parskip}{0pt}}
\setcounter{secnumdepth}{-\maxdimen} % remove section numbering

\date{}

\begin{document}

\hypertarget{monsters}{%
\section{Monsters}\label{monsters}}

A monster's statistics, sometimes referred to as its \textbf{stat
block}, provide the essential information that you need to run the
monster.

\hypertarget{size}{%
\subsection{Size}\label{size}}

A monster can be Tiny, Small, Medium, Large, Huge, or Gargantuan. The
Size Categories table shows how much space a creature of a particular
size controls in combat. See the \emph{Player's Handbook} for more
information on creature size and space.

\hypertarget{size-categories}{%
\paragraph{Size Categories}\label{size-categories}}

\begin{longtable}[]{@{}lll@{}}
\toprule
Size & Space & Examples\tabularnewline
\midrule
\endhead
Tiny & 2½ by 2½ ft. & Imp, sprite\tabularnewline
Small & 5 by 5 ft. & Giant rat, goblin\tabularnewline
Medium & 5 by 5 ft. & Orc, werewolf\tabularnewline
Large & 10 by 10 ft. & Hippogriff, ogre\tabularnewline
Huge & 15 by 15 ft. & Fire giant, treant\tabularnewline
Gargantuan & 20 by 20 ft. or larger & Kraken, purple worm\tabularnewline
\bottomrule
\end{longtable}

\begin{quote}
\hypertarget{modifying-creatures}{%
\subsection{Modifying Creatures}\label{modifying-creatures}}
\end{quote}

\begin{quote}
Despite the versatile collection of monsters in this book, you might be
at a loss when it comes to finding the perfect creature for part of an
adventure. Feel free to tweak an existing creature to make it into
something more useful for you, perhaps by borrowing a trait or two from
a different monster or by using a \textbf{variant} or \textbf{template},
such as the ones in this book. Keep in mind that modifying a monster,
including when you apply a template to it, might change its challenge
rating.
\end{quote}

\hypertarget{type}{%
\subsection{Type}\label{type}}

A monster's type speaks to its fundamental nature. Certain spells, magic
items, class features, and other effects in the game interact in special
ways with creatures of a particular type. For example, an \emph{arrow of
dragon slaying} deals extra damage not only to dragons but also other
creatures of the dragon type, such as dragon turtles and wyverns.

The game includes the following monster types, which have no rules of
their own.

\textbf{Aberrations} are utterly alien beings. Many of them have innate
magical abilities drawn from the creature's alien mind rather than the
mystical forces of the world. The quintessential aberrations are
aboleths, beholders, mind flayers, and slaadi.

\textbf{Beasts} are nonhumanoid creatures that are a natural part of the
fantasy ecology. Some of them have magical powers, but most are
unintelligent and lack any society or language. Beasts include all
varieties of ordinary animals, dinosaurs, and giant versions of animals.

\textbf{Celestials} are creatures native to the Upper Planes. Many of
them are the servants of deities, employed as messengers or agents in
the mortal realm and throughout the planes. Celestials are good by
nature, so the exceptional celestial who strays from a good alignment is
a horrifying rarity. Celestials include angels, couatls, and pegasi.

\textbf{Constructs} are made, not born. Some are programmed by their
creators to follow a simple set of instructions, while others are imbued
with sentience and capable of independent thought. Golems are the iconic
constructs. Many creatures native to the outer plane of Mechanus, such
as modrons, are constructs shaped from the raw material of the plane by
the will of more powerful creatures.

\textbf{Dragons} are large reptilian creatures of ancient origin and
tremendous power. True dragons, including the good metallic dragons and
the evil chromatic dragons, are highly intelligent and have innate
magic. Also in this category are creatures distantly related to true
dragons, but less powerful, less intelligent, and less magical, such as
wyverns and pseudodragons.

\textbf{Elementals} are creatures native to the elemental planes. Some
creatures of this type are little more than animate masses of their
respective elements, including the creatures simply called elementals.
Others have biological forms infused with elemental energy. The races of
genies, including djinn and efreet, form the most important
civilizations on the elemental planes. Other elemental creatures include
azers and invisible stalkers.

\textbf{Fey} are magical creatures closely tied to the forces of nature.
They dwell in twilight groves and misty forests. In some worlds, they
are closely tied to the Feywild, also called the Plane of Faerie. Some
are also found in the Outer Planes, particularly the planes of Arborea
and the Beastlands. Fey include dryads, pixies, and satyrs.

\textbf{Fiends} are creatures of wickedness that are native to the Lower
Planes. A few are the servants of deities, but many more labor under the
leadership of archdevils and demon princes. Evil priests and mages
sometimes summon fiends to the material world to do their bidding. If an
evil celestial is a rarity, a good fiend is almost inconceivable. Fiends
include demons, devils, hell hounds, rakshasas, and yugoloths.

\textbf{Giants} tower over humans and their kind. They are humanlike in
shape, though some have multiple heads (ettins) or deformities
(fomorians). The six varieties of true giant are hill giants, stone
giants, frost giants, fire giants, cloud giants, and storm giants.
Besides these, creatures such as ogres and trolls are giants.

\textbf{Humanoids} are the main peoples of a fantasy gaming world, both
civilized and savage, including humans and a tremendous variety of other
species. They have language and culture, few if any innate magical
abilities (though most humanoids can learn spellcasting), and a bipedal
form. The most common humanoid races are the ones most suitable as
player characters: humans, dwarves, elves, and halflings. Almost as
numerous but far more savage and brutal, and almost uniformly evil, are
the races of goblinoids (goblins, hobgoblins, and bugbears), orcs,
gnolls, lizardfolk, and kobolds.

\textbf{Monstrosities} are monsters in the strictest sense---frightening
creatures that are not ordinary, not truly natural, and almost never
benign. Some are the results of magical experimentation gone awry (such
as owlbears), and others are the product of terrible curses (including
minotaurs and yuan-ti). They defy categorization, and in some sense
serve as a catch-all category for creatures that don't fit into any
other type.

\textbf{Oozes} are gelatinous creatures that rarely have a fixed shape.
They are mostly subterranean, dwelling in caves and dungeons and feeding
on refuse, carrion, or creatures unlucky enough to get in their way.
Black puddings and gelatinous cubes are among the most recognizable
oozes.

\textbf{Plants} in this context are vegetable creatures, not ordinary
flora. Most of them are ambulatory, and some are carnivorous. The
quintessential plants are the shambling mound and the treant. Fungal
creatures such as the gas spore and the myconid also fall into this
category.

\textbf{Undead} are once-living creatures brought to a horrifying state
of undeath through the practice of necromantic magic or some unholy
curse. Undead include walking corpses, such as vampires and zombies, as
well as bodiless spirits, such as ghosts and specters.

\hypertarget{tags}{%
\subsubsection{Tags}\label{tags}}

A monster might have one or more tags appended to its type, in
parentheses. For example, an orc has the \emph{humanoid (orc)} type. The
parenthetical tags provide additional categorization for certain
creatures. The tags have no rules of their own, but something in the
game, such as a magic item, might refer to them. For instance, a spear
that is especially effective at fighting demons would work against any
monster that has the demon tag.

\hypertarget{alignment}{%
\subsection{Alignment}\label{alignment}}

A monster's alignment provides a clue to its disposition and how it
behaves in a roleplaying or combat situation. For example, a chaotic
evil monster might be difficult to reason with and might attack
characters on sight, whereas a neutral monster might be willing to
negotiate. See the \emph{Player's Handbook} for descriptions of the
different alignments.

The alignment specified in a monster's stat block is the default. Feel
free to depart from it and change a monster's alignment to suit the
needs of your campaign. If you want a good-aligned green dragon or an
evil storm giant, there's nothing stopping you.

Some creatures can have \textbf{any alignment} . In other words, you
choose the monster's alignment. Some monster's alignment entry indicates
a tendency or aversion toward law, chaos, good, or evil. For example, a
berserker can be any chaotic alignment (chaotic good, chaotic neutral,
or chaotic evil), as befits its wild nature.

Many creatures of low intelligence have no comprehension of law or
chaos, good or evil. They don't make moral or ethical choices, but
rather act on instinct. These creatures are \textbf{unaligned}, which
means they don't have an alignment.

\hypertarget{armor-class}{%
\subsection{Armor Class}\label{armor-class}}

A monster that wears armor or carries a shield has an Armor Class (AC)
that takes its armor, shield, and Dexterity into account. Otherwise, a
monster's AC is based on its Dexterity modifier and natural armor, if
any. If a monster has natural armor, wears armor, or carries a shield,
this is noted in parentheses after its AC value.

\hypertarget{hit-points}{%
\subsection{Hit Points}\label{hit-points}}

A monster usually dies or is destroyed when it drops to 0 hit points.
For more on hit points, see the \emph{Player's Handbook.}

A monster's hit points are presented both as a die expression and as an
average number. For example, a monster with 2d8 hit points has 9 hit
points on average (2 × 4½).

A monster's size determines the die used to calculate its hit points, as
shown in the Hit Dice by Size table.

\hypertarget{hit-dice-by-size}{%
\paragraph{Hit Dice by Size}\label{hit-dice-by-size}}

\begin{longtable}[]{@{}lll@{}}
\toprule
Monster Size & Hit Die & Average HP per Die\tabularnewline
\midrule
\endhead
Tiny & d4 & 2½\tabularnewline
Small & d6 & 3½\tabularnewline
Medium & d8 & 4½\tabularnewline
Large & d10 & 5½\tabularnewline
Huge & d12 & 6½\tabularnewline
Gargantuan & d20 & 10½\tabularnewline
\bottomrule
\end{longtable}

A monster's Constitution modifier also affects the number of hit points
it has. Its Constitution modifier is multiplied by the number of Hit
Dice it possesses, and the result is added to its hit points. For
example, if a monster has a Constitution of 12 (+1 modifier) and 2d8 Hit
Dice, it has 2d8 + 2 hit points (average 11).

\hypertarget{speed}{%
\subsection{Speed}\label{speed}}

A monster's speed tells you how far it can move on its turn. For more
information on speed, see the \emph{Player's Handbook.}

All creatures have a walking speed, simply called the monster's speed.
Creatures that have no form of ground-based locomotion have a walking
speed of 0 feet.

Some creatures have one or more of the following additional movement
modes.

\hypertarget{burrow}{%
\subsubsection{Burrow}\label{burrow}}

A monster that has a burrowing speed can use that speed to move through
sand, earth, mud, or ice. A monster can't burrow through solid rock
unless it has a special trait that allows it to do so.

\hypertarget{climb}{%
\subsubsection{Climb}\label{climb}}

A monster that has a climbing speed can use all or part of its movement
to move on vertical surfaces. The monster doesn't need to spend extra
movement to climb.

\hypertarget{fly}{%
\subsubsection{Fly}\label{fly}}

A monster that has a flying speed can use all or part of its movement to
fly. Some monsters have the ability to \textbf{hover}, which makes them
hard to knock out of the air (as explained in the rules on flying in the
\emph{Player's Handbook}). Such a monster stops hovering when it dies.

\hypertarget{swim}{%
\subsubsection{Swim}\label{swim}}

A monster that has a swimming speed doesn't need to spend extra movement
to swim.

\hypertarget{ability-scores}{%
\subsection{Ability Scores}\label{ability-scores}}

Every monster has six ability scores (Strength, Dexterity, Constitution,
Intelligence, Wisdom, and Charisma) and corresponding modifiers. For
more information on ability scores and how they're used in play, see the
\emph{Player's Handbook.}

\hypertarget{saving-throws}{%
\subsection{Saving Throws}\label{saving-throws}}

The Saving Throws entry is reserved for creatures that are adept at
resisting certain kinds of effects. For example, a creature that isn't
easily charmed or frightened might gain a bonus on its Wisdom saving
throws. Most creatures don't have special saving throw bonuses, in which
case this section is absent.

A saving throw bonus is the sum of a monster's relevant ability modifier
and its proficiency bonus, which is determined by the monster's
challenge rating (as shown in the Proficiency Bonus by Challenge Rating
table).

\hypertarget{proficiency-bonus-by-challenge-rating}{%
\paragraph{Proficiency Bonus by Challenge
Rating}\label{proficiency-bonus-by-challenge-rating}}

\begin{longtable}[]{@{}ll@{}}
\toprule
Challenge & Proficiency Bonus\tabularnewline
\midrule
\endhead
0 & +2\tabularnewline
⅛ & +2\tabularnewline
¼ & +2\tabularnewline
½ & +2\tabularnewline
1 & +2\tabularnewline
2 & +2\tabularnewline
3 & +2\tabularnewline
4 & +2\tabularnewline
5 & +3\tabularnewline
6 & +3\tabularnewline
7 & +3\tabularnewline
8 & +3\tabularnewline
9 & +4\tabularnewline
10 & +4\tabularnewline
11 & +4\tabularnewline
12 & +4\tabularnewline
13 & +5\tabularnewline
14 & +5\tabularnewline
15 & +5\tabularnewline
16 & +5\tabularnewline
17 & +6\tabularnewline
18 & +6\tabularnewline
19 & +6\tabularnewline
20 & +6\tabularnewline
21 & +7\tabularnewline
22 & +7\tabularnewline
23 & +7\tabularnewline
24 & +7\tabularnewline
25 & +8\tabularnewline
26 & +8\tabularnewline
27 & +8\tabularnewline
28 & +8\tabularnewline
29 & +9\tabularnewline
30 & +9\tabularnewline
\bottomrule
\end{longtable}

\hypertarget{skills}{%
\subsection{Skills}\label{skills}}

The Skills entry is reserved for monsters that are proficient in one or
more skills. For example, a monster that is very perceptive and stealthy
might have bonuses to Wisdom (Perception) and Dexterity (Stealth)
checks.

A skill bonus is the sum of a monster's relevant ability modifier and
its proficiency bonus, which is determined by the monster's challenge
rating (as shown in the Proficiency Bonus by Challenge Rating table).
Other modifiers might apply. For instance, a monster might have a
larger-than-expected bonus (usually double its proficiency bonus) to
account for its heightened expertise.

\hypertarget{vulnerabilities-resistances-and-immunities}{%
\subsection{Vulnerabilities, Resistances, and
Immunities}\label{vulnerabilities-resistances-and-immunities}}

Some creatures have vulnerability, resistance, or immunity to certain
types of damage. Particular creatures are even resistant or immune to
damage from nonmagical attacks (a magical attack is an attack delivered
by a spell, a magic item, or another magical source). In addition, some
creatures are immune to certain conditions.

\hypertarget{senses}{%
\subsection{Senses}\label{senses}}

The Senses entry notes a monster's passive Wisdom (Perception) score, as
well as any special senses the monster might have. Special senses are
described below.

\hypertarget{blindsight}{%
\subsubsection{Blindsight}\label{blindsight}}

A monster with blindsight can perceive its surroundings without relying
on sight, within a specific radius.

Creatures without eyes, such as grimlocks and gray oozes, typically have
this special sense, as do creatures with echolocation or heightened
senses, such as bats and true dragons.

If a monster is naturally blind, it has a parenthetical note to this
effect, indicating that the radius of its blindsight defines the maximum
range of its perception.

\hypertarget{darkvision}{%
\subsubsection{Darkvision}\label{darkvision}}

A monster with darkvision can see in the dark within a specific radius.
The monster can see in dim light within the radius as if it were bright
light, and in darkness as if it were dim light. The monster can't
discern color in darkness, only shades of gray. Many creatures that live
underground have this special sense.

\begin{quote}
\mbox{}%
\hypertarget{armor-weapon-and-tool-proficiencies}{%
\paragraph{Armor, Weapon, and Tool
Proficiencies}\label{armor-weapon-and-tool-proficiencies}}
\end{quote}

\begin{quote}
Assume that a creature is proficient with its armor, weapons, and tools.
If you swap them out, you decide whether the creature is proficient with
its new equipment.
\end{quote}

\begin{quote}
For example, a hill giant typically wears hide armor and wields a
greatclub. You could equip a hill giant with chain mail and a greataxe
instead, and assume the giant is proficient with both, one or the other,
or neither.
\end{quote}

\begin{quote}
See the \emph{Player's Handbook} for rules on using armor or weapons
without proficiency.
\end{quote}

\hypertarget{tremorsense}{%
\subsubsection{Tremorsense}\label{tremorsense}}

A monster with tremorsense can detect and pinpoint the origin of
vibrations within a specific radius, provided that the monster and the
source of the vibrations are in contact with the same ground or
substance. Tremorsense can't be used to detect flying or incorporeal
creatures. Many burrowing creatures, such as ankhegs and umber hulks,
have this special sense.

\hypertarget{truesight}{%
\subsubsection{Truesight}\label{truesight}}

A monster with truesight can, out to a specific range, see in normal and
magical darkness, see invisible creatures and objects, automatically
detect visual illusions and succeed on saving throws against them, and
perceive the original form of a shapechanger or a creature that is
transformed by magic.

Furthermore, the monster can see into the Ethereal Plane within the same
range.

\hypertarget{languages}{%
\subsection{Languages}\label{languages}}

The languages that a monster can speak are listed in alphabetical order.
Sometimes a monster can understand a language but can't speak it, and
this is noted in its entry. A ``---'' indicates that a creature neither
speaks nor understands any language.

\hypertarget{telepathy}{%
\subsubsection{Telepathy}\label{telepathy}}

Telepathy is a magical ability that allows a monster to communicate
mentally with another creature within a specified range. The contacted
creature doesn't need to share a language with the monster to
communicate in this way with it, but it must be able to understand at
least one language. A creature without telepathy can receive and respond
to telepathic messages but can't initiate or terminate a telepathic
conversation.

A telepathic monster doesn't need to see a contacted creature and can
end the telepathic contact at any time. The contact is broken as soon as
the two creatures are no longer within range of each other or if the
telepathic monster contacts a different creature within range. A
telepathic monster can initiate or terminate a telepathic conversation
without using an action, but while the monster is incapacitated, it
can't initiate telepathic contact, and any current contact is
terminated.

A creature within the area of an \emph{antimagic field} or in any other
location where magic doesn't function can't send or receive telepathic
messages.

\hypertarget{challenge}{%
\subsection{Challenge}\label{challenge}}

A monster's \textbf{challenge rating} tells you how great a threat the
monster is. An appropriately equipped and well-rested party of four
adventurers should be able to defeat a monster that has a challenge
rating equal to its level without suffering any deaths. For example, a
party of four 3rd-level characters should find a monster with a
challenge rating of 3 to be a worthy challenge, but not a deadly one.

Monsters that are significantly weaker than 1st- level characters have a
challenge rating lower than 1. Monsters with a challenge rating of 0 are
insignificant except in large numbers; those with no effective attacks
are worth no experience points, while those that have attacks are worth
10 XP each.

Some monsters present a greater challenge than even a typical 20th-level
party can handle. These monsters have a challenge rating of 21 or higher
and are specifically designed to test player skill.

\hypertarget{experience-points}{%
\subsubsection{Experience Points}\label{experience-points}}

The number of experience points (XP) a monster is worth is based on its
challenge rating. Typically, XP is awarded for defeating the monster,
although the GM may also award XP for neutralizing the threat posed by
the monster in some other manner.

Unless something tells you otherwise, a monster summoned by a spell or
other magical ability is worth the XP noted in its stat block.

\hypertarget{experience-points-by-challenge-rating}{%
\paragraph{Experience Points by Challenge
Rating}\label{experience-points-by-challenge-rating}}

\begin{longtable}[]{@{}ll@{}}
\toprule
Challenge & XP\tabularnewline
\midrule
\endhead
0 & 0 or 10\tabularnewline
⅛ & 25\tabularnewline
¼ & 50\tabularnewline
½ & 100\tabularnewline
1 & 200\tabularnewline
2 & 450\tabularnewline
3 & 700\tabularnewline
4 & 1,100\tabularnewline
5 & 1,800\tabularnewline
6 & 2,300\tabularnewline
7 & 2,900\tabularnewline
8 & 3,900\tabularnewline
9 & 5,000\tabularnewline
10 & 5,900\tabularnewline
11 & 7,200\tabularnewline
12 & 8,400\tabularnewline
13 & 10,000\tabularnewline
14 & 11,500\tabularnewline
15 & 13,000\tabularnewline
16 & 15,000\tabularnewline
17 & 18,000\tabularnewline
18 & 20,000\tabularnewline
19 & 22,000\tabularnewline
20 & 25,000\tabularnewline
21 & 33,000\tabularnewline
22 & 41,000\tabularnewline
23 & 50,000\tabularnewline
24 & 62,000\tabularnewline
25 & 75,000\tabularnewline
26 & 90,000\tabularnewline
27 & 105,000\tabularnewline
28 & 120,000\tabularnewline
29 & 135,000\tabularnewline
30 & 155,000\tabularnewline
\bottomrule
\end{longtable}

\hypertarget{special-traits}{%
\subsection{Special Traits}\label{special-traits}}

Special traits (which appear after a monster's challenge rating but
before any actions or reactions) are characteristics that are likely to
be relevant in a combat encounter and that require some explanation.

\hypertarget{innate-spellcasting}{%
\subsubsection{Innate Spellcasting}\label{innate-spellcasting}}

A monster with the innate ability to cast spells has the Innate
Spellcasting special trait. Unless noted otherwise, an innate spell of
1st level or higher is always cast at its lowest possible level and
can't be cast at a higher level. If a monster has a cantrip where its
level matters and no level is given, use the monster's challenge rating.

An innate spell can have special rules or restrictions. For example, a
drow mage can innately cast the \emph{levitate} spell, but the spell has
a ``self only'' restriction, which means that the spell affects only the
drow mage.

A monster's innate spells can't be swapped out with other spells. If a
monster's innate spells don't require attack rolls, no attack bonus is
given for them.

\hypertarget{spellcasting}{%
\subsubsection{Spellcasting}\label{spellcasting}}

A monster with the Spellcasting special trait has a spellcaster level
and spell slots, which it uses to cast its spells of 1st level and
higher (as explained in the \emph{Player's Handbook}). The spellcaster
level is also used for any cantrips included in the feature.

The monster has a list of spells known or prepared from a specific
class. The list might also include spells from a feature in that class,
such as the Divine Domain feature of the cleric or the Druid Circle
feature of the druid. The monster is considered a member of that class
when attuning to or using a magic item that requires membership in the
class or access to its spell list.

A monster can cast a spell from its list at a higher level if it has the
spell slot to do so. For example, a drow mage with the 3rd-level
\emph{lightning bolt} spell can cast it as a 5th-level spell by using
one of its 5th-level spell slots.

You can change the spells that a monster knows or has prepared,
replacing any spell on its spell list with a spell of the same level and
from the same class list. If you do so, you might cause the monster to
be a greater or lesser threat than suggested by its challenge rating.

\hypertarget{psionics}{%
\subsubsection{Psionics}\label{psionics}}

A monster that casts spells using only the power of its mind has the
psionics tag added to its Spellcasting or Innate Spellcasting special
trait. This tag carries no special rules of its own, but other parts of
the game might refer to it. A monster that has this tag typically
doesn't require any components to cast its spells.

\hypertarget{actions}{%
\subsection{Actions}\label{actions}}

When a monster takes its action, it can choose from the options in the
Actions section of its stat block or use one of the actions available to
all creatures, such as the Dash or Hide action, as described in the
\emph{Player's Handbook}.

\hypertarget{melee-and-ranged-attacks}{%
\subsubsection{Melee and Ranged
Attacks}\label{melee-and-ranged-attacks}}

The most common actions that a monster will take in combat are melee and
ranged attacks. These can be spell attacks or weapon attacks, where the
``weapon'' might be a manufactured item or a natural weapon, such as a
claw or tail spike. For more information on different kinds of attacks,
see the \emph{Player's Handbook}.

\textbf{Creature vs. Target.} The target of a melee or ranged attack is
usually either one creature or one target, the difference being that a
``target'' can be a creature or an object.

\textbf{Hit.} Any damage dealt or other effects that occur as a result
of an attack hitting a target are described after the ``\emph{Hit}''
notation. You have the option of taking average damage or rolling the
damage; for this reason, both the average damage and the die expression
are presented.

\textbf{Miss.} If an attack has an effect that occurs on a miss, that
information is presented after the ``\emph{Miss:}'' notation.

\hypertarget{multiattack}{%
\subsubsection{Multiattack}\label{multiattack}}

A creature that can make multiple attacks on its turn has the
Multiattack action. A creature can't use Multiattack when making an
opportunity attack, which must be a single melee attack.

\hypertarget{ammunition}{%
\subsubsection{Ammunition}\label{ammunition}}

A monster carries enough ammunition to make its ranged attacks. You can
assume that a monster has 2d4 pieces of ammunition for a thrown weapon
attack, and 2d10 pieces of ammunition for a projectile weapon such as a
bow or crossbow.

\hypertarget{reactions}{%
\subsection{Reactions}\label{reactions}}

If a monster can do something special with its reaction, that
information is contained here. If a creature has no special reaction,
this section is absent.

\hypertarget{limited-usage}{%
\subsection{Limited Usage}\label{limited-usage}}

Some special abilities have restrictions on the number of times they can
be used.

\textbf{X/Day.} The notation ``X/Day'' means a special ability can be
used X number of times and that a monster must finish a long rest to
regain expended uses. For example, ``1/Day'' means a special ability can
be used once and that the monster must finish a long rest to use it
again.

\textbf{Recharge X--Y.} The notation ``Recharge X--Y'' means a monster
can use a special ability once and that the ability then has a random
chance of recharging during each subsequent round of combat. At the
start of each of the monster's turns, roll a d6. If the roll is one of
the numbers in the recharge notation, the monster regains the use of the
special ability. The ability also recharges when the monster finishes a
short or long rest.

For example, ``Recharge 5--6'' means a monster can use the special
ability once. Then, at the start of the monster's turn, it regains the
use of that ability if it rolls a 5 or 6 on a d6.

\textbf{Recharge after a Short or Long Rest.} This notation means that a
monster can use a special ability once and then must finish a short or
long rest to use it again.

\begin{quote}
\mbox{}%
\hypertarget{grapple-rules-for-monsters}{%
\paragraph{Grapple Rules for
Monsters}\label{grapple-rules-for-monsters}}
\end{quote}

\begin{quote}
Many monsters have special attacks that allow them to quickly grapple
prey. When a monster hits with such an attack, it doesn't need to make
an additional ability check to determine whether the grapple succeeds,
unless the attack says otherwise.
\end{quote}

\begin{quote}
A creature grappled by the monster can use its action to try to escape.
To do so, it must succeed on a Strength (Athletics) or Dexterity
(Acrobatics) check against the escape DC in the monster's stat block. If
no escape DC is given, assume the DC is 10 + the monster's Strength
(Athletics) modifier.
\end{quote}

\hypertarget{equipment}{%
\subsection{Equipment}\label{equipment}}

A stat block rarely refers to equipment, other than armor or weapons
used by a monster. A creature that customarily wears clothes, such as a
humanoid, is assumed to be dressed appropriately.

You can equip monsters with additional gear and trinkets however you
like, and you decide how much of a monster's equipment is recoverable
after the creature is slain and whether any of that equipment is still
usable. A battered suit of armor made for a monster is rarely usable by
someone else, for instance.

If a spellcasting monster needs material components to cast its spells,
assume that it has the material components it needs to cast the spells
in its stat block.

\end{document}
