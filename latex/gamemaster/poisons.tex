% Options for packages loaded elsewhere
\PassOptionsToPackage{unicode}{hyperref}
\PassOptionsToPackage{hyphens}{url}
%
\documentclass[
]{article}
\usepackage{lmodern}
\usepackage{amssymb,amsmath}
\usepackage{ifxetex,ifluatex}
\ifnum 0\ifxetex 1\fi\ifluatex 1\fi=0 % if pdftex
  \usepackage[T1]{fontenc}
  \usepackage[utf8]{inputenc}
  \usepackage{textcomp} % provide euro and other symbols
\else % if luatex or xetex
  \usepackage{unicode-math}
  \defaultfontfeatures{Scale=MatchLowercase}
  \defaultfontfeatures[\rmfamily]{Ligatures=TeX,Scale=1}
\fi
% Use upquote if available, for straight quotes in verbatim environments
\IfFileExists{upquote.sty}{\usepackage{upquote}}{}
\IfFileExists{microtype.sty}{% use microtype if available
  \usepackage[]{microtype}
  \UseMicrotypeSet[protrusion]{basicmath} % disable protrusion for tt fonts
}{}
\makeatletter
\@ifundefined{KOMAClassName}{% if non-KOMA class
  \IfFileExists{parskip.sty}{%
    \usepackage{parskip}
  }{% else
    \setlength{\parindent}{0pt}
    \setlength{\parskip}{6pt plus 2pt minus 1pt}}
}{% if KOMA class
  \KOMAoptions{parskip=half}}
\makeatother
\usepackage{xcolor}
\IfFileExists{xurl.sty}{\usepackage{xurl}}{} % add URL line breaks if available
\IfFileExists{bookmark.sty}{\usepackage{bookmark}}{\usepackage{hyperref}}
\hypersetup{
  hidelinks,
  pdfcreator={LaTeX via pandoc}}
\urlstyle{same} % disable monospaced font for URLs
\usepackage{longtable,booktabs}
% Correct order of tables after \paragraph or \subparagraph
\usepackage{etoolbox}
\makeatletter
\patchcmd\longtable{\par}{\if@noskipsec\mbox{}\fi\par}{}{}
\makeatother
% Allow footnotes in longtable head/foot
\IfFileExists{footnotehyper.sty}{\usepackage{footnotehyper}}{\usepackage{footnote}}
\makesavenoteenv{longtable}
\setlength{\emergencystretch}{3em} % prevent overfull lines
\providecommand{\tightlist}{%
  \setlength{\itemsep}{0pt}\setlength{\parskip}{0pt}}
\setcounter{secnumdepth}{-\maxdimen} % remove section numbering

\date{}

\begin{document}

\hypertarget{poisons}{%
\section{Poisons}\label{poisons}}

Given their insidious and deadly nature, poisons are illegal in most
societies but are a favorite tool among assassins, drow, and other evil
creatures.

Poisons come in the following four types.

\textbf{Contact.} Contact poison can be smeared on an object and remains
potent until it is touched or washed off. A creature that touches
contact poison with exposed skin suffers its effects.

\textbf{Ingested.} A creature must swallow an entire dose of ingested
poison to suffer its effects. The dose can be delivered in food or a
liquid. You may decide that a partial dose has a reduced effect, such as
allowing advantage on the saving throw or dealing only half damage on a
failed save.

\textbf{Inhaled.} These poisons are powders or gases that take effect
when inhaled. Blowing the powder or releasing the gas subjects creatures
in a 5-foot cube to its effect. The resulting cloud dissipates
immediately afterward. Holding one 's breath is ineffective against
inhaled poisons, as they affect nasal membranes, tear ducts, and other
parts of the body.

\textbf{Injury.} Injury poison can be applied to weapons, ammunition,
trap components, and other objects that deal piercing or slashing damage
and remains potent until delivered through a wound or washed off. A
creature that takes piercing or slashing damage from an object coated
with the poison is exposed to its effects.

\hypertarget{poisons-1}{%
\paragraph{Poisons}\label{poisons-1}}

\begin{longtable}[]{@{}lll@{}}
\toprule
Item & Type & Price per Dose\tabularnewline
\midrule
\endhead
Assassin's blood & Ingested & 150~gp\tabularnewline
Burnt othur fumes & Inhaled & 500~gp\tabularnewline
Crawler mucus & Contact & 200~gp\tabularnewline
Drow poison & Injury & 200~gp\tabularnewline
Essence of ether & Inhaled & 300~gp\tabularnewline
Malice & Inhaled & 250~gp\tabularnewline
Midnight tears & Ingested & 1,500~gp\tabularnewline
Oil of taggit & Contact & 400~gp\tabularnewline
Pale tincture & Ingested & 250~gp\tabularnewline
Purple worm poison & Injury & 2,000~gp\tabularnewline
Serpent venom & Injury & 200~gp\tabularnewline
Torpor & Ingested & 600~gp\tabularnewline
Truth serum & Ingested & 150~gp\tabularnewline
Wyvern poison & Injury & 1,200~gp\tabularnewline
\bottomrule
\end{longtable}

\hypertarget{sample-poisons}{%
\subsection{Sample Poisons}\label{sample-poisons}}

Each type of poison has its own debilitating effects.

\textbf{Assassin's Blood (Ingested).} A creature subjected to this
poison must make a DC 10 Constitution saving throw. On a failed save, it
takes 6 (1d12) poison damage and is poisoned for 24 hours. On a
successful save, the creature takes half damage and isn't poisoned.

\textbf{Burnt Othur Fumes (Inhaled).} A creature subjected to this
poison must succeed on a DC 13 Constitution saving throw or take 10
(3d6) poison damage, and must repeat the saving throw at the start of
each of its turns. On each successive failed save, the character takes 3
(1d6) poison damage. After three successful saves, the poison ends.

\textbf{Crawler Mucus (Contact).} This poison must be harvested from a
dead or incapacitated crawler. A creature subjected to this poison must
succeed on a DC 13 Constitution saving throw or be poisoned for 1
minute. The poisoned creature is paralyzed. The creature can repeat the
saving throw at the end of each of its turns, ending the effect on
itself on a success.

\textbf{Drow Poison (Injury).} This poison is typically made only by the
drow, and only in a place far removed from sunlight. A creature
subjected to this poison must succeed on a DC 13 Constitution saving
throw or be poisoned for 1 hour. If the saving throw fails by 5 or more,
the creature is also unconscious while poisoned in this way. The
creature wakes up if it takes damage or if another creature takes an
action to shake it awake.

\textbf{Essence of Ether (Inhaled).} A creature subjected to this poison
must succeed on a DC 15 Constitution saving throw or become poisoned for
8 hours. The poisoned creature is unconscious. The creature wakes up if
it takes damage or if another creature takes an action to shake it
awake.

\textbf{Malice (Inhaled).} A creature subjected to this poison must
succeed on a DC 15 Constitution saving throw or become poisoned for 1
hour. The poisoned creature is blinded.

\textbf{Midnight Tears (Ingested).} A creature that ingests this poison
suffers no effect until the stroke of midnight. If the poison has not
been neutralized before then, the creature must succeed on a DC 17
Constitution saving throw, taking 31 (9d6) poison damage on a failed
save, or half as much damage on a successful one.

\textbf{Oil of Taggit (Contact).} A creature subjected to this poison
must succeed on a DC 13 Constitution saving throw or become poisoned for
24 hours. The poisoned creature is unconscious. The creature wakes up if
it takes damage.

\textbf{Pale Tincture (Ingested).} A creature subjected to this poison
must succeed on a DC 16 Constitution saving throw or take 3 (1d6) poison
damage and become poisoned. The poisoned creature must repeat the saving
throw every 24 hours, taking 3 (1d6) poison damage on a failed save.
Until this poison ends, the damage the poison deals can't be healed by
any means. After seven successful saving throws, the effect ends and the
creature can heal normally.

\textbf{Purple Worm Poison (Injury).} This poison must be harvested from
a dead or incapacitated purple worm. A creature subjected to this poison
must make a DC 19 Constitution saving throw, taking 42 (12d6) poison
damage on a failed save, or half as much damage on a successful one.

\textbf{Serpent Venom (Injury).} This poison must be harvested from a
dead or incapacitated giant poisonous snake. A creature subjected to
this poison must succeed on a DC 11 Constitution saving throw, taking 10
(3d6) poison damage on a failed save, or half as much damage on a
successful one.

\textbf{Torpor (Ingested).} A creature subjected to this poison must
succeed on a DC 15 Constitution saving throw or become poisoned for 4d6
hours. The poisoned creature is incapacitated.

\textbf{Truth Serum (Ingested).} A creature subjected to this poison
must succeed on a DC 11 Constitution saving throw or become poisoned for
1 hour. The poisoned creature can't knowingly speak a lie, as if under
the effect of a \emph{zone of truth} spell.

\textbf{Wyvern Poison (Injury).} This poison must be harvested from a
dead or incapacitated wyvern. A creature subjected to this poison must
make a DC 15 Constitution saving throw, taking 24 (7d6) poison damage on
a failed save, or half as much damage on a successful one.

\end{document}
