% Options for packages loaded elsewhere
\PassOptionsToPackage{unicode}{hyperref}
\PassOptionsToPackage{hyphens}{url}
%
\documentclass[
]{article}
\usepackage{lmodern}
\usepackage{amssymb,amsmath}
\usepackage{ifxetex,ifluatex}
\ifnum 0\ifxetex 1\fi\ifluatex 1\fi=0 % if pdftex
  \usepackage[T1]{fontenc}
  \usepackage[utf8]{inputenc}
  \usepackage{textcomp} % provide euro and other symbols
\else % if luatex or xetex
  \usepackage{unicode-math}
  \defaultfontfeatures{Scale=MatchLowercase}
  \defaultfontfeatures[\rmfamily]{Ligatures=TeX,Scale=1}
\fi
% Use upquote if available, for straight quotes in verbatim environments
\IfFileExists{upquote.sty}{\usepackage{upquote}}{}
\IfFileExists{microtype.sty}{% use microtype if available
  \usepackage[]{microtype}
  \UseMicrotypeSet[protrusion]{basicmath} % disable protrusion for tt fonts
}{}
\makeatletter
\@ifundefined{KOMAClassName}{% if non-KOMA class
  \IfFileExists{parskip.sty}{%
    \usepackage{parskip}
  }{% else
    \setlength{\parindent}{0pt}
    \setlength{\parskip}{6pt plus 2pt minus 1pt}}
}{% if KOMA class
  \KOMAoptions{parskip=half}}
\makeatother
\usepackage{xcolor}
\IfFileExists{xurl.sty}{\usepackage{xurl}}{} % add URL line breaks if available
\IfFileExists{bookmark.sty}{\usepackage{bookmark}}{\usepackage{hyperref}}
\hypersetup{
  hidelinks,
  pdfcreator={LaTeX via pandoc}}
\urlstyle{same} % disable monospaced font for URLs
\setlength{\emergencystretch}{3em} % prevent overfull lines
\providecommand{\tightlist}{%
  \setlength{\itemsep}{0pt}\setlength{\parskip}{0pt}}
\setcounter{secnumdepth}{-\maxdimen} % remove section numbering

\date{}

\begin{document}

\hypertarget{diseases}{%
\section{Diseases}\label{diseases}}

A plague ravages the kingdom, setting the adventurers on a quest to find
a cure. An adventurer emerges from an ancient tomb, unopened for
centuries, and soon finds herself suffering from a wasting illness. A
warlock offends some dark power and contracts a strange affliction that
spreads whenever he casts spells.

A simple outbreak might amount to little more than a small drain on
party resources, curable by a casting of \emph{lesser restoration}. A
more complicated outbreak can form the basis of one or more adventures
as characters search for a cure, stop the spread of the disease, and
deal with the consequences.

A disease that does more than infect a few party members is primarily a
plot device. The rules help describe the effects of the disease and how
it can be cured, but the specifics of how a disease works aren't bound
by a common set of rules. Diseases can affect any creature, and a given
illness might or might not pass from one race or kind of creature to
another. A plague might affect only constructs or undead, or sweep
through a halfling neighborhood but leave other races untouched. What
matters is the story you want to tell.

\hypertarget{sample-diseases}{%
\subsection{Sample Diseases}\label{sample-diseases}}

The diseases here illustrate the variety of ways disease can work in the
game. Feel free to alter the saving throw DCs, incubation times,
symptoms, and other characteristics of these diseases to suit your
campaign.

\hypertarget{cackle-fever}{%
\subsubsection{Cackle Fever}\label{cackle-fever}}

This disease targets humanoids, although gnomes are strangely immune.
While in the grips of this disease, victims frequently succumb to fits
of mad laughter, giving the disease its common name and its morbid
nickname: ``the shrieks.''

Symptoms manifest 1d4 hours after infection and include fever and
disorientation. The infected creature gains one level of exhaustion that
can't be removed until the disease is cured.

Any event that causes the infected creature great stress---including
entering combat, taking damage, experiencing fear, or having a
nightmare---forces the creature to make a DC 13 Constitution saving
throw. On a failed save, the creature takes 5 (1d10) psychic damage and
becomes incapacitated with mad laughter for 1 minute. The creature can
repeat the saving throw at the end of each of its turns, ending the mad
laughter and the incapacitated condition on a success.

Any humanoid creature that starts its turn within 10 feet of an infected
creature in the throes of mad laughter must succeed on a DC 10
Constitution saving throw or also become infected with the disease. Once
a creature succeeds on this save, it is immune to the mad laughter of
that particular infected creature for 24 hours.

At the end of each long rest, an infected creature can make a DC 13
Constitution saving throw. On a successful save, the DC for this save
and for the save to avoid an attack of mad laughter drops by 1d6. When
the saving throw DC drops to 0, the creature recovers from the disease.
A creature that fails three of these saving throws gains a randomly
determined form of indefinite madness, as described later.

\hypertarget{sewer-plague}{%
\subsubsection{Sewer Plague}\label{sewer-plague}}

Sewer plague is a generic term for a broad category of illnesses that
incubate in sewers, refuse heaps, and stagnant swamps, and which are
sometimes transmitted by creatures that dwell in those areas, such as
rats and otyughs.

When a humanoid creature is bitten by a creature that carries the
disease, or when it comes into contact with filth or offal contaminated
by the disease, the creature must succeed on a DC 11 Constitution saving
throw or become infected.

It takes 1d4 days for sewer plague's symptoms to manifest in an infected
creature. Symptoms include fatigue and cramps. The infected creature
suffers one level of exhaustion, and it regains only half the normal
number of hit points from spending Hit Dice and no hit points from
finishing a long rest.

At the end of each long rest, an infected creature must make a DC 11
Constitution saving throw. On a failed save, the character gains one
level of exhaustion. On a successful save, the character's exhaustion
level decreases by one level. If a successful saving throw reduces the
infected creature's level of exhaustion below 1, the creature recovers
from the disease.

\hypertarget{sight-rot}{%
\subsubsection{Sight Rot}\label{sight-rot}}

This painful infection causes bleeding from the eyes and eventually
blinds the victim.

A beast or humanoid that drinks water tainted by sight rot must succeed
on a DC 15 Constitution saving throw or become infected. One day after
infection, the creature's vision starts to become blurry. The creature
takes a -1 penalty to attack rolls and ability checks that rely on
sight. At the end of each long rest after the symptoms appear, the
penalty worsens by 1. When it reaches -5, the victim is blinded until
its sight is restored by magic such as \emph{lesser restoration} or
\emph{heal}.

Sight rot can be cured using a rare flower called Eyebright, which grows
in some swamps. Given an hour, a character who has proficiency with an
herbalism kit can turn the flower into one dose of ointment. Applied to
the eyes before a long rest, one dose of it prevents the disease from
worsening after that rest. After three doses, the ointment cures the
disease entirely.

\end{document}
