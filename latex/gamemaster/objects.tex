% Options for packages loaded elsewhere
\PassOptionsToPackage{unicode}{hyperref}
\PassOptionsToPackage{hyphens}{url}
%
\documentclass[
]{article}
\usepackage{lmodern}
\usepackage{amssymb,amsmath}
\usepackage{ifxetex,ifluatex}
\ifnum 0\ifxetex 1\fi\ifluatex 1\fi=0 % if pdftex
  \usepackage[T1]{fontenc}
  \usepackage[utf8]{inputenc}
  \usepackage{textcomp} % provide euro and other symbols
\else % if luatex or xetex
  \usepackage{unicode-math}
  \defaultfontfeatures{Scale=MatchLowercase}
  \defaultfontfeatures[\rmfamily]{Ligatures=TeX,Scale=1}
\fi
% Use upquote if available, for straight quotes in verbatim environments
\IfFileExists{upquote.sty}{\usepackage{upquote}}{}
\IfFileExists{microtype.sty}{% use microtype if available
  \usepackage[]{microtype}
  \UseMicrotypeSet[protrusion]{basicmath} % disable protrusion for tt fonts
}{}
\makeatletter
\@ifundefined{KOMAClassName}{% if non-KOMA class
  \IfFileExists{parskip.sty}{%
    \usepackage{parskip}
  }{% else
    \setlength{\parindent}{0pt}
    \setlength{\parskip}{6pt plus 2pt minus 1pt}}
}{% if KOMA class
  \KOMAoptions{parskip=half}}
\makeatother
\usepackage{xcolor}
\IfFileExists{xurl.sty}{\usepackage{xurl}}{} % add URL line breaks if available
\IfFileExists{bookmark.sty}{\usepackage{bookmark}}{\usepackage{hyperref}}
\hypersetup{
  hidelinks,
  pdfcreator={LaTeX via pandoc}}
\urlstyle{same} % disable monospaced font for URLs
\usepackage{longtable,booktabs}
% Correct order of tables after \paragraph or \subparagraph
\usepackage{etoolbox}
\makeatletter
\patchcmd\longtable{\par}{\if@noskipsec\mbox{}\fi\par}{}{}
\makeatother
% Allow footnotes in longtable head/foot
\IfFileExists{footnotehyper.sty}{\usepackage{footnotehyper}}{\usepackage{footnote}}
\makesavenoteenv{longtable}
\setlength{\emergencystretch}{3em} % prevent overfull lines
\providecommand{\tightlist}{%
  \setlength{\itemsep}{0pt}\setlength{\parskip}{0pt}}
\setcounter{secnumdepth}{-\maxdimen} % remove section numbering

\date{}

\begin{document}

\hypertarget{objects}{%
\section{Objects}\label{objects}}

When characters need to saw through ropes, shatter a window, or smash a
vampire's coffin, the only hard and fast rule is this: given enough time
and the right tools, characters can destroy any destructible object. Use
common sense when determining a character's success at damaging an
object. Can a fighter cut through a section of a stone wall with a
sword? No, the sword is likely to break before the wall does.

For the purpose of these rules, an object is a discrete, inanimate item
like a window, door, sword, book, table, chair, or stone, not a building
or a vehicle that is composed of many other objects.

\hypertarget{statistics-for-objects}{%
\subsection{Statistics for Objects}\label{statistics-for-objects}}

When time is a factor, you can assign an Armor Class and hit points to a
destructible object. You can also give it immunities, resistances, and
vulnerabilities to specific types of damage.

\textbf{Armor Class.} An object's Armor Class is a measure of how
difficult it is to deal damage to the object when striking it (because
the object has no chance of dodging out of the way). The Object Armor
Class table provides suggested AC values for various substances.

\hypertarget{object-armor-class}{%
\paragraph{Object Armor Class}\label{object-armor-class}}

\begin{longtable}[]{@{}ll@{}}
\toprule
Substance & AC\tabularnewline
\midrule
\endhead
Cloth, paper, rope & 11\tabularnewline
Crystal, glass, ice & 13\tabularnewline
Wood, bone & 15\tabularnewline
Stone & 17\tabularnewline
Iron, steel & 19\tabularnewline
Mithral & 21\tabularnewline
Adamantine & 23\tabularnewline
\bottomrule
\end{longtable}

\textbf{Hit Points.} An object's hit points measure how much damage it
can take before losing its structural integrity. Resilient objects have
more hit points than fragile ones. Large objects also tend to have more
hit points than small ones, unless breaking a small part of the object
is just as effective as breaking the whole thing. The Object Hit Points
table provides suggested hit points for fragile and resilient objects
that are Large or smaller.

\hypertarget{object-hit-points}{%
\paragraph{Object Hit Points}\label{object-hit-points}}

\begin{longtable}[]{@{}lll@{}}
\toprule
Size & Fragile & Resilient\tabularnewline
\midrule
\endhead
Tiny (bottle, lock) & 2~(1d4) & 5~(2d4)\tabularnewline
Small (chest, lute) & 3~(1d6) & 10~(3d6)\tabularnewline
Medium (barrel, chandelier) & 4~(1d8) & 18~(4d8)\tabularnewline
Large (cart, 10-ft.-by--10-ft. window) & 5~(1d10) &
27~(5d10)\tabularnewline
\bottomrule
\end{longtable}

\textbf{Huge and Gargantuan Objects.} Normal weapons are of little use
against many Huge and Gargantuan objects, such as a colossal statue,
towering column of stone, or massive boulder. That said, one torch can
burn a Huge tapestry, and an \emph{earthquake} spell can reduce a
colossus to rubble. You can track a Huge or Gargantuan object's hit
points if you like, or you can simply decide how long the object can
withstand whatever weapon or force is acting against it. If you track
hit points for the object, divide it into Large or smaller sections, and
track each section's hit points separately. Destroying one of those
sections could ruin the entire object. For example, a Gargantuan statue
of a human might topple over when one of its Large legs is reduced to 0
hit points.

\textbf{Objects and Damage Types.} Objects are immune to poison and
psychic damage. You might decide that some damage types are more
effective against a particular object or substance than others. For
example, bludgeoning damage works well for smashing things but not for
cutting through rope or leather. Paper or cloth objects might be
vulnerable to fire and lightning damage. A pick can chip away stone but
can't effectively cut down a tree. As always, use your best judgment.

\textbf{Damage Threshold.} Big objects such as castle walls often have
extra resilience represented by a damage threshold. An object with a
damage threshold has immunity to all damage unless it takes an amount of
damage from a single attack or effect equal to or greater than its
damage threshold, in which case it takes damage as normal. Any damage
that fails to meet or exceed the object's damage threshold is considered
superficial and doesn't reduce the object's hit points.

\end{document}
