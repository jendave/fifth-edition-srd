% Options for packages loaded elsewhere
\PassOptionsToPackage{unicode}{hyperref}
\PassOptionsToPackage{hyphens}{url}
%
\documentclass[
]{article}
\usepackage{lmodern}
\usepackage{amssymb,amsmath}
\usepackage{ifxetex,ifluatex}
\ifnum 0\ifxetex 1\fi\ifluatex 1\fi=0 % if pdftex
  \usepackage[T1]{fontenc}
  \usepackage[utf8]{inputenc}
  \usepackage{textcomp} % provide euro and other symbols
\else % if luatex or xetex
  \usepackage{unicode-math}
  \defaultfontfeatures{Scale=MatchLowercase}
  \defaultfontfeatures[\rmfamily]{Ligatures=TeX,Scale=1}
\fi
% Use upquote if available, for straight quotes in verbatim environments
\IfFileExists{upquote.sty}{\usepackage{upquote}}{}
\IfFileExists{microtype.sty}{% use microtype if available
  \usepackage[]{microtype}
  \UseMicrotypeSet[protrusion]{basicmath} % disable protrusion for tt fonts
}{}
\makeatletter
\@ifundefined{KOMAClassName}{% if non-KOMA class
  \IfFileExists{parskip.sty}{%
    \usepackage{parskip}
  }{% else
    \setlength{\parindent}{0pt}
    \setlength{\parskip}{6pt plus 2pt minus 1pt}}
}{% if KOMA class
  \KOMAoptions{parskip=half}}
\makeatother
\usepackage{xcolor}
\IfFileExists{xurl.sty}{\usepackage{xurl}}{} % add URL line breaks if available
\IfFileExists{bookmark.sty}{\usepackage{bookmark}}{\usepackage{hyperref}}
\hypersetup{
  hidelinks,
  pdfcreator={LaTeX via pandoc}}
\urlstyle{same} % disable monospaced font for URLs
\usepackage{longtable,booktabs}
% Correct order of tables after \paragraph or \subparagraph
\usepackage{etoolbox}
\makeatletter
\patchcmd\longtable{\par}{\if@noskipsec\mbox{}\fi\par}{}{}
\makeatother
% Allow footnotes in longtable head/foot
\IfFileExists{footnotehyper.sty}{\usepackage{footnotehyper}}{\usepackage{footnote}}
\makesavenoteenv{longtable}
\setlength{\emergencystretch}{3em} % prevent overfull lines
\providecommand{\tightlist}{%
  \setlength{\itemsep}{0pt}\setlength{\parskip}{0pt}}
\setcounter{secnumdepth}{-\maxdimen} % remove section numbering

\date{}

\begin{document}

\hypertarget{traps}{%
\section{Traps}\label{traps}}

Traps can be found almost anywhere. One wrong step in an ancient tomb
might trigger a series of scything blades, which cleave through armor
and bone. The seemingly innocuous vines that hang over a cave entrance
might grasp and choke anyone who pushes through them. A net hidden among
the trees might drop on travelers who pass underneath. In a fantasy
game, unwary adventurers can fall to their deaths, be burned alive, or
fall under a fusillade of poisoned darts.

A trap can be either mechanical or magical in nature. \textbf{Mechanical
traps} include pits, arrow traps, falling blocks, water-filled rooms,
whirling blades, and anything else that depends on a mechanism to
operate. \textbf{Magic traps} are either magical device traps or spell
traps. Magical device traps initiate spell effects when activated. Spell
traps are spells such as \emph{glyph of warding} and \emph{symbol} that
function as traps.

\hypertarget{traps-in-play}{%
\subsection{Traps in Play}\label{traps-in-play}}

When adventurers come across a trap, you need to know how the trap is
triggered and what it does, as well as the possibility for the
characters to detect the trap and to disable or avoid it.

\hypertarget{triggering-a-trap}{%
\subsubsection{Triggering a Trap}\label{triggering-a-trap}}

Most traps are triggered when a creature goes somewhere or touches
something that the trap's creator wanted to protect. Common triggers
include stepping on a pressure plate or a false section of floor,
pulling a trip wire, turning a doorknob, and using the wrong key in a
lock. Magic traps are often set to go off when a creature enters an area
or touches an object. Some magic traps (such as the \emph{glyph of
warding} spell) have more complicated trigger conditions, including a
password that prevents the trap from activating.

\hypertarget{detecting-and-disabling-a-trap}{%
\subsubsection{Detecting and Disabling a
Trap}\label{detecting-and-disabling-a-trap}}

Usually, some element of a trap is visible to careful inspection.
Characters might notice an uneven flagstone that conceals a pressure
plate, spot the gleam of light off a trip wire, notice small holes in
the walls from which jets of flame will erupt, or otherwise detect
something that points to a trap's presence.

A trap's description specifies the checks and DCs needed to detect it,
disable it, or both. A character actively looking for a trap can attempt
a Wisdom (Perception) check against the trap's DC. You can also compare
the DC to detect the trap with each character's passive Wisdom
(Perception) score to determine whether anyone in the party notices the
trap in passing. If the adventurers detect a trap before triggering it,
they might be able to disarm it, either permanently or long enough to
move past it. You might call for an Intelligence (Investigation) check
for a character to deduce what needs to be done, followed by a Dexterity
check using thieves' tools to perform the necessary sabotage.

Any character can attempt an Intelligence (Arcana) check to detect or
disarm a magic trap, in addition to any other checks noted in the trap's
description. The DCs are the same regardless of the check used. In
addition, \emph{dispel magic} has a chance of disabling most magic
traps. A magic trap's description provides the DC for the ability check
made when you use \emph{dispel magic} .

In most cases, a trap's description is clear enough that you can
adjudicate whether a character's actions locate or foil the trap. As
with many situations, you shouldn't allow die rolling to override clever
play and good planning. Use your common sense, drawing on the trap's
description to determine what happens. No trap's design can anticipate
every possible action that the characters might attempt.

You should allow a character to discover a trap without making an
ability check if an action would clearly reveal the trap's presence. For
example, if a character lifts a rug that conceals a pressure plate, the
character has found the trigger and no check is required.

Foiling traps can be a little more complicated. Consider a trapped
treasure chest. If the chest is opened without first pulling on the two
handles set in its sides, a mechanism inside fires a hail of poison
needles toward anyone in front of it. After inspecting the chest and
making a few checks, the characters are still unsure if it's trapped.
Rather than simply open the chest, they prop a shield in front of it and
push the chest open at a distance with an iron rod. In this case, the
trap still triggers, but the hail of needles fires harmlessly into the
shield.

Traps are often designed with mechanisms that allow them to be disarmed
or bypassed. Intelligent monsters that place traps in or around their
lairs need ways to get past those traps without harming themselves. Such
traps might have hidden levers that disable their triggers, or a secret
door might conceal a passage that goes around the trap.

\hypertarget{trap-effects}{%
\subsubsection{Trap Effects}\label{trap-effects}}

The effects of traps can range from inconvenient to deadly, making use
of elements such as arrows, spikes, blades, poison, toxic gas, blasts of
fire, and deep pits. The deadliest traps combine multiple elements to
kill, injure, contain, or drive off any creature unfortunate enough to
trigger them. A trap's description specifies what happens when it is
triggered.

The attack bonus of a trap, the save DC to resist its effects, and the
damage it deals can vary depending on the trap's severity. Use the Trap
Save DCs and Attack Bonuses table and the Damage Severity by Level table
for suggestions based on three levels of trap severity.

A trap intended to be a \textbf{setback} is unlikely to kill or
seriously harm characters of the indicated levels, whereas a
\textbf{dangerous} trap is likely to seriously injure (and potentially
kill) characters of the indicated levels. A \textbf{deadly} trap is
likely to kill characters of the indicated levels.

\hypertarget{trap-save-dcs-and-attack-bonuses}{%
\paragraph{Trap Save DCs and Attack
Bonuses}\label{trap-save-dcs-and-attack-bonuses}}

\begin{longtable}[]{@{}lll@{}}
\toprule
Trap Danger & Save DC & Attack Bonus\tabularnewline
\midrule
\endhead
Setback & 10--11 & +3 to +5\tabularnewline
Dangerous & 12--15 & +6 to +8\tabularnewline
Deadly & 16--20 & +9 to +12\tabularnewline
\bottomrule
\end{longtable}

\hypertarget{damage-severity-by-level}{%
\paragraph{Damage Severity by Level}\label{damage-severity-by-level}}

\begin{longtable}[]{@{}llll@{}}
\toprule
Character Level & Setback & Dangerous & Deadly\tabularnewline
\midrule
\endhead
1st--4th & 1d10 & 2d10 & 4d10\tabularnewline
5th--10th & 2d10 & 4d10 & 10d10\tabularnewline
11th--16th & 4d10 & 10d10 & 18d10\tabularnewline
17th--20th & 10d10 & 18d10 & 24d10\tabularnewline
\bottomrule
\end{longtable}

\hypertarget{complex-traps}{%
\subsubsection{Complex Traps}\label{complex-traps}}

Complex traps work like standard traps, except once activated they
execute a series of actions each round. A complex trap turns the process
of dealing with a trap into something more like a combat encounter.

When a complex trap activates, it rolls initiative. The trap's
description includes an initiative bonus. On its turn, the trap
activates again, often taking an action. It might make successive
attacks against intruders, create an effect that changes over time, or
otherwise produce a dynamic challenge. Otherwise, the complex trap can
be detected and disabled or bypassed in the usual ways.

For example, a trap that causes a room to slowly flood works best as a
complex trap. On the trap's turn, the water level rises. After several
rounds, the room is completely flooded.

\hypertarget{sample-traps}{%
\subsection{Sample Traps}\label{sample-traps}}

The magical and mechanical traps presented here vary in deadliness and
are presented in alphabetical order.

\hypertarget{collapsing-roof}{%
\subsubsection{Collapsing Roof}\label{collapsing-roof}}

\emph{Mechanical trap}

This trap uses a trip wire to collapse the supports keeping an unstable
section of a ceiling in place.

The trip wire is 3 inches off the ground and stretches between two
support beams. The DC to spot the trip wire is 10. A successful DC 15
Dexterity check using thieves' tools disables the trip wire harmlessly.
A character without thieves' tools can attempt this check with
disadvantage using any edged weapon or edged tool. On a failed check,
the trap triggers.

Anyone who inspects the beams can easily determine that they are merely
wedged in place. As an action, a character can knock over a beam,
causing the trap to trigger.

The ceiling above the trip wire is in bad repair, and anyone who can see
it can tell that it's in danger of collapse.

When the trap is triggered, the unstable ceiling collapses. Any creature
in the area beneath the unstable section must succeed on a DC 15
Dexterity saving throw, taking 22 (4d10) bludgeoning damage on a failed
save, or half as much damage on a successful one. Once the trap is
triggered, the floor of the area is filled with rubble and becomes
difficult terrain.

\hypertarget{falling-net}{%
\subsubsection{Falling Net}\label{falling-net}}

\emph{Mechanical trap}

This trap uses a trip wire to release a net suspended from the ceiling.

The trip wire is 3 inches off the ground and stretches between two
columns or trees. The net is hidden by cobwebs or foliage. The DC to
spot the trip wire and net is 10. A successful DC 15 Dexterity check
using thieves' tools breaks the trip wire harmlessly. A character
without thieves' tools can attempt this check with disadvantage using
any edged weapon or edged tool. On a failed check, the trap triggers.

When the trap is triggered, the net is released, covering a
10-foot-square area. Those in the area are trapped under the net and
restrained, and those that fail a DC 10 Strength saving throw are also
knocked prone. A creature can use its action to make a DC 10 Strength
check, freeing itself or another creature within its reach on a success.
The net has AC 10 and 20 hit points. Dealing 5 slashing damage to the
net (AC 10) destroys a 5-foot-square section of it, freeing any creature
trapped in that section.

\hypertarget{fire-breathing-statue}{%
\subsubsection{Fire-Breathing Statue}\label{fire-breathing-statue}}

\emph{Magic trap}

This trap is activated when an intruder steps on a hidden pressure
plate, releasing a magical gout of flame from a nearby statue. The
statue can be of anything, including a dragon or a wizard casting a
spell.

The DC is 15 to spot the pressure plate, as well as faint scorch marks
on the floor and walls. A spell or other effect that can sense the
presence of magic, such as \emph{detect magic} , reveals an aura of
evocation magic around the statue.

The trap activates when more than 20 pounds of weight is placed on the
pressure plate, causing the statue to release a 30-foot cone of fire.
Each creature in the fire must make a DC 13 Dexterity saving throw,
taking 22 (4d10) fire damage on a failed save, or half as much damage on
a successful one.

Wedging an iron spike or other object under the pressure plate prevents
the trap from activating. A successful \emph{dispel magic} (DC 13) cast
on the statue destroys the trap.

\hypertarget{pits}{%
\subsubsection{Pits}\label{pits}}

\emph{Mechanical trap}

Four basic pit traps are presented here.

\textbf{Simple Pit.} A simple pit trap is a hole dug in the ground. The
hole is covered by a large cloth anchored on the pit's edge and
camouflaged with dirt and debris.

The DC to spot the pit is 10. Anyone stepping on the cloth falls through
and pulls the cloth down into the pit, taking damage based on the pit's
depth (usually 10 feet, but some pits are deeper).

\textbf{Hidden Pit.} This pit has a cover constructed from material
identical to the floor around it.

A successful DC 15 Wisdom (Perception) check discerns an absence of foot
traffic over the section of floor that forms the pit's cover. A
successful DC 15 Intelligence (Investigation) check is necessary to
confirm that the trapped section of floor is actually the cover of a
pit.

When a creature steps on the cover, it swings open like a trapdoor,
causing the intruder to spill into the pit below. The pit is usually 10
or 20 feet deep but can be deeper.

Once the pit trap is detected, an iron spike or similar object can be
wedged between the pit's cover and the surrounding floor in such a way
as to prevent the cover from opening, thereby making it safe to cross.
The cover can also be magically held shut using the \emph{arcane lock}
spell or similar magic.

\textbf{Locking Pit.} This pit trap is identical to a hidden pit trap,
with one key exception: the trap door that covers the pit is
spring-loaded. After a creature falls into the pit, the cover snaps shut
to trap its victim inside.

A successful DC 20 Strength check is necessary to pry the cover open.
The cover can also be smashed open. A character in the pit can also
attempt to disable the spring mechanism from the inside with a DC 15
Dexterity check using thieves' tools, provided that the mechanism can be
reached and the character can see. In some cases, a mechanism (usually
hidden behind a secret door nearby) opens the pit.

\textbf{Spiked Pit.} This pit trap is a simple, hidden, or locking pit
trap with sharpened wooden or iron spikes at the bottom. A creature
falling into the pit takes 11 (2d10) piercing damage from the spikes, in
addition to any falling damage. Even nastier versions have poison
smeared on the spikes. In that case, anyone taking piercing damage from
the spikes must also make a DC 13 Constitution saving throw, taking an
22 (4d10) poison damage on a failed save, or half as much damage on a
successful one.

\hypertarget{poison-darts}{%
\subsubsection{Poison Darts}\label{poison-darts}}

\emph{Mechanical trap}

When a creature steps on a hidden pressure plate, poison-tipped darts
shoot from spring-loaded or pressurized tubes cleverly embedded in the
surrounding walls. An area might include multiple pressure plates, each
one rigged to its own set of darts.

The tiny holes in the walls are obscured by dust and cobwebs, or
cleverly hidden amid bas-reliefs, murals, or frescoes that adorn the
walls. The DC to spot them is 15. With a successful DC 15 Intelligence
(Investigation) check, a character can deduce the presence of the
pressure plate from variations in the mortar and stone used to create
it, compared to the surrounding floor. Wedging an iron spike or other
object under the pressure plate prevents the trap from activating.
Stuffing the holes with cloth or wax prevents the darts contained within
from launching.

The trap activates when more than 20 pounds of weight is placed on the
pressure plate, releasing four darts. Each dart makes a ranged attack
with a +8 bonus against a random target within 10 feet of the pressure
plate (vision is irrelevant to this attack roll). (If there are no
targets in the area, the darts don't hit anything.) A target that is hit
takes 2 (1d4) piercing damage and must succeed on a DC 15 Constitution
saving throw, taking 11 (2d10) poison damage on a failed save, or half
as much damage on a successful one.

\hypertarget{poison-needle}{%
\subsubsection{Poison Needle}\label{poison-needle}}

\emph{Mechanical trap}

A poisoned needle is hidden within a treasure chest's lock, or in
something else that a creature might open. Opening the chest without the
proper key causes the needle to spring out, delivering a dose of poison.

When the trap is triggered, the needle extends 3 inches straight out
from the lock. A creature within range takes 1 piercing damage and 11
(2d10) poison damage, and must succeed on a DC 15 Constitution saving
throw or be poisoned for 1 hour.

A successful DC 20 Intelligence (Investigation) check allows a character
to deduce the trap's presence from alterations made to the lock to
accommodate the needle. A successful DC 15 Dexterity check using
thieves' tools disarms the trap, removing the needle from the lock.
Unsuccessfully attempting to pick the lock triggers the trap.

\hypertarget{rolling-sphere}{%
\subsubsection{Rolling Sphere}\label{rolling-sphere}}

\emph{Mechanical trap}

When 20 or more pounds of pressure are placed on this trap's pressure
plate, a hidden trapdoor in the ceiling opens, releasing a
10-foot-diameter rolling sphere of solid stone.

With a successful DC 15 Wisdom (Perception) check, a character can spot
the trapdoor and pressure plate. A search of the floor accompanied by a
successful DC 15 Intelligence (Investigation) check reveals variations
in the mortar and stone that betray the pressure plate's presence. The
same check made while inspecting the ceiling notes variations in the
stonework that reveal the trapdoor. Wedging an iron spike or other
object under the pressure plate prevents the trap from activating.

Activation of the sphere requires all creatures present to roll
initiative. The sphere rolls initiative with a +8 bonus. On its turn, it
moves 60 feet in a straight line. The sphere can move through creatures'
spaces, and creatures can move through its space, treating it as
difficult terrain. Whenever the sphere enters a creature's space or a
creature enters its space while it's rolling, that creature must succeed
on a DC 15 Dexterity saving throw or take 55 (10d10) bludgeoning damage
and be knocked prone.

The sphere stops when it hits a wall or similar barrier. It can't go
around corners, but smart dungeon builders incorporate gentle, curving
turns into nearby passages that allow the sphere to keep moving.

As an action, a creature within 5 feet of the sphere can attempt to slow
it down with a DC 20 Strength check. On a successful check, the sphere's
speed is reduced by 15 feet. If the sphere's speed drops to 0, it stops
moving and is no longer a threat.

\hypertarget{sphere-of-annihilation}{%
\subsubsection{Sphere of Annihilation}\label{sphere-of-annihilation}}

\emph{Magic trap}

Magical, impenetrable darkness fills the gaping mouth of a stone face
carved into a wall. The mouth is 2 feet in diameter and roughly
circular. No sound issues from it, no light can illuminate the inside of
it, and any matter that enters it is instantly obliterated.

A successful DC 20 Intelligence (Arcana) check reveals that the mouth
contains a \emph{sphere of annihilation} that can't be controlled or
moved. It is otherwise identical to a normal \emph{sphere of
annihilation} .

Some versions of the trap include an enchantment placed on the stone
face, such that specified creatures feel an overwhelming urge to
approach it and crawl inside its mouth. This effect is otherwise like
the \emph{sympathy} aspect of the \emph{antipathy/sympathy} spell. A
successful \emph{dispel magic} (DC 18) removes this enchantment.

\end{document}
