% Options for packages loaded elsewhere
\PassOptionsToPackage{unicode}{hyperref}
\PassOptionsToPackage{hyphens}{url}
%
\documentclass[
]{article}
\usepackage{lmodern}
\usepackage{amssymb,amsmath}
\usepackage{ifxetex,ifluatex}
\ifnum 0\ifxetex 1\fi\ifluatex 1\fi=0 % if pdftex
  \usepackage[T1]{fontenc}
  \usepackage[utf8]{inputenc}
  \usepackage{textcomp} % provide euro and other symbols
\else % if luatex or xetex
  \usepackage{unicode-math}
  \defaultfontfeatures{Scale=MatchLowercase}
  \defaultfontfeatures[\rmfamily]{Ligatures=TeX,Scale=1}
\fi
% Use upquote if available, for straight quotes in verbatim environments
\IfFileExists{upquote.sty}{\usepackage{upquote}}{}
\IfFileExists{microtype.sty}{% use microtype if available
  \usepackage[]{microtype}
  \UseMicrotypeSet[protrusion]{basicmath} % disable protrusion for tt fonts
}{}
\makeatletter
\@ifundefined{KOMAClassName}{% if non-KOMA class
  \IfFileExists{parskip.sty}{%
    \usepackage{parskip}
  }{% else
    \setlength{\parindent}{0pt}
    \setlength{\parskip}{6pt plus 2pt minus 1pt}}
}{% if KOMA class
  \KOMAoptions{parskip=half}}
\makeatother
\usepackage{xcolor}
\IfFileExists{xurl.sty}{\usepackage{xurl}}{} % add URL line breaks if available
\IfFileExists{bookmark.sty}{\usepackage{bookmark}}{\usepackage{hyperref}}
\hypersetup{
  hidelinks,
  pdfcreator={LaTeX via pandoc}}
\urlstyle{same} % disable monospaced font for URLs
\usepackage{longtable,booktabs}
% Correct order of tables after \paragraph or \subparagraph
\usepackage{etoolbox}
\makeatletter
\patchcmd\longtable{\par}{\if@noskipsec\mbox{}\fi\par}{}{}
\makeatother
% Allow footnotes in longtable head/foot
\IfFileExists{footnotehyper.sty}{\usepackage{footnotehyper}}{\usepackage{footnote}}
\makesavenoteenv{longtable}
\setlength{\emergencystretch}{3em} % prevent overfull lines
\providecommand{\tightlist}{%
  \setlength{\itemsep}{0pt}\setlength{\parskip}{0pt}}
\setcounter{secnumdepth}{-\maxdimen} % remove section numbering

\date{}

\begin{document}

\hypertarget{madness}{%
\section{Madness}\label{madness}}

In a typical campaign, characters aren't driven mad by the horrors they
face and the carnage they inflict day after day, but sometimes the
stress of being an adventurer can be too much to bear. If your campaign
has a strong horror theme, you might want to use madness as a way to
reinforce that theme, emphasizing the extraordinarily horrific nature of
the threats the adventurers face.

\hypertarget{going-mad}{%
\subsection{Going Mad}\label{going-mad}}

Various magical effects can inflict madness on an otherwise stable mind.
Certain spells, such as \emph{contact other plane} and \emph{symbol},
can cause insanity, and you can use the madness rules here instead of
the spell effects of those spells. Diseases, poisons, and planar effects
such as psychic wind or the howling winds of Pandemonium can all inflict
madness. Some artifacts can also break the psyche of a character who
uses or becomes attuned to them.

Resisting a madness-inducing effect usually requires a Wisdom or
Charisma saving throw.

\hypertarget{madness-effects}{%
\subsection{Madness Effects}\label{madness-effects}}

Madness can be short-term, long-term, or indefinite. Most relatively
mundane effects impose short-term madness, which lasts for just a few
minutes. More horrific effects or cumulative effects can result in
long-term or indefinite madness.

A character afflicted with \textbf{short-term madness} is subjected to
an effect from the Short-Term Madness table for 1d10 minutes.

A character afflicted with \textbf{long-term madness} is subjected to an
effect from the Long-Term Madness table for 1d10 × 10 hours.

A character afflicted with \textbf{indefinite madness} gains a new
character flaw from the Indefinite Madness table that lasts until cured.

\hypertarget{short-term-madness}{%
\paragraph{Short-Term Madness}\label{short-term-madness}}

\begin{longtable}[]{@{}ll@{}}
\toprule
d100 & Effect (lasts 1d10 minutes)\tabularnewline
\midrule
\endhead
01--20 & The character retreats into his or her mind and becomes
paralyzed. The effect ends if the character takes any
damage.\tabularnewline
21--30 & The character becomes incapacitated and spends the duration
screaming, laughing, or weeping.\tabularnewline
31--40 & The character becomes frightened and must use his or her action
and movement each round to flee from the source of the
fear.\tabularnewline
41--50 & The character begins babbling and is incapable of normal speech
or spellcasting.\tabularnewline
51--60 & The character must use his or her action each round to attack
the nearest creature.\tabularnewline
61--70 & The character experiences vivid hallucinations and has
disadvantage on ability checks.\tabularnewline
71--75 & The character does whatever anyone tells him or her to do that
isn't obviously self-destructive.\tabularnewline
76--80 & The character experiences an overpowering urge to eat something
strange such as dirt, slime, or offal.\tabularnewline
81--90 & The character is stunned.\tabularnewline
91--100 & The character falls unconscious.\tabularnewline
\bottomrule
\end{longtable}

\hypertarget{long-term-madness}{%
\paragraph{Long-Term Madness}\label{long-term-madness}}

\begin{longtable}[]{@{}ll@{}}
\toprule
d100 & Effect (lasts 1d10 × 10 hours)\tabularnewline
\midrule
\endhead
01--10 & The character feels compelled to repeat a specific activity
over and over, such as washing hands, touching things, praying, or
counting coins.\tabularnewline
11--20 & The character experiences vivid hallucinations and has
disadvantage on ability checks.\tabularnewline
21--30 & The character suffers extreme paranoia. The character has
disadvantage on Wisdom and Charisma checks.\tabularnewline
31--40 & The character regards something (usually the source of madness)
with intense revulsion, as if affected by the antipathy effect of the
antipathy/sympathy spell.\tabularnewline
41--45 & The character experiences a powerful delusion. Choose a potion.
The character imagines that he or she is under its
effects.\tabularnewline
46--55 & The character becomes attached to a ``lucky charm,'' such as a
person or an object, and has disadvantage on attack rolls, ability
checks, and saving throws while more than 30 feet from
it.\tabularnewline
56--65 & The character is blinded (25\%) or deafened
(75\%).\tabularnewline
66--75 & The character experiences uncontrollable tremors or tics, which
impose disadvantage on attack rolls, ability checks, and saving throws
that involve Strength or Dexterity.\tabularnewline
76--85 & The character suffers from partial amnesia. The character knows
who he or she is and retains racial traits and class features, but
doesn't recognize other people or remember anything that happened before
the madness took effect.\tabularnewline
86--90 & Whenever the character takes damage, he or she must succeed on
a DC 15 Wisdom saving throw or be affected as though he or she failed a
saving throw against the confusion spell. The confusion effect lasts for
1 minute.\tabularnewline
91--95 & The character loses the ability to speak.\tabularnewline
96--100 & The character falls unconscious. No amount of jostling or
damage can wake the character.\tabularnewline
\bottomrule
\end{longtable}

\hypertarget{indefinite-madness}{%
\paragraph{Indefinite Madness}\label{indefinite-madness}}

\begin{longtable}[]{@{}ll@{}}
\toprule
d100 & Flaw (lasts until cured)\tabularnewline
\midrule
\endhead
01--15 & ``Being drunk keeps me sane.''\tabularnewline
16--25 & ``I keep whatever I find.''\tabularnewline
26--30 & ``I try to become more like someone else I know---adopting his
or her style of dress, mannerisms, and name.''\tabularnewline
31--35 & ``I must bend the truth, exaggerate, or outright lie to be
interesting to other people.''\tabularnewline
36--45 & ``Achieving my goal is the only thing of interest to me, and
I'll ignore everything else to pursue it.''\tabularnewline
46--50 & ``I find it hard to care about anything that goes on around
me.''\tabularnewline
51--55 & ``I don't like the way people judge me all the
time.''\tabularnewline
56--70 & ``I am the smartest, wisest, strongest, fastest, and most
beautiful person I know.''\tabularnewline
71--80 & ``I am convinced that powerful enemies are hunting me, and
their agents are everywhere I go. I am sure they're watching me all the
time.''\tabularnewline
81--85 & ``There's only one person I can trust. And only I can see this
special friend.''\tabularnewline
86--95 & ``I can't take anything seriously. The more serious the
situation, the funnier I find it.''\tabularnewline
96--100 & ``I've discovered that I really like killing
people.''\tabularnewline
\bottomrule
\end{longtable}

\hypertarget{curing-madness}{%
\subsection{Curing Madness}\label{curing-madness}}

A \emph{calm emotions} spell can suppress the effects of madness, while
a \emph{lesser restoration} spell can rid a character of a short-term or
long-term madness. Depending on the source of the madness, \emph{remove
curse} or \emph{dispel evil} might also prove effective. A \emph{greater
restoration} spell or more powerful magic is required to rid a character
of indefinite madness.

\end{document}
